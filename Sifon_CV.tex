\documentclass[11pt]{article}

\usepackage{bm}
\usepackage{enumerate}
\usepackage{etaremune}
% \usepackage{fullpage}
\usepackage{graphicx}
\usepackage{hyperref}

\addtolength{\oddsidemargin}{-.875in}
\addtolength{\evensidemargin}{-.875in}
\addtolength{\textwidth}{1.75in}
\addtolength{\topmargin}{-1in}
\addtolength{\textheight}{2in}

% For links of references
\hypersetup{colorlinks,
  linkcolor=blue,
  filecolor=blue,
  urlcolor=blue,
  citecolor=blue}

\newcommand\technical[2]{
  \noindent
%  \begin{minipage}{0.03\linewidth}
%    \null
%  \end{minipage}
%  \begin{minipage}{0.97\linewidth}
    {\large\bf #1:} #2\\
%  \end{minipage}\\
  }

\newcommand\sectitle[1]{
  \vspace{0.5cm}
  \noindent
  \textbf{\large #1}\\
  \vspace{-0.2cm}
}


\begin{document}

% \begin{figure}[t]
\begin{minipage}[b]{0.46\linewidth}
\flushleft
% \noindent
\hspace{-0.7cm}
{\bf\huge Crist\'obal Sif\'on}\\\vspace{0.2cm}
\hspace{-0.5cm}{\large Postdoctoral Research Associate\\
\hspace{-0.65cm}        Department of Astrophysical Sciences\\
\hspace{-0.65cm}        Princeton University\\
\hspace{-0.65cm}        Peyton Hall, 4 Ivy Ln, Princeton, NJ 08544}\\
\end{minipage}
\begin{minipage}[b]{0.49\linewidth}
\flushright
{\large E-mail: {\texttt sifon@astro.princeton.edu}\\
        Phone: +1 609 258 2303\\
        \url{http://www.astro.princeton.edu/~sifon/}
        \url{https://github.com/cristobal-sifon/}}
\end{minipage}
\vspace{0.4cm}
\hline


\sectitle{Research Interests}

%{\large
I am an observational astrophysicist. My research focuses on galaxy 
cluster physics including observable--mass scaling relations for cosmological 
analyses, brightest cluster galaxies, the mass content of cluster galaxies and 
diffuse radio emission. I am also interested in intrinsic galaxy alignments, both 
as contaminants for cosmic shear and as a physical mechanism in its own right. I 
use various tools and techniques to study these phenomena, including weak 
gravitational lensing, spectroscopy, and the exploitation of optical surveys in 
general. I have presented my work at various conferences, workshops and seminars 
in Chile, Europe and the US.

\technical{Collaborations}
{Atacama Cosmology Telescope (ACT), Canadian Cluster Comparison Project (CCCP), 
Kilo-Degree Survey (KiDS), Multi-Epoch Nearby Cluster Survey (MENeaCS), 
Hyper-Suprime Cam Survey (HSC).}

\vspace{-0.5cm}
\sectitle{Employment and Education}

\noindent
\emph{Present} --- Postdoctoral Research Associate, Princeton University\\
\emph{2016} --- Ph.D.~Astrophysics, Universiteit Leiden, The Netherlands\\
\emph{2012} --- M.Sc.~Astrophysics, P.~Universidad Cat\'olica de Chile, Chile\\
\emph{2010} --- B.Sc.~Astronomy, P.~Universidad Cat\'olica de Chile, Chile


\sectitle{Observing Proposals (as PI)}

\noindent
\emph{VLT Survey Telescope} -- 6h for weak lensing observations of a massive 
galaxy cluster\\
\emph{Gemini-South} -- 24 h for imaging and spectroscopic follow-up of massive 
high-redshift cluster candidates\\
\emph{Giant Metrewave Radio Telescope} -- 44 h to search for diffuse radio 
emission in massive galaxy clusters\\


\technical{Observing Experience}
{I have spent roughly 180 hours observing in 
optical (Gemini-South) and near-infrared (ESO-NTT) telescopes performing both 
imaging and spectroscopy.}\\

\vspace{-0.5cm}
\technical{Technical skills}
{I am an experienced \texttt{python} programmer, and I also have some experience 
with IRAF/PyRAF. I have written {\tt pygmos}, a Python/PyRAF pipeline to reduce 
Gemini-GMOS spectra which is available 
\href{https://github.com/cristobal-sifon/pygmos/}{\texttt{here}}. Other codes I 
have written are posted at my 
\href{https://github.com/cristobal-sifon}{\texttt{github}} page.}


%\sectitle{Language skills}
%
%{Native Spanish, fluent English, basic Dutch}

\technical{Community Activity}
{I have served as a referee for A\&A, ApJ, and Nature Astronomy.}


%\pagebreak
\vspace{-0.5cm}
\sectitle{Teaching Assistant Experience}

\noindent
\emph{2013-B} -- Stellar Dynamics (Leiden, Prof.~S.~Portegies Zwart)\\
\emph{2012-A} -- Extragalactic Astrophysics (PUC, Prof.~L.~F.~Barrientos)\\
\emph{2011-A} -- Extragalactic Astrophysics (PUC, Prof.~L.~F.~Barrientos)\\
\emph{2011-A} -- Laboratory of Thermodynamics and Kinetic Theory (PUC, Prof.~U.~Volkmann)\\
\emph{2010-B} -- Experimental Astrophysics (PUC, Prof.~L.~F.~Barrientos)


\sectitle{Other Work Experience}

\noindent
\emph{2007 --- 2008} -- Ski instructor at Homewood Mountain Ski Resort in Lake Tahoe, CA. Obtained the certification as Level I ski instructor  by the Professional Ski Instructors of America (PSIA).\\
\emph{2006 --- 2007} -- Lift operator at Sun Valley Resort, Sun Valley, ID.

\vspace{1cm}
\hline
\vspace{1cm}

\noindent
{\bf\Large References}
% \vspace{-0.6cm}
\begin{itemize}
%  \item 
 \item Prof.~David Spergel\\
       Department of Astrophysical Sciences\\
       Princeton University\\
       4 Ivy Ln, Princeton, NJ 08544, USA\\
       E-mail: dns@astro.princeton.edu
 \item Prof.~Henk Hoekstra (\textit{PhD advisor})\\
       Leiden Observatory\\
       Universiteit Leiden\\
       Niels Bohrweg 2, NL-2333 CA Leiden, The Netherlands\\
       E-mail: hoekstra@strw.leidenuniv.nl
 \item Prof.~John Hughes\\
       Department of Physics and Astronomy\\
       Rutgers University\\
       136 Frelinghuysen Rd., Piscataway, NJ 08854, USA\\
       E-mail: jph@physics.rutgers.edu
 \item Dr.~Felipe Menanteau\\
       Department of Astronomy\\
       University of Illinois at Urbana-Champaign\\
       1002 W.\ Green St., Urbana, IL 61801, USA\\
       E-mail: felipe@illinois.edu
 \item Prof.~L.~Felipe Barrientos (\textit{MSc advisor})\\
       Departamento de Astronom\'ia y Astrof\'isica\\
       P. Universidad Cat\'olica de Chile\\
       Casilla 306, Santiago 22, Chile\\
       E-mail: barrientos@astro.puc.cl
\end{itemize}


%%%%%%%%%%%%%%%%%%%%%%%%%%%%%%%%%%%%%%
%%%%%%%%%%%%%%%%%%%%%%%%%%%%%%%%%%%%%%
%%%%%%%%%% PUBLICATION LIST %%%%%%%%%%
%%%%%%%%%%%%%%%%%%%%%%%%%%%%%%%%%%%%%%
%%%%%%%%%%%%%%%%%%%%%%%%%%%%%%%%%%%%%%

%% Uncomment these two lines to get the publication list in the main pdf

% \pagebreak
% \documentclass{article}

\usepackage{color}
\usepackage{etaremune}
\usepackage{fullpage}
\usepackage{hyperref}

\hypersetup{
    colorlinks,
    linkcolor=blue,
    urlcolor=blue
    }

\def\myself{\textbf{\color{red} C.~Sif\'on}}

\def\aap{A\&A}
\def\apj{ApJ}
\def\baas{\textit{Bull.\ of the Am.\ Ast.\ Soc.}}
\def\gemfoc{Gemini Focus}
\def\jcap{JCAP}
\def\mnras{MNRAS}
\def\msngr{The Messenger}
\def\pasj{PASJ}
\def\prd{Phys.\ Rev.\ D}
\def\prl{Phys.\ Rev.\ Letters}
\def\ssr{Space Sci.\ Rev.}

\newcommand{\accepted}[1]{accepted for publication in #1}
\newcommand{\submitted}[1]{submitted to #1}
\newcommand{\paper}[1]{\textbf{``#1''}}

% \pagestyle{empty}

\begin{document}

%%% ADS custom formats %%%
% Refereed:
% %N,\n\{\\bf "%T"\},\n%Y,\\href{\%u\}\{%q,%V,%P\}
% arXiv:
% %N,\n\{\\bf "%T"\},%Y,\\href{\%u\}\{%Q\}
% 
% New version:
% %I,\n\{\\bf ``%T''\},\n%Y,\\href{\%u\}\{%q,%V,%P\}
%%% or use the file $HOME/Documents/papers/authorlist.py to get the author list formatted
%%% this way from the pdf

\noindent
{\bf\huge Publication list}\\

\noindent
I have co-authored 44 papers intended for publication in peer-reviewed journals, 
including 7 first-author papers. They have been cited more than 1,200 times and 
have an $h$-index of 20, with more than 150 citations on my first-author papers. 
They also include three companion reviews on galaxy alignments written for a 
special issue of Space Science Reviews (B.\ Joachimi et al.\ 2015, A.\ Kiessling 
et al.\ 2015, D.\ Kirk et al.\ 2015). The full list of publications is summarized 
below, and can be accessed at \href{https://goo.gl/LAu9G4}{this url}.

\vspace{0.4cm}
\noindent
{\bf\Large First-Author Papers}\\

\newcommand{\etal}[1]{et al.\ (#1 co-authors)}

\vspace{-0.5cm}
\begin{etaremune}

\item
\myself, R.~Herbonnet, H.~Hoekstra, R.~F.~J.~van~der~Burg, M.~Viola,
\paper{The galaxy-subhalo connection in low-redshift galaxy clusters from
weak gravitational lensing},
2017, \href{http://adsabs.harvard.edu/abs/2017arXiv170606125S}{arXiv:1706.06125},
\submitted{\mnras}

\item
\myself, R.~F.~J.~van~der~Burg, H.~Hoekstra, A.~Muzzin, R.~Herbonnet,
\paper{A First Constraint on the Average Mass of Ultra Diffuse Galaxies from 
Weak Gravitational Lensing},
2017, \href{http://adsabs.harvard.edu/abs/2017arXiv170407847S}{arXiv:1704.07847},
\submitted{\mnras}

\item 
\myself, N.~Battaglia, M.~Hasselfield, \etal{25},
% F.~Menanteau, L.~F.~Barrientos, J.~R.~Bond, D.~Crichton, 
% M.~J.~Devlin, R.~D\"unner, M.~Hilton, A.~D.~Hincks, R.~Hlozek, K.~M.~Huffenberger, J.~P.~Hughes, 
% L.~Infante, A.~Kosowsky, D.~Marsden, T.~A.~Marriage, K.~Moodley, M.~D.~Niemack, L.~A.~Page, 
% D.~N.~Spergel, S.~T.~Staggs, H.~Trac, E.~J.~Wollack,
\paper{The Atacama Cosmology Telescope: Dynamical Masses for 44 SZ-Selected Galaxy Clusters over 
755 Square Degrees},
2016, \href{http://adsabs.harvard.edu/abs/2016MNRAS.461..248S}{\mnras, 461, 248} 

\item
\myself, M.~Cacciato, H.~Hoekstra, \etal{26},
% M.~Brouwer, E.~van~Uitert, M.~Viola, I.~K.~Baldry, 
% S.~Brough, M.~J.~I.~Brown, A.~Choi, S.~P.~Driver, T.~Erben, A.~Grado, C.~Heymans, H.~Hildebrandt, 
% B.~Joachimi, J.~T.~A.~de~Jong, K.~Kuijken, J.~McFarland, L.~Miller, R.~Nakajima, N.~Napolitano, 
% P.~Norberg, A.~S.~G.~Robotham, P.~Schneider, G.~Verdoes~Kleijn,
\paper{The Masses of Satellites in GAMA Galaxy Groups from 100 Square Degrees of KiDS Weak Lensing 
Data},
2015, \href{https://adsabs.harvard.edu/abs/2015MNRAS.454.3938S}{\mnras, 454, 3938}

\item
\myself, H.~Hoekstra, M.~Cacciato, M.~Viola, F.~K\"ohlinger, R.~F.~J.~van~der~Burg, D.~J.~Sand, 
M.~L.~Graham,
\paper{Constraints on the Alignments of Galaxies in Galaxy Clusters from $\sim$14,000 Spectroscopic 
Members},
2015, \href{https://adsabs.harvard.edu/abs/2015A&A...575A..48S}{\aap, 575, A48}

\item
\myself, F.~Menanteau, J.~P.~Hughes, M.~Carrasco, L.~F.~Barrientos,
\paper{Strong Lensing Analysis of PLCK~G004.5$-$19.5, a Planck-Discovered Cluster Hosting a Radio 
Relic at $z=0.52$},
2014, \href{https://adsabs.harvard.edu/abs/2014A&A...562A..43S}{\aap, 562, A43}

\item 
\myself, F.~Menanteau, M.~Hasselfield, \etal{36},
% T.~A.~Marriage, J.~P.~Hughes, L.~F.~Barrientos, 
% J.~Gonz\'alez, L.~Infante, G.~E.~Addison, A.~J.~Baker, N.~Battaglia, J.~R.~Bond, D.~Crichton, 
% S.~Das, M.~J.~Devlin, J.~Dunkley, R.~D\"unner, M.~B.~Gralla, A.~Hajian, M.~Hilton, A.~D.~Hincks, 
% A.~B.~Kosowsky, D.~Marsden, K.~Moodley, M.~D.~Niemack, M.~R.~Nolta, L.~A.~Page, B.~Partridge, 
% E.~D.~Reese, N.~Sehgal, J.~Sievers, D.~N.~Spergel, S.~T.~Staggs, R.~J.~Thornton, H.~Trac, 
% E.~J.~Wollack,
\paper{The Atacama Cosmology Telescope: Dynamical Masses and Scaling Relations for a Sample of
Massive Sunyaev-Zel'dovich Effect Selected Galaxy Clusters}, 
2013, \href{https://adsabs.harvard.edu/abs/2013ApJ...772...25S}{\apj, 772, 25}

\end{etaremune}


%%%%%%%%%%%%%%%%%%%%%
%%%%%%%%%%%%%%%%%%%%%


\vspace{0.4cm}
\noindent
{\bf\Large Major Contributor Papers}\\

\vspace{-0.5cm}
\begin{etaremune}

\item
R.~F.~J.~van~der~Burg, H.~Hoekstra, A.~Muzzin, \myself, \etal{17},
% M.~Viola, M.~N.~Bremer, S .~Brough, S.~P.~Driver;, C.~Heymans,
% H.~Hildebrandt, B.~W.~Holwerda, K.~Kuijken, 
% S.~Mcgee, R.~Nakajima, N.~Napolitano, E.~N.~Taylor, E.~Valentijn,
\paper{The Abundance of Ultra-diffuse Galaxies from Groups to Clusters: UDGs 
are Relatively More Common in More Massive Haloes},
2017, \href{http://adsabs.harvard.edu/abs/2017arXiv170602704V}{arXiv:1706.02704},
\submitted{\aap}

\item
E.~Medezinski, N.~Battaglia, K.~Umetsu, M.~Oguri, H.~Miyatake, A.~Nishizawa,
\myself, D.~N.~Spergel, I-N.~Chiu, Y.-T.~Lin, N.~Bahcall, Y.~Komiyama,
\paper{Planck Sunyaev-Zel’dovich Cluster Mass Calibration using Hyper Suprime-Cam Weak
Lensing},
2017, \href{http://adsabs.harvard.edu/abs/2017arXiv170600434M}{arXiv:1706.00434},
\submitted{\pasj}

\item
E.~van~Uitert, M.~Cacciato, H.~Hoekstra, M.~Brouwer, \myself, \etal{29},
% M.~Viola, I.~Baldry, J.~Bland-hawthorn, S.~Brough, M.~J.~I.~Brown, A.~Choi, S.~P.~Driver, T.~Erben, C.~Heymans, H.~Hildebrandt, B.~J.~, K.~Kuijken, J.~Liske, J.~Loveday, J.~Mcfarland, L.~Miller, R.~Nakajima, J.~Peacock, M.~Radovich, A.~S.~G.~Robotham, P.~Schneider, G.~Sikkema, E.~N.~Taylor, G.~V.~Kleijn,
\paper{The Stellar-to-Halo Mass Relation of GAMA Galaxies from 100 Square Degrees of KiDS Weak Lensing Data},
2016, \href{http://adsabs.harvard.edu/abs/2016MNRAS.459.3251V}{\mnras, 459, 3251}

\item
D.~Kirk, M.~L.~Brown, H.~Hoekstra, B.~Joachimi, T.~D.~Kitching, R.~Mandelbaum, \myself, 
M.~Cacciato, A.~Choi, A.~Kiessling, A.~Leonard, A.~Rassat, B.~Malte~Sch\"afer,
\paper{Galaxy Alignments: Observations and Impact on Cosmology},
2015, \href{http://adsabs.harvard.edu/doi/10.1007/s11214-015-0213-4}{\ssr, 193, 139}

\item
A.~Kiessling, M.~Cacciato, B.~Joachimi, D.~Kirk, T.~D.~Kitching, A.~Leonard, R.~Mandelbaum, 
B.~Malte~Sch\"afer, \myself, M.~L.~Brown, A.~Rassat
\paper{Galaxy Alignments: Theory, Modelling \& Simulations},
2015, \href{http://adsabs.harvard.edu/doi/10.1007/s11214-015-0203-6}{\ssr, 193, 67}

\item
B.~Joachimi, M.~Cacciato, T.~D.~Kitching, A.~Leonard, R.~Mandelbaum, B.~Malte~Sch\"afer, \myself, 
H.~Hoekstra, A.~Kiessling, D.~Kirk, A.~Rassat,
\paper{Galaxy Alignments: an Overview},
2015, \href{http://adsabs.harvard.edu/doi/10.1007/s11214-015-0177-4}{\ssr, 193, 1}

\item
R.~F.~J.~van~der~Burg, H.~Hoekstra, A.~Muzzin, \myself, M.~L.~Balogh, S.~McGee,
\paper{Evidence for the Inside-Out Growth of the Stellar Mass Distribution in Galaxy Clusters 
since $z\sim1$},
2015, \href{http://adsabs.harvard.edu/adsabs/abs/2015A&A...577A..19V}{\aap, 577, 19}

\item
M.~Hilton, M.~Hasselfield, \myself, \etal{26},
% A.~J.~Baker, L.~F.~Barrientos, N.~Battaglia, J.~R.~Bond, 
% D.~Crichton, M.~J.~Devlin, M.~Gralla, K.~D.~Irwin, A.~Hajian, A.~D.~Hincks, J.~P.~Hughes, 
% L.~Infante, Y.-T.~Lin, T.~A.~Marriage, D.~Marsden, F.~Menanteau, K.~Moodley, M.~R.~Nolta, 
% L.~A.~Page, E.~D.~Reese, J.~Sievers, D.~N.~Spergel, E.~J.~Wollack,
\paper{The Atacama Cosmology Telescope: The Stellar Content of Galaxy Clusters Selected Using the 
Sunyaev-Zel'dovich Effect},
2013, \href{https://ui.adsabs.harvard.edu/#abs/2013MNRAS.435.3469H/abstract}{\mnras, 435, 3469}

\item
F.~Menanteau, \myself, L.~F.~Barrientos, \etal{26},
% N.~Battaglia, D.~Crichton, M.~J.~Devlin, R.~D\"unner, 
% M.~Gralla, A.~Hajian, M.~Hasselfield, M.~Hilton, A.~D.~Hincks, J.~P.~Hughes, L.~Infante, 
% A.~Kosowsky, D.~Marsden, T.~A.~Marriage, K.~Moodley, M.~D.~Niemack, L.~A.~Page, B.~Partridge, 
% E.~D.~Reese, J.~Sievers, D.~N.~Spergel, S.~T.~Staggs, E.~J.~Wollack,
\paper{The Atacama Cosmology Telescope: Physical Properties of Sunyaev-Zel'dovich Effect Clusters 
on the Celestial Equator},
2013, \href{https://ui.adsabs.harvard.edu/#abs/2013ApJ...765...67M/abstract}{\apj, 765, 67}

\item
F.~Menanteau, J.~P.~Hughes, \myself, \etal{27},
% M.~Hilton, J.~Gonz\'alez, L.~Infante, L.~F.~Barrientos, 
% A.~J.~Baker, J.~R.~Bond, S.~Das, M.~J.~Devlin, J.~Dunkley, A.~Hajian, A.~D.~Hincks, A.~Kosowsky, 
% D.~Marsden, T.~A.~Marriage, K.~Moodley, M.~D.~Niemack, M.~R.~Nolta, L.~A.~Page, E.~D.~Reese, 
% N.~Sehgal, J.~Sievers, D.~N.~Spergel, S.~T.~Staggs, E.~J.~Wollack,
\paper{The Atacama Cosmology Telescope: ACT-CL J0102--4915 ``El Gordo,'' a Massive
Merging Cluster at Redshift 0.87},
2012, \href{https://ui.adsabs.harvard.edu/#abs/2012ApJ...748....7M/abstract}{\apj, 748, 7}

\end{etaremune}


%%%%%%%%%%%%%%%%%%%%%%%%%%%%%%%%%
%%%%%%%%%%%%%%%%%%%%%%%%%%%%%%%%%


\vspace{0.4cm}
\noindent
{\bf\Large Contributing Author Papers}\\

\vspace{-0.5cm}
\begin{etaremune}

\item
E.~Medezinski, M.~Oguri, A.~Nishizawa, \etal{16}
\paper{Source Selection for Cluster Weak Lensing Measurements in the Hyper Suprime-Cam
Survey},
2017, \href{http://adsabs.harvard.edu/abs/2017arXiv170600427M}{arXiv:1706.00427},
\submitted{\pasj}

\item
R.~Mandelbaum, H.~Miyatake, T.~Hamana, \etal{27}
% M.~Simet, R.~Armstrong, J.~Bosch, F.~Lanusse, 
% A.~Leauthaud, J.~Coupon, M.~Takada, S.~Miyazaki, J.~S.~Speagle, M.~Shirasaki, \myself, 
% S.~Huang, A.~J., E.~Medezinski, Y.~Okura, N.~Okabe, N.~Czakon, R.~Takahashi, W.~Coulton, 
% C.~Hikage, Y.~Komiyama, R.~H.~Lupton, M.~A.~Strauss, T.~Y.~Utsumi,
\paper{The First-Year Shear Catalog of the Subaru Hyper Suprime-Cam SSP Survey},
2017, \href{http://adsabs.harvard.edu/abs/2017arXiv170506745M}{arXiv:1705.06745},
\submitted{\pasj}


\item
A.~Dvornik, 1.~M.~Cacciato, K.~Kuijken, \etal{22}
% M.~Viola, H.~Hoekstra, R.~Nakajima, 
% E.~V.~Uitert, M.~Brouwer, A.~Choi, T.~Erben, I.~F.~Conti, 4.~D.~J.~Farrow, 
% R.~Herbonnet, C.~Heymans, H.~Hildebrandt, A.~M.~Hopkins, J.~Mcfarland, 
% P.~Norberg, P.~Schneider, \myself, E.~Valentijn, L.~Wang,
\paper{A KiDS Weak Lensing Analysis of Assembly Bias in GAMA Galaxy Groups},
2017, \href{http://adsabs.harvard.edu/abs/2017arXiv170306657D}{arXiv:1703.06657},
\submitted{\mnras}


\item
M.~M.~Brouwer, M.~R.~Visser, A.~Dvornik, \etal{22},
% H.~Hoekstra, K.~Kuijken, 
% E.~A.~Valentijn, M.~Bilicki, C.~Blake, S.~Brough, H.~Buddelmeijer, T.~Erben, 
% C.~Heymans, H.~Hildebrandt, B.~W.~Holwerda, A.~M.~Hopkins, D.~Klaes, J.~Liske, 
% J.~Loveday, J.~McFarland, R.~Nakajima, \myself, E.~N.~Taylor,
\paper{First Test of Verlinde's Theory of Emergent Gravity Using Weak 
Gravitational Lensing Measurements},
2017, \href{http://adsabs.harvard.edu/abs/2017MNRAS.466.2547B}{\mnras, 466, 2547}


\item M.~Velliscig, M.~Cacciato, H.~Hoekstra, \etal{17}
% J.~Schaye, C.~Heymans, H.~Hildebrandt, J.~Loveday, P.~Norberg, \myself,
% P.~Schneider, E.~van~Uitert, M.~Viola, S.~Brough, T.~Erben, B.~W.~Holwerda,
% A.~M.~Hopkins, K.~Kuijken,
\paper{Galaxy-Galaxy Lensing in EAGLE: Comparison with Data from 180 Square
Degrees of the KiDS and GAMA Surveys},
2016, \href{http://adsabs.harvard.edu/abs/2016arXiv161204825V}{arXiv:1612.04825},
\submitted{\mnras}


\item
M.~M.~Brouwer, M.~Cacciato, A.~Dvornik, \etal{36},
% L.~Eardley, C.~Heymans, H.~Hoekstra, K.~Kuijken, T.~Mcnaught-roberts, \myself, M.~Viola, M.~Alpaslan, M.~Bilicki, J.~B.~, S.~Brough, A.~Choi, S.~P.~Driver;, T.~Erben, A.~Grado, H.~Hildebrandt, B.~W.~Holwerda, A.~M.~H.~, J.~T.~A.~D.~Jong, J.~Liske, J.~Mcfarland, R.~Nakajima, N.~R.~N.~P.~Norberg, J.~A.~Peacock, M.~Radovich, A.~S.~G.~Robotham, P.~Schneider, G.~Sikkema, E.~V.~Uitert, G.~Verdoes~Kleijn, 
\paper{Dependence of GAMA Galaxy Halo Masses on the Cosmic Web Environment from 100 square degrees of KiDS Weak Lensing Data},
2016, \href{http://adsabs.harvard.edu/abs/2016MNRAS.462.4451B}{\mnras, 462, 4451}


\item
N.~Battaglia, A.~Leauthaud, H.~Miyatake, \etal{39},
% M.~Hasselfield, M.~B.~Gralla, R.~Allison, J.~R.~Bond, 
% D.~Crichton, M.~J.~Devlin, J.~Dunkley, R.~D\"unner, T.~Erben, S.~Ferrara, M.~Halpern, J.~C.~Hill, 
% A.~D.~Hincks, R.~Hlozek, K.~M.~Huffenberger, J.~P.~Hughes, J.~P.~Kneib, M.~Makler, 
% T.~A.~Marriage, % F.~Menanteau;, L.~Miller, K.~Moodley, B.~Moraes;, D.~Niemack, L.~Page, H.~Shan, 
% N.~Sehgal, % B.~D.~Sherwin, J.~L.~Sievers, \myself, D.~N.~Spergel, T.~Staggs, J.~Taylor, 
% R.~Thornton, % L.~V.~Waerbeke, E.~J.~Wollack,
\paper{Weak-Lensing Mass Calibration of the Atacama Cosmology Telescope Equatorial 
Sunyaev-Zel'dovich Cluster Sample with the Canada-France-Hawaii Telescope Stripe 82 Survey},
2016, \href{http://adsabs.harvard.edu/abs/2016JCAP...08..013B}{\jcap, 08, 013}


\item
S.~Bellstedt, C.~Lidman, A.~Muzzin, \etal{16},
% M.~Franx, S.~Guatelli, A.~R.~Hill, H.~Hoekstra, N.~Kurinsky;, I.~Labbe, D.~Marchesini, Z.~C.~Marsan, M.~Safavi-naeini, \myself, M.~Stefanon, J.~V.~D.~Sande, P.~V.~D.~C.~Weigel,
\paper{The Evolution In the Stellar Mass of Brightest Cluster Galaxies over the Past 10 Billion Years},
2016, \href{http://adsabs.harvard.edu/abs/2016MNRAS.460.2862B}{\mnras, 460, 2862}


\item
K.~Knowles, H.~T.~Intema, A.~J.~Baker, \etal{21},
% J.~R.~Bond, C.~Cress, N.~Gupta, A.~Hajian, M.~Hilton, 
% A.~D.~Hincks, R.~Hlozek, J.~P.~Hughes, R.~R.~Lindner, T.~A.~Marriage, F.~Menanteau, K.~Moodley, 
% M.~D.~Niemack, E.~D.~Reese, J.~Sievers, \myself, R.~Srianand, E.~J.~Wollack,
\paper{A Giant Radio Halo in a Low-Mass SZ-Selected Galaxy Cluster: ACT-CL~J0256.5+0006},
2016, \href{http://adsabs.harvard.edu/adsabs/abs/2016MNRAS.459.4240K}{\mnras, 459, 4240}

\item
D.~Crichton, M.~B.~Gralla;, K.~Hall, \etal{22},
% T.~A.~Marriage, N.~L.~Zakamska, N.~Battaglia, J.~R.~Bond, 
% M.~J.~Devlin, J.~C.~Hill, M.~Hilton, A.~D.~Hincks, K.~M.~Huenberger, J.~P.~Hughes, A.~Kosowsky, 
% K.~Moodley, M.~D.~Niemack, L.~A.~Page, B.~Partridge, J.~L.~Sievers, \myself, S.~T.~Staggs, 
% M.~P.~V.~E.~J.~Wollack,
\paper{Evidence for the Thermal Sunyaev-Zel'dovich Effect Associated with Quasar Feedback},
2016, \href{http://adsabs.harvard.edu/abs/2016MNRAS.458.1478C}{\mnras, 458, 1478},

\item
J.~T.~A.~de~Jong, G.~A.~Verdoes~Kleijn, D.~R.~Boxhoorn, \etal{49},
% H.~Buddelmeijer, M.~Capaccioli, 
% F.~Getman, A.~Grado, E.~Helmich, Z.~Huang, N.~Irisarri, K.~H.~Kuijken, F.~Labarbera, 
% J.~P.~McFarland, N.~R.~Napolitano, M.~Radovich, G.~Sikkema, E.~A.~Valentijn, K.~G.~Begeman, 
% M.~Brescia, S.~Cavuoti, A.~Choi, O.~Cordes, G.~Covone, M.~Dall’ora, H.~Hildebrandt, G.~Longo, 
% R.~Nakajima, M.~Paolillo, E.~Puddu, A.~Rifatto, C.~Tortora, E.~van~Uitert, A.~Buddendiek, 
% J.~Harnois-D\'eraps, T.~Erben, M.~B.~Eriksen, C.~Heymans, H.~Hoekstra, B.~Joachimi, 
% T.~D.~Kitching, D.~Klaes, L.~V.~E.~Koopmans, F.~K\"ohlinger, N.~Roy, \myself, P.~Schneider, 
% W.~J.~Sutherland, M.~Viola, W.~Vriend,
\paper{The First and Second Data Releases of the Kilo Degree Survey},
2015, \href{http://adsabs.harvard.edu/abs/2015A&A...582A..62D}{\aap, 582, 62}

\item
K.~Kuijken, C.~Heymans, H.~Hildebrandt, \etal{35},
% R.~Nakajima, T.~Erben, J.~T.~de~Jong, M.~Viola, 
% A.~Choi, H.~Hoekstra, L.~Miller, E.~van~Uitert, A.~Amon, C.~Blake, M.~Brouwer, A.~Buddendiek, 
% I.~Fenech~Conti, M.~Eriksen, A.~Grado, J.~Harnois-D\'eraps, E.~Helmich, R.~Herbonnet, 
% N.~Irisarri, T.~Kitching, D.~Klaes, F.~Labarbera, N.~Napolitano, M.~Radovich, P.~Schneider, 
% \myself, G.~Sikkema, P.~Simon, A.~Tudorica, E.~Valentijn, G.~Verdoes~Kleijn, L.~van~Waerbeke,
\paper{Gravitational Lensing Analysis of the Kilo Degree Survey},
2015, \href{http://adsabs.harvard.edu/abs/2015MNRAS.454.3500K}{\mnras, 454, 3500}

\item
K.~Y.~Ng, W.~A.~Dawson, D.~Wittman, M.~J.~Jee, J.~P.~Hughes, F.~Menanteau, \myself,
\paper{The Return of the Merging Galaxy Subclusters of El Gordo?},
2015, \href{http://adsabs.harvard.edu/abs/2015MNRAS.453.1531N}{\mnras, 453, 1531}

\item
M.~Viola, M.~Cacciato, M.~Brouwer, \etal{27},
% K.~Kuijken, H.~Hoekstra, P.~Norberg, A.~S.~G,~Robotham, 
% E.~van~Uitert, M.~Alpaslan, I.~K.~Baldry, A.~Choi, J.~T.~A.~de~Jong, S.~P.~Driver, T.~Erben, 
% A.~Grado, A.~W.~Graham, C.~Heymans, H.~Hildebrandt, A.~M.~Hopkins, N.~Irisarri, B.~Joachimi, 
% J.~Loveday, L.~Miller, R.~Nakajima, P.~Schneider, \myself, G.~Verdoes~Kleijn,
\paper{Dark Matter Halo Properties of GAMA Galaxy Groups from 100 Square Degrees of KiDS Weak 
Lensing Data},
2015, \href{http://adsabs.harvard.edu/adsabs/abs/2015MNRAS.452.3528V}{\mnras, 452, 3529}

\item
R.~R.~Lindner, P.~Aguirre, A.~J.~Baker, \etal{25},
% J.~R.~Bond, D.~Crichton, M.~J.~Devlin, T.~Essinger-Hileman, 
% M.~Hilton, A.~D.~Hincks, K.~M.~Huffenberger, J.~P.~Hughes, L.~Infante, M.~Lima, T.~A.~Marriage, 
% F.~Menanteau, M.~D.~Niemack, L.~A.~Page, B.~L.~Schmitt, N.~Sehgal, J.~L.~Sievers, \myself, 
% S.~T.~Staggs, D.~Swetz, A.~Weiss, E.~J.~Wollack,
\paper{The Atacama Cosmology Telescope: the LABOCA/ACT Survey of Clusters at All Redshifts},
2015, \href{http://adsabs.harvard.edu/adsabs/abs/2015ApJ...803...79L}{\apj, 803, 79}

\item
B.~Kirk, M.~Hilton, C.~Cress, \etal{23},
% S.~M.~Crawford, J.~P.~Hughes, N.~Battaglia, J.~R.~Bond, C.~Burke, 
% M.~B.~Gralla, A.~Hajian, M.~Hasselfield, A.~D.~Hincks, L.~Infante, A.~Kosowsky, T.~A.~Marriage, 
% F.~Menanteau, K.~Moodley, M.~D.~Niemack, J.~L.~Sievers, \myself, S.~Wilson, E.~J.~Wollack, 
% C.~Zunckel,
\paper{SALT Spectroscopic Observations of Galaxy Clusters Detected by ACT and a Type II Quasar 
Hosted by a Brightest Cluster Galaxy},
2015, \href{http://adsabs.harvard.edu/adsabs/abs/2015MNRAS.449.4010K}{\mnras, 449, 4010}

\item
L.~Old, R.~Wojtak, G.~A.~Mamon, \etal{24},
% R.~A.~Skibba, F.~R.~Pearce, D.~Croton, S.~Bamford, P.~Behroozi, 
% R.~de~Carvalho, J.~C.~Mu\~noz-Cuartas, D.~Gifford, M.~E.~Gray, A.~von~der~Linden, M.~Merrifield, 
% S.~I.~Muldrew, V.~M\"uller, R.~J.~Pearson, T.~J.~Ponman, E.~Rozo, E.~Rykoff, A.~Saro, T.~Sepp, 
% \myself, E.~Tempel,
\paper{Galaxy Cluster Mass Reconstruction Project: II. Results for Galaxy-Based Techniques with 
Improved Models},
2015, \href{http://adsabs.harvard.edu/adsabs/abs/2015MNRAS.449.1897O}{\mnras, 449, 1897}

\item
M.~B.~Gralla, D.~Crichton, T.~A.~Marriage, \etal{41},
% W.~Mo, P.~Aguirre, G.~E.~Addison, V.~Asboth, 
% N.~Battaglia, J.~Bock, J.~R.~Bond, M.~J.~Devlin, R.~D\"unner, A.~Hajian, M.~Halpern, M.~Hilton, 
% A.~D.~Hincks, R.~A.~Hlozek, K.~M.~Huffenberger, J.~P.~Hughes, R.~J.~Ivison, A.~Kosowsky, 
% Y.-T.~Lin, D.~Marsden, F.~Menanteau, K.~Moodley, G.~Morales, M.~D.~Niemack, S.~Oliver, L.~A.~Page, 
% B.~Partridge, E.~D.~Reese, F.~Rojas, N.~Sehgal, J.~Sievers, \myself, D.~N.~Spergel, S.~T.~Staggs, 
% E.~R.~Switzer, M.~P.~Viero, E.~J.~Wollack, M.~B.~Zemcov,
\paper{A Measurement of the Millimeter Emission and the Sunyaev-Zel'dovich Effect Associated with 
Low-Frequency Radio Sources},
2014, \href{http://adsabs.harvard.edu/adsabs/abs/2014MNRAS.445..460G}{\mnras, 445, 460}

\item
L.~Old, R.~A.~Skibba, F.~R.~Pearce, \etal{21},
% D.~Croton, S.~I.~Muldrew, J.~C.~Mu\~noz-Cuartas, D.~Gifford, 
% M.~E.~Gray, A.~von~der~Linden, G.~A.~Mamon, M.~Merrifield, V.~M\"uller, R.~J.~Pearson, 
% T.~J.~Ponman, A.~Saro, T.~Sepp, \myself, E.~Tempel, E.~Tundo, Y.~O.~Wang, R.~Wojtak,
\paper{Galaxy Cluster Mass Reconstruction Project: I. Methods and First Results on Galaxy-Based 
Techniques},
2014, \href{http://adsabs.harvard.edu/adsabs/abs/2014MNRAS.441.1513O}{\mnras, 441, 1513}

\item
M.~J.~Jee, J.~P.~Hughes, F.~Menanteau, \myself, L.~F.~Barrientos, L.~Infante, R.~Mandelbaum, 
K.~Y.~Ng,
\paper{Weighing ``El Gordo'' with a Precision Scale: Hubble Space Telescope Weak-Lensing Analysis
of the Galaxy Cluster ACT-CL J0102$\textbf{-}$4915 at $z=0.87$},
2014, \href{http://adsabs.harvard.edu/adsabs/abs/2014ApJ...785...20J}{\apj, 785, 20}

\item
M.~Hasselfield, M.~Hilton, T.~A.~Marriage, \etal{44},
% G.~E.~Addison, L.~F.~Barrientos, N.~Battaglia, 
% E.~S.~Battistelli, J.~R.~Bond, D.~Crichton, S.~Das, M.~J.~Devlin, S.~R.~Dicker, J.~Dunkley, 
% R.~D\"unner, J.~W.~Fowler, M.~B.~Gralla, A.~Hajian, M.~Halpern, A.~D.~Hincks, R.~Hlozek, 
% J.~P.~Hughes, L.~Infante, K.~D.~Irwin, A.~Kosowsky, D.~Marsden, F.~Menanteau, K.~Moodley, 
% M.~D.~Niemack, M.~R.~Nolta, L.~A.~Page, B.~Partridge, E.~D.~Reese, B.~L.~Schmitt, N.~Sehgal, 
% B.~D.~Sherwin, J.~Sievers, \myself, D.~N.~Spergel, S.~T.~Staggs, D.~S.~Swetz, E.~R.~Switzer, 
% R.~Thornton, H.~Trac, E.~J.~Wollack,
\paper{The Atacama Cosmology Telescope: Sunyaev-Zel'dovich Selected Galaxy Clusters at 148 GHz from 
Three Seasons of Data},
2013, \href{http://adsabs.harvard.edu/adsabs/abs/2013arXiv1301.0816H}{\jcap, 07, 008}

\item
E.~Calabrese, R.~A.~Hlozek, N.~Battaglia, \etal{34},
% E.~S.~Battistelli, J.~R.~Bond, J.~Chluba, D.~Crichton, 
% S.~Das, M.~J.~Devlin, J.~Dunkley, R.~D\"unner, M.~Farhang, M.~B.~Gralla, A.~Hajian, M.~Halpern, 
% M.~Hasselfield, A.~D.~Hincks, K.~D.~Irwin, A.~Kosowsky, T.~Louis, T.~A.~Marriage, K.~Moodley, 
% L.~Newburgh, M.~D.~Niemack, M.~R.~Nolta, L.~A.~Page, N.~Sehgal, B.~D.~Sherwin, J.~L.~Sievers, 
% \myself, D.~N.~Spergel, S.~T.~Staggs, E.~R.~Switzer, E.~J.~Wollack,
\paper{Cosmological Parameters from Pre-Planck Cosmic Microwave Background Measurements},
2013, \href{http://adsabs.harvard.edu/adsabs/abs/2013arXiv1302.1841C}{\prd, 87, 103012}

\item
N.~Sehgal, G.~E.~Addison, N.~Battaglia, \etal{36},
% E.~S.~Battistelli, J.~R.~Bond, S.~Das, M.~J.~Devlin,
% J.~Dunkley, R.~D\"unner, M.~B.~Gralla, A.~Hajian, M.~Halpern, M.~Hasselfield, M.~Hilton,
% A.~D.~Hincks, R.~A.~Hlozek, J.~P.~Hughes, A.~B.~Kosowsky, Y.-T~Lin, T.~Louis, T.~A.~Marriage,
% D.~Marsden, F.~Menanteau, K.~Moodley, M.~D.~Niemack, L.~A.~Page, B.~Partridge, E.~D.~Reese,
% B.~D.~Sherwin, J.~Sievers, \myself, D.~N.~Spergel, S.~T.~Staggs, D.~S.~Swetz, E.~R.~Switzer, 
% E.~J.~Wollack,
\paper{The Atacama Cosmology Telescope: Relation Between Galaxy Cluster Optical Richness and
Sunyaev-Zel'dovich Effect},
2013, \href{http://adsabs.harvard.edu/adsabs/abs/2013ApJ...767...38S}{\apj, 767, 38}

\item
H.~Miyatake, A.~J.~Nishizawa, M.~Takada, \etal{28},
% R.~Mandelbaum, S.~Mineo, H.~Aihara, D.~N.~Spergel,
% S.~J.~Bickerton, J.~R.~Bond, A.~Hajian, M.~Hilton, A.~D.~Hincks, J.~P.~Hughes, L.~Infante,
% Y.-T.~Lin, R.~H.~Lupton, T.~A.~Marriage, D.~Marsden, F.~Menanteau, S.~Miyazaki, K.~Moodley,
% M.~D.~Niemack, M.~Oguri, P.~A.~Price, E.~D.~Reese, \myself, E.~J.~Wollack, N.~Yasuda,
\paper{Subaru Weak-Lensing Measurement of a $z=0.81$ Cluster Discovered by the Atacama Cosmology
Telescope Survey},
2013, \href{http://adsabs.harvard.edu/adsabs/abs/2013MNRAS.429.3627M}{\mnras, 429, 3627}

\item
B.~D.~Sherwin, S.~Das, A.~Hajian, \etal{31},
% G.~Addison, J.~R.~Bond, D.~Crichton, M.~J.~Devlin, J.~Dunkley,
% M.~B.~Gralla, M.~Halpern, J.~C.~Hill, A.~D.~Hincks, J.~P.~Hughes, K.~Huffenberger, R.~Hlozek,
% A.~Kosowsky, T.~Louis, T.~A.~Marriage, D.~Marsden, F.~Menanteau, K.~Moodley, M.~D.~Niemack,
% L.~A.~Page, E.~D.~Reese, N.~Sehgal, J.~Sievers, \myself, D.~N.~Spergel, S.~T.~Staggs, 
% E.~R.~Switzer, E.~J.~Wollack,
\paper{The Atacama Cosmology Telescope: Cross-correlation of CMB Lensing and Quasars},
2012, \href{http://adsabs.harvard.edu/adsabs/abs/2012PhRvD..86h3006S}{\prd, 86, 083006}

\item 
N.~Hand, G.~E.~Addison, E.~Aubourg,  \etal{58},
% N.~Battaglia, E.~S.~Battistelli, D.~Bizyaev, J.~R.~Bond,
% H.~Brewington, J.~Brinkmann, B.~R.~Brown, S.~Das, K.~S.~Dawson, M.~J.~Devlin, J.~Dunkley,
% R.~D\"unner, D.~J.~Eisenstein, J.~W.~Fowler, M.~B.~Gralla, A.~Hajian, M.~Halpern, M.~Hilton,
% A.~D.~Hincks, R.~Hlozek, J.~P.~Hughes, L.~Infante, K.~D.~Irwin, A.~B.~Kosowsky, Y.-T~Lin,
% E.~Malanushenko, V.~Malanushenko, T.~A.~Marriage, D.~Marsden, F.~Menanteau, K.~Moodley,
% M.~D.~Niemack, M.~R.~Nolta, D.~Oravetz, L.~A.~Page, N.~Palanque-Delabrouille, K.~Pan, 
% E.~D.~Reese,
% D.~J.~Schlegel, D.~P.~Schneider, N.~Sehgal, A.~Shelden, J.~Sievers, \myself, A.~Simmons, 
% S.~Snedden, 
% D.~N.~Spergel, S.~T.~Staggs, D.~S.~Swetz, E.~R.~Switzer, H.~Trac, B.~A.~Weaver, E.~J.~Wollack, 
% C.~Yeche, C.~Zunckel,
\paper{Evidence of Galaxy Cluster Motions with the Kinematic Sunyaev-Zel'dovich Effect}, 
2012, \href{http://adsabs.harvard.edu/adsabs/abs/2012PhRvL.109d1101H}{\prl, 109, 041101}

\item
E.~D.~Reese, T.~Mroczkowski, F.~Menanteau, \etal{44},
% M.~Hilton, J.~Sievers, P.~Aguirre, J.~W.~Appel,
% A.~J.~Baker, J.~R.~Bond, S.~Das, M.~J.~Devlin, S.~R.~Dicker, R.~D\"unner, T.~Essinger-Hileman,
% J.~W.~Fowler, A.~Hajian, M.~Halpern, M.~Hasselfield, J.~C.~Hill, A.~D.~Hincks, K.~M.~Huffenberger,
% J.~P.~Hughes, K.~D.~Irwin, J.~Klein, A.~Kosowsky, Y.~Lin, T.~A.~Marriage, D.~Marsden, K.~Moodley,
% M.~D.~Niemack, M.~R.~Nolta, L.~A.~Page, L.~Parker, B.~Partridge, F.~Rojas, N.~Sehgal, \myself, 
% D.~N.~Spergel, S.~T.~Staggs, D.~S.~Swetz, E.~R.~Switzer, R.~Thornton, H.~Trac, E.~J.~Wollack,
\paper{The Atacama Cosmology Telescope: High-Resolution Sunyaev-Zel'dovich Array Observations of ACT
SZE-selected Clusters from the Equatorial Strip}, 
2012, \href{http://adsabs.harvard.edu/adsabs/abs/2012ApJ...751...12R}{ApJ, 751, 12}


\end{etaremune}



\end{document}


\end{document}

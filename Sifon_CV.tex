\documentclass[11pt]{article}


\usepackage{bm}
\usepackage{enumerate}
\usepackage{etaremune}
% \usepackage{fullpage}
\usepackage{graphicx}
\usepackage{hyperref}

\addtolength{\oddsidemargin}{-.875in}
\addtolength{\evensidemargin}{-.875in}
\addtolength{\textwidth}{1.75in}
\addtolength{\topmargin}{-1in}
\addtolength{\textheight}{2in}

% For links of references
\hypersetup{colorlinks,
  linkcolor=blue,
  filecolor=blue,
  urlcolor=blue,
  citecolor=blue}

\renewcommand{\familydefault}{\sfdefault}

\newcommand\sectitle[1]{
  \hline
  \vspace{0.5cm}
  \noindent
  %\underline{
    \textbf{\Large #1}
  %}
  \\
  \vspace{-0.2cm}
}

\newcommand\subsectitle[1]{
  \vspace{0.3cm}
  \noindent
  %\underline{
    \textbf{\large #1}
  %}
  \\
  \vspace{-0.3cm}
}

\newcommand\technical[2]{
  \noindent
    {\large\bf #1:} #2\\
  }

\newcommand\itemdate[1]{\textbf{[#1]}}
\newcommand\itemdates[2]{\textbf{[#1 -- #2]}}
\newcommand\email[1]{\href{mailto:#1}{\texttt{#1}}}



%not used if publication list not shown
\input{publist_preamble}


\begin{document}

% \begin{figure}[t]
\begin{minipage}[b]{0.46\linewidth}
\flushleft
% \noindent
\hspace{-0.7cm}
{\bf\huge Crist\'obal Sif\'on}\\\vspace{0.2cm}
\hspace{-0.5cm}{\large Postdoctoral Research Associate\\
\hspace{-0.65cm}        Department of Astrophysical Sciences\\
\hspace{-0.65cm}        Princeton University\\
\hspace{-0.65cm}        Peyton Hall, 4 Ivy Ln, Princeton, NJ 08544}\\
\end{minipage}
\begin{minipage}[b]{0.49\linewidth}
\flushright
{\large E-mail: {\texttt sifon@astro.princeton.edu}\\
        Phone: +1 (609) 258 0026\\
        \url{http://www.astro.princeton.edu/~sifon/}
        \url{https://github.com/cristobal-sifon/}}
\end{minipage}
\vspace{0.4cm}
\hline



\sectitle{Research Interests}

%{\large
I am an observational astrophysicist. My research focuses on galaxy cluster physics including observable--mass scaling relations for cosmological analyses, brightest cluster galaxies, the mass content of cluster galaxies and diffuse radio emission. I am also interested in intrinsic galaxy alignments, both as contaminants for cosmic shear and as a physical mechanism in their own right. I use various tools and techniques to study these phenomena, including weak gravitational lensing, spectroscopy, the exploitation of optical surveys in general, and most recently analyses involving hydrodynamical simulations. %I have presented my work at various conferences, workshops and seminars in Chile, Europe and the US.

\vspace{0.5cm}
\technical{Collaborations}
{
 Atacama Cosmology Telescope (ACT) ---
 Canadian Cluster Comparison Project (CCCP) ---
 Galaxy Cluster Mass Reconstruction Project ---
 Hyper-Suprime Cam survey (HSC) ---
 Kilo-Degree Survey (KiDS) ---
 Large Synoptic Survey Telescope Dark Energy Science Collaboration (LSST-DESC) ---
 Multi-Epoch Nearby Cluster Survey (MENeaCS).
}


\sectitle{Employment and Education}

\noindent
\itemdates{2016}{Present} Postdoctoral Research Associate, Princeton University, USA\\
\itemdates{2012}{2016} Ph.D.~Astrophysics, Universiteit Leiden, The Netherlands\\
\itemdates{2010}{2012} M.Sc.~Astrophysics, P.~Universidad Cat\'olica de Chile, Chile\\
\itemdates{2005}{2010} B.Sc.~Astronomy, P.~Universidad Cat\'olica de Chile, Chile


\subsectitle{Internships}

\noindent
\itemdate{2011} Science Intern, Gemini South Observatory\\
\itemdate{2011} Internship, Rutgers University\\
\itemdate{2009} Science Intern, Gemini South Observatory (\emph{B.Sc.\ thesis})\\

%%%

\sectitle{Teaching \& Mentoring}

\subsectitle{Student Mentoring}

\noindent
\itemdates{2018}{Present} Malik Walker, Princeton University: Undegraduate Summer Research Program and Junior Project.\\
\itemdates{2017}{Present} Naomi Robertson, Oxford University (UK): PhD thesis project co-advising.\\
\itemdate{2014} Joshua Albert, Universiteit Leiden: co-advised MSc thesis project. 

\subsectitle{Teaching Assistant}

%\noindent
%\itemdate{2013B} Stellar Dynamics (Leiden, Prof.~S.~Portegies Zwart)\\
%\itemdate{2012A} Extragalactic Astrophysics (PUC, Prof.~L.~F.~Barrientos)\\
%\itemdate{2011A} Extragalactic Astrophysics (PUC, Prof.~L.~F.~Barrientos)\\
%\itemdate{2011A} Laboratory of Thermodynamics and Kinetic Theory (PUC, Prof.~U.~Volkmann)\\
%\itemdate{2010B} Experimental Astrophysics (PUC, Prof.~L.~F.~Barrientos)
\noindent
\itemdate{Leiden} Stellar dynamics\\
\itemdate{U.\ Cat\'olica} Extragalactic astrophysics; Experimental astrophysics; Laboratory of thermodynamics and kinetic theory\\


\pagebreak


\sectitle{Successful Observing Proposals (as PI)}

\noindent
\emph{Gemini South} -- 24 h for imaging and spectroscopy of massive 
high-redshift galaxy clusters\\
\emph{Giant Metrewave Radio Telescope} -- 44 h to search for diffuse radio 
emission in massive galaxy clusters\\
\emph{VLT Survey Telescope} -- 6 h for weak lensing observations of a massive 
galaxy cluster\\


\technical{Observing Experience}
{I have spent roughly 180 hours observing in 
optical (Gemini South) and near-infrared (La Silla-2.2m) telescopes performing both 
imaging and spectroscopy of galaxy clusters.}


\sectitle{Technical skills}

I am an experienced \texttt{python} programmer, and I also have some experience 
with IRAF/PyRAF. I have written {\tt pygmos}, a Python/PyRAF pipeline to reduce 
Gemini-GMOS spectra which is available 
\href{https://github.com/cristobal-sifon/pygmos/}{\texttt{here}}. Other codes I 
have written are posted at my 
\href{https://github.com/cristobal-sifon}{\texttt{github}} page.\\


%\sectitle{Language skills}
%
%{Native Spanish, fluent English, basic Dutch}

\sectitle{Community Activity}

\noindent
I have served as a referee for A\&A, ApJ, MNRAS, and Nature Astronomy.

\subsectitle{Informal courses}

\noindent
\itemdate{2016} \emph{Making Better Figures}, Universiteit Leiden, \url{http://bit.ly/2NTznxW}

\subsectitle{Outreach}

\noindent
\itemdate{2018} \emph{Public Astronomical Observations in Spanish}, Princeton University.\\
\itemdate{2012} \emph{Astronomy Course for Seniors}, U.\ Cat\'olica.\\
\itemdate{2011} \emph{Stars Night}, observation nights for elementary and middle school students in social risk organized by ESO-Santiago.\\
\itemdate{2010} Invited talk on board the ``FFG14 Almirante Latorre'' Chilean Navy ship, Valparaiso, Chile.\\
\itemdate{2010} \emph{The Universe}, series of talks for elementary school students in social risk organized by U.\ Cat\'olica.\\


\sectitle{Other Work Experience}

\noindent
\itemdates{2007}{2008} Ski instructor at Homewood Mountain Ski Resort in Lake Tahoe, CA. Obtained the certification as Level I ski instructor  by the Professional Ski Instructors of America (PSIA).\\
\itemdates{2006}{2007} Ski lift operator at Sun Valley Resort, Sun Valley, ID.\\

%\vspace{1cm}
%\hline
\hline
\vspace{0.5cm}

\pagebreak


\sectitle{References}

\begin{itemize}
%  \item 
 \item Prof.~Henk Hoekstra (\textit{PhD advisor})\\
       Leiden Observatory, Universiteit Leiden\\
       Niels Bohrweg 2, NL-2333 CA Leiden, The Netherlands\\
       Phone: +31 (71) 527 5594\\
       E-mail: \email{hoekstra@strw.leidenuniv.nl}
 \item Prof.~David N.~Spergel\\
       Department of Astrophysical Sciences, Princeton University\\
       4 Ivy Ln, Princeton, NJ 08544, USA\\
       Phone: +1 (609) 258 3589\\
       E-mail: \email{dns@astro.princeton.edu}
 \item Prof.~John P.~Hughes\\
       Department of Physics and Astronomy, Rutgers University\\
       136 Frelinghuysen Rd., Piscataway, NJ 08854, USA\\
       Phone: +1 (848) 445 8878\\
       E-mail: \email{jph@physics.rutgers.edu}
 \item Prof.~L.~Felipe Barrientos (\textit{MSc advisor})\\
       Instituto de Astrof\'isica, P. Universidad Cat\'olica de Chile\\
       Casilla 306, Santiago 22, Chile\\
       Phone: +56 (2) 2354 4941\\
       E-mail: \email{barrientos@astro.uc.cl}
 \item Dr.~Felipe Menanteau\\
       Department of Astronomy, University of Illinois at Urbana-Champaign\\
       1002 W.\ Green St., Urbana, IL 61801, USA\\
       Phone: +1 (217) 244 6297\\
       E-mail: \email{felipe@illinois.edu}
\end{itemize}

\vspace{0.3cm}
\hline


%%%%%%%%%%%%%%%%%%%%%%%%%%%%%%%%%%%%%%
%%%%%%%%%%%%%%%%%%%%%%%%%%%%%%%%%%%%%%
%%%%%%%%%% PUBLICATION LIST %%%%%%%%%%
%%%%%%%%%%%%%%%%%%%%%%%%%%%%%%%%%%%%%%
%%%%%%%%%%%%%%%%%%%%%%%%%%%%%%%%%%%%%%

%% Uncomment these two lines to get the publication list in the main pdf

 \pagebreak
 %%% ADS custom formats %%%
% Refereed:
% %N,\n\{\\bf "%T"\},\n%Y,\\href{\%u\}\{%q,%V,%P\}
% arXiv:
% %N,\n\{\\bf "%T"\},%Y,\\href{\%u\}\{%Q\}
% 
% New version:
% %I,\n\{\\bf ``%T''\},\n%Y,\\href{\%u\}\{%q,%V,%P\}
%%% or use the file $HOME/Documents/papers/authorlist.py to get the author list formatted
%%% this way from the pdf

\title{Publication list}


\noindent
I have co-authored 165 scientific articles intended for peer-reviewed 
publication, including 9 first-author papers. They have been cited more than 
9,900 times, with more than 400 citations on my 
first-author papers. The full list of publications can be accessed at the 
\href{https://goo.gl/LAu9G4}{SAO/NASA Astrophysics Data System}.
%
This document is maintained live on
\href{https://github.com/cristobal-sifon/cv/blob/master/Sifon_publications.pdf}{\texttt{github}}.




\section*{First-Author Papers}

\begin{etaremune}
    % \item\n%O,\n\paper{%T},\n%Y, \href{%u}{%j, %V, %p}\n\arxiv{%X}\n

\item
\myself, A.~Finoguenov, C.~P.~Haines, Y.~Jaffé, B.~M.~Amrutha, R.~Demarco, E.~V.~R.~Lima, C.~Lima-Dias, H.~Méndez-Hernández, P.~Merluzzi, A.~Monachesi, G.~S.~M.~Teixeira, N.~Tejos, P.~Araya-Araya, M.~Argudo-Fernández, R.~Baier-Soto, L.~E.~Bilton, C.~R.~Bom, J.~P.~Calderón, L.~P.~Cassarà, J.~Comparat, H.~M.~Courtois, G.~D'Ago, A.~Dupuy, A.~Fritz, R.~F.~Haack, F.~R.~Herpich, E.~Ibar, U.~Kuchner, A.~R.~Lopes, S.~Lopez, E.~Lösch, S.~McGee, C.~Mendes de Oliveira, L.~Morelli, A.~Moretti, D.~Pallero, F.~Piraino-Cerda, E.~Pompei, U.~Rescigno, R.~Smith, A.~V.~Smith Castelli, L.~Sodré Jr, and E.~Tempel,
\paper{CHANCES, The Chilean Cluster Galaxy Evolution Survey: selection and initial characterisation of clusters and superclusters},
2024, \href{https://ui.adsabs.harvard.edu/abs/2024arXiv241113655S}{arXiv:2411.13655}
\accepted{\aap}

\item
\myself\ and J.~Han,
\paper{The history and mass content of cluster galaxies in the EAGLE simulation},
2024, \href{https://ui.adsabs.harvard.edu/abs/2024A&A...686A.163S}{\aap, 686, A163}
\arxiv{2312.12529}

\item
\myself, R.~Herbonnet, H.~Hoekstra, R.~F.~J.~van der Burg, and M.~Viola,
\paper{The galaxy-subhalo connection in low-redshift galaxy clusters from weak gravitational lensing},
2018, \href{https://ui.adsabs.harvard.edu/abs/2018MNRAS.478.1244S}{\mnras, 478, 1244}
\arxiv{1706.06125}

\item
\myself, R.~F.~J.~van der Burg, H.~Hoekstra, A.~Muzzin, and R.~Herbonnet,
\paper{A first constraint on the average mass of ultra-diffuse galaxies from weak gravitational lensing},
2018, \href{https://ui.adsabs.harvard.edu/abs/2018MNRAS.473.3747S}{\mnras, 473, 3747}
\arxiv{1704.07847}

\item
\myself, N.~Battaglia, M.~Hasselfield, F.~Menanteau, L.~F.~Barrientos, J.~R.~Bond, D.~Crichton, M.~J.~Devlin, R.~Dünner, M.~Hilton, A.~D.~Hincks, R.~Hlozek, K.~M.~Huffenberger, J.~P.~Hughes, L.~Infante, A.~Kosowsky, D.~Marsden, T.~A.~Marriage, K.~Moodley, M.~D.~Niemack, L.~A.~Page, D.~N.~Spergel, S.~T.~Staggs, H.~Trac, and E.~J.~Wollack,
\paper{The Atacama Cosmology Telescope: dynamical masses for 44 SZ-selected galaxy clusters over 755 square degrees},
2016, \href{https://ui.adsabs.harvard.edu/abs/2016MNRAS.461..248S}{\mnras, 461, 248}
\arxiv{1512.00910}

\item
\myself, M.~Cacciato, H.~Hoekstra, M.~Brouwer, E.~van Uitert, M.~Viola, I.~Baldry, S.~Brough, M.~J.~I.~Brown, A.~Choi, S.~P.~Driver, T.~Erben, A.~Grado, C.~Heymans, H.~Hildebrandt, B.~Joachimi, J.~T.~A.~de Jong, K.~Kuijken, J.~McFarland, L.~Miller, R.~Nakajima, N.~Napolitano, P.~Norberg, A.~S.~G.~Robotham, P.~Schneider, and G.~Verdoes Kleijn,
\paper{The masses of satellites in GAMA galaxy groups from 100 square degrees of KiDS weak lensing data},
2015, \href{https://ui.adsabs.harvard.edu/abs/2015MNRAS.454.3938S}{\mnras, 454, 3938}
\arxiv{1507.00737}

\item
\myself, H.~Hoekstra, M.~Cacciato, M.~Viola, F.~Köhlinger, R.~F.~J.~van der Burg, D.~J.~Sand, and M.~L.~Graham,
\paper{Constraints on the alignment of galaxies in galaxy clusters from $\sim$~14,000 spectroscopic members},
2015, \href{https://ui.adsabs.harvard.edu/abs/2015A&A...575A..48S}{\aap, 575, A48}
\arxiv{1406.5196}

\item
\myself, F.~Menanteau, J.~P.~Hughes, M.~Carrasco, and L.~F.~Barrientos,
\paper{Strong lensing analysis of PLCK G004.5-19.5, a Planck-discovered cluster hosting a radio relic at z=0.52},
2014, \href{https://ui.adsabs.harvard.edu/abs/2014A&A...562A..43S}{\aap, 562, A43}
\arxiv{1304.0686}

\item
\myself, F.~Menanteau, M.~Hasselfield, T.~A.~Marriage, J.~P.~Hughes, L.~F.~Barrientos, J.~González, L.~Infante, G.~E.~Addison, A.~J.~Baker, N.~Battaglia, J.~R.~Bond, D.~Crichton, S.~Das, M.~J.~Devlin, J.~Dunkley, R.~Dünner, M.~B.~Gralla, A.~Hajian, M.~Hilton, A.~D.~Hincks, A.~B.~Kosowsky, D.~Marsden, K.~Moodley, M.~D.~Niemack, M.~R.~Nolta, L.~A.~Page, B.~Partridge, E.~D.~Reese, N.~Sehgal, J.~Sievers, D.~N.~Spergel, S.~T.~Staggs, R.~J.~Thornton, H.~Trac, and E.~J.~Wollack,
\paper{The Atacama Cosmology Telescope: Dynamical Masses and Scaling Relations for a Sample of Massive Sunyaev-Zel'dovich Effect Selected Galaxy Clusters},
2013, \href{https://ui.adsabs.harvard.edu/abs/2013ApJ...772...25S}{\apj, 772, 25}
\arxiv{1201.0991}


\end{etaremune}


\section*{Major Contributor Papers}

\begin{etaremune}
    \item
M.~Shirasaki, \myself, \etal{15}
\paper{Masses of Sunyaev-Zel'dovich Galaxy Clusters Detected by The Atacama Cosmology
Telescope: Stacked Lensing Measurements with Subaru HSC Year 3 data}
2024, \href{https://ui.adsabs.harvard.edu/abs/2024arXiv240708201S/abstract}{arXiv:2407.08201},
\submitted{\prd}

\item
N.~C.~Robertson, \myself, \etal{23}
\paper{ACT-DR5 Sunyaev-Zel’dovich Clusters: Weak Lensing Mass Calibration with KiDS},
2024, \href{https://ui.adsabs.harvard.edu/abs/2024A&A...681A..87R/abstract}{\aap, 681, 87}
\arxiv{2304.10219}

\item
A.~Dolfi, F.~A.~G\'omez, A.~Monachesi, S.~Varela-Lav\'in, P.~B.~Tissera, \myself, G.~Galaz, 
\paper{Lopsidedness as a Tracer of Early Galactic Assembly History},
2023, \href{https://ui.adsabs.harvard.edu/abs/2023MNRAS.526..567D/abstract}{\mnras, 526, 567}
\arxiv{2306.04639}

\item
M.~Hilton, \myself, \etal{133}
\paper{The Atacama Cosmology Telescope: a Catalog of $>$4000 Sunyaev-Zel’dovich 
Galaxy Clusters},
2021, \href{https://ui.adsabs.harvard.edu/abs/2021ApJS..253....3H/abstract}{\apjs, 253, 3}
\arxiv{2009.11043}

\item
M.~S.~Madhavacheril, \myself, \etal{61}
\paper{The Atacama Cosmology Telescope: Weighing Distant Clusters with the Most 
Ancient Light},
2020, \href{https://ui.adsabs.harvard.edu/abs/2020ApJ...903L..13M/abstract}{\apjl, 903, 13}
\arxiv{2009.07772}

\item
R.~Herbonnet, \myself, H.~Hoekstra, Y.~Bah\'e, R.~F.~J.~van~der~Burg, 
J.-B.~Melin, A.~von~der~Linden, D.~Sand, S.~Kay, D.~Barnes,
\paper{CCCP and MENeaCS: (Updated) Weak-Lensing Masses for 100 Galaxy Clusters},
2020, \href{https://ui.adsabs.harvard.edu/abs/2020MNRAS.497.4684H/abstract}{\mnras, 497, 4684}
\arxiv{1912.04414}

\item
M.~Hilton, M.~Hasselfield, \myself, \etal{43}
\paper{The Atacama Cosmology Telescope: The Two-Season ACTPol Sunyaev-Zel'dovich 
Effect Selected Cluster Catalog},
2018, \href{https://ui.adsabs.harvard.edu/abs/2018ApJS..235...20H}{\apjs, 235, 20}
\arxiv{1709.05600}

\item
J.~G.~Albert, \myself, A.~Stroe, F.~Mernier, H.~T.~Intema, H.~J.~A.~R\"ottgering, 
G.~Brunetti,
\paper{Complex Diffuse Emission in the $z=0.52$ Cluster PLCK G004.5$-$19.5},
2017, \href{https://ui.adsabs.harvard.edu/abs/2017A&A...607A...4A}{\aap, 607, A4}
\arxiv{1708.00789}

\item
R.~F.~J.~van~der~Burg, H.~Hoekstra, A.~Muzzin, \myself, \etal{17}
\paper{The Abundance of Ultra-Diffuse Galaxies from Groups to Clusters: UDGs Are 
Relatively More Common in More Massive Haloes},
2017, \href{https://ui.adsabs.harvard.edu/abs/2017A&A...607A..79V}{\aap, 607, A79}
\arxiv{1706.02704}

\item
E.~van~Uitert, M.~Cacciato, H.~Hoekstra, M.~Brouwer, \myself, \etal{29}
\paper{The Stellar-to-Halo Mass Relation of GAMA Galaxies from 100 Square 
Degrees of KiDS Weak Lensing Data},
2016, \href{https://ui.adsabs.harvard.edu/abs/2016MNRAS.459.3251V}{\mnras, 459, 3251}
\arxiv{1601.06791}

\item
D.~Kirk, M.~L.~Brown, H.~Hoekstra, B.~Joachimi, T.~D.~Kitching, R.~Mandelbaum, 
\myself, M.~Cacciato, A.~Choi, A.~Kiessling, A.~Leonard, A.~Rassat, 
B.~Malte~Sch\"afer,
\paper{Galaxy Alignments: Observations and Impact on Cosmology},
2015, \href{https://ui.adsabs.harvard.edu/abs/2015SSRv..193..139K/abstract}{\ssr, 193, 139}
\arxiv{1504.05465}

\item
A.~Kiessling, M.~Cacciato, B.~Joachimi, D.~Kirk, T.~D.~Kitching, A.~Leonard, 
R.~Mandelbaum, B.~Malte~Sch\"afer, \myself, M.~L.~Brown, A.~Rassat,
\paper{Galaxy Alignments: Theory, Modelling \& Simulations},
2015, \href{https://ui.adsabs.harvard.edu/abs/2015SSRv..193...67K/abstract}{\ssr, 193, 67}
\arxiv{1504.05546}

\item
B.~Joachimi, M.~Cacciato, T.~D.~Kitching, A.~Leonard, R.~Mandelbaum, 
B.~Malte~Sch\"afer, \myself, H.~Hoekstra, A.~Kiessling, D.~Kirk, A.~Rassat,
\paper{Galaxy Alignments: an Overview},
2015, \href{https://ui.adsabs.harvard.edu/abs/2015SSRv..193....1J/abstract}{\ssr, 193, 1}
\arxiv{1504.05456}

\item
R.~F.~J.~van~der~Burg, H.~Hoekstra, A.~Muzzin, \myself, M.~L.~Balogh, S.~McGee,
\paper{Evidence for the Inside-Out Growth of the Stellar Mass Distribution in 
Galaxy Clusters since $z\sim1$},
2015, \href{https://ui.adsabs.harvard.edu/abs/2015A&A...577A..19V}{\aap, 577, 19}
\arxiv{1412.2137}

\item
M.~Hilton, M.~Hasselfield, \myself, \etal{26}
\paper{The Atacama Cosmology Telescope: The Stellar Content of Galaxy Clusters 
Selected Using the Sunyaev-Zel'dovich Effect},
2013, \href{https://ui.adsabs.harvard.edu/abs/2013MNRAS.435.3469H/abstract}{\mnras, 435, 3469}
\arxiv{1301.0780}

\item
F.~Menanteau, \myself, \etal{26}
\paper{The Atacama Cosmology Telescope: Physical Properties of 
Sunyaev-Zel'dovich Effect Clusters on the Celestial Equator},
2013, \href{https://ui.adsabs.harvard.edu/abs/2013ApJ...765...67M/abstract}{\apj, 765, 67}
\arxiv{1210.4048}

\item
F.~Menanteau, J.~P.~Hughes, \myself, \etal{27}
\paper{The Atacama Cosmology Telescope: ACT-CL J0102--4915 ``El Gordo,'' a 
Massive Merging Cluster at Redshift 0.87},
2012, \href{https://ui.adsabs.harvard.edu/abs/2012ApJ...748....7M/abstract}{\apj, 748, 7}
\arxiv{1109.0953}


\end{etaremune}


\section*{Contributing Author Papers {\small (All including \myself)}}

\begin{etaremune}
    
%%%%%%%%%%%%%%%%%
%%% Submitted %%%
%%%%%%%%%%%%%%%%%

\item
V.~Calafut \etal{53}
\paper{The Atacama Cosmology Telescope: Detection of the Pairwise Kinematic 
Sunyaev-Zel’dovich Effect with SDSS Dr15 Galaxies},
2020, \href{https://ui.adsabs.harvard.edu/abs/2021arXic210108374C/abstract}{arXiv:2101.08374}

\item
E.~M.~Vavagiakis \etal{52}
\paper{The Atacama Cosmology Telescope: Probing the Baryon Content of SDSS DR15 
Galaxies with the Thermal and Kinematic Sunyaev-Zel'dovich Effects},
2020, \href{https://ui.adsabs.harvard.edu/abs/2021arXic210108373V/abstract}{arXiv:2101.08373}

\item
N.~C.~Robertson \etal{46}
\paper{Strong Detection of the CMB Lensing $\times$ Galaxy Weak Lensing 
Cross-Correlation from ACT-DR4, \textit{Planck} Legacy and KiDS-1000},
2020, \href{https://ui.adsabs.harvard.edu/abs/2020arXiv201111613R/abstract}{arXiv:2011.11613},
\submitted{\aap}

\item
S.~Adhikari \etal{113}
\paper{Probing Galaxy Evolution in Massive Clusters using ACT and DES: 
Splashback as a Cosmic Clock},
2020, \href{https://ui.adsabs.harvard.edu/abs/2020arXiv200811663A/abstract}{arXiv:2008.11663},
\submitted{\apj}


%%%%%%%%%%%%%%%%
%%% Accepted %%%
%%%%%%%%%%%%%%%%



%%%%%%%%%%%%%%%%%
%%% Published %%%
%%%%%%%%%%%%%%%%%


\item
K.~Knowles \etal{28}
\paper{MERGHERS Pilot: MeerKAT Discovery of Diffuse Emission in Nine Massive 
Sunyaev-Zel'dovich-Selected Galaxy Clusters from ACT},
2021, \href{https://ui.adsabs.harvard.edu/abs/2021MNRAS.504.1749K/abstract}{\mnras, 504, 1749}
\arxiv{2012.15088}


\item
S.~Amodeo \etal{55}
\paper{The Atacama Cosmology Telescope: Modelling the Gas Thermodynamics in BOSS 
CMASS Galaxies from Kinematic and Thermal Sunyaev-Zel’dovich Measurements},
2021, \href{https://ui.adsabs.harvard.edu/abs/2021PhRvD.103f3514A/abstract}{\prd, 103, 063514}
\arxiv{2009.05558}

\item
E.~Schaan \etal{61}
\paper{The Atacama Cosmology Telescope: Combined Kinematic and Thermal 
Sunyaev-Zel’dovich Measurements from BOSS CMASS and LOWZ Halos},
2021, \href{https://ui.adsabs.harvard.edu/abs/2021PhRvD.103f3513S/abstract}{\prd, 103, 063513}
\arxiv{2009.05557}

\item
B.~Fuzia \etal{22}
\paper{The Atacama Cosmology Telescope: SZ-Based Masses and Dust Emission from 
IR-Selected Cluster Candidates in the SHELA Survey},
2021, \href{https://ui.adsabs.harvard.edu/abs/2021MNRAS.502.4026F/abstract}{\mnras, 502, 4026}
\arxiv{2001.09587}

\item
O.~Darwish \etal{45}
\paper{The Atacama Cosmology Telescope: a CMB Lensing Mass Map over 2100 Square 
Degrees of Sky and its Cross-Correlation with BOSS-CMASS Galaxies},
2021, \href{https://ui.adsabs.harvard.edu/abs/2021MNRAS.500.2250D/abstract}{\mnras, 500, 2250}
\arxiv{2004.01139}

\item
S.~Aiola \etal{140}
\paper{The Atacama Cosmology Telescope: DR4 Maps and Cosmological Parameters},
2020, \href{https://ui.adsabs.harvard.edu/abs/2020JCAP...12..047A/abstract}{\jcap, 12, 047}
\arxiv{2007.07288}

\item
S.~Naess \etal{61}
\paper{The Atacama Cosmology Telescope: Arcminute-Resolution Maps of 18,000 
Square Degrees of the Microwave Sky from ACT 2008-2018 Data Combined with Planck},
2020, \href{https://ui.adsabs.harvard.edu/abs/2020JCAP...12..046N/abstract}{\jcap, 12, 046}
\arxiv{2007.07290}

\item
S.~K.~Choi \etal{138}
\paper{The Atacama Cosmology Telescope: a Measurement of the Cosmic Microwave 
Background Power Spectra at 98 and 150 GHz},
2020, \href{https://ui.adsabs.harvard.edu/abs/2020JCAP...12..045C/abstract}{\jcap, 12, 045}
\arxiv{2007.07289}

\item
E.~N.~Taylor \etal{17}
\paper{GAMA+KiDS: Empirical Correlations between Halo Mass and other Galaxy 
Properties near the Knee of the Stellar-to-Halo Mass Relation},
2020, \href{https://ui.adsabs.harvard.edu/abs/2020MNRAS.499.2896T/abstract}{\mnras, 499, 2896}
\arxiv{2006.10040}

\item
Z.~Li \etal{27}
\paper{The Cross Correlation of the ABS and ACT Maps},
2020, \href{https://ui.adsabs.harvard.edu/abs/2020JCAP...09..010L/abstract}{\jcap, 09, 010}
\arxiv{2002.05717}

\item
Y.~Rong \etal{13}
\paper{Intrinsic Morphology Evolution of Ultra-diffuse Galaxies},
2019, \href{https://ui.adsabs.harvard.edu/abs/2020ApJ...899...78R/abstract}{\apj, 899, 78}
\arxiv{1907.10079}

\item
L.~Linke \etal{12}
\paper{KiDS+VIKING+GAMA: Testing Semi-Analytic Models of Galaxy Evolution with 
Galaxy-Galaxy-Galaxy-Lensing},
2020. \href{https://ui.adsabs.harvard.edu/abs/2020A&A...640A..59L/abstract}{\aap, 640, 59}
\arxiv{2005.02419}

\item
M.~Madhavacheril \etal{49}
\paper{The Atacama Cosmology Telescope: Component-Separated Maps of CMB 
Temperature and the Thermal Sunyaev-Zel'dovich Effect},
2020, \href{https://ui.adsabs.harvard.edu/abs/2020PhRvD.102b3534M/abstract}{\prd, 102, 023534}
\arxiv{1911.05717}

\item
T.~Namikawa \etal{55}
\paper{The Atacama Cosmology Telescope: Constraints on Cosmic Birefringence},
2020, \href{https://ui.adsabs.harvard.edu/abs/2020PhRvD.101h3527N/abstract}{\prd, 101, 083527}
\arxiv{2001.10465}

\item
S.~Huang \etal{12}
%A.~Leauthaud, A.~Hearin, P.~Behroozi, C.~Bradshaw, F.~Ardila, J.~Speagle,
%A.~Tenneti, K.~Bundy, J.~Greene, \myself, N.~Bahcall,
\paper{Weak Lensing Reveals a Tight Connection Between Dark Matter Halo Mass and 
the Distribution of Stellar Mass in Massive Galaxies},
2020, \href{https://ui.adsabs.harvard.edu/abs/2020MNRAS.492.3685H/abstract}{\mnras, 492, 3685}
\arxiv{1811.01139}

\item
Q.~Xia \etal{13}
\paper{A Gravitational Lensing Detection of Filamentary Structures Connecting 
Luminous Red Galaxies},
2020, \href{https://ui.adsabs.harvard.edu/abs/2020A&A...633A..89X/abstract}{\aap, 633, 89}
\arxiv{1909.05852}

\item
H.~Hildebrandt \etal{28}
\paper{KiDS+VIKING-450: Cosmic Shear Tomography with Optical+infrared Data},
2020, \href{https://ui.adsabs.harvard.edu/abs/2020A&A...633A..69H/abstract}{\aap, 633, 69}
\arxiv{1812.06076}

\item
J.~S.~Speagle \etal{12}
\paper{Galaxy-Galaxy Lensing in HSC: Validation Tests and the Impact of 
Heterogeneous Spectroscopic Training Sets},
2019, \href{https://ui.adsabs.harvard.edu/abs/2019MNRAS.490.5658S/abstract}{\mnras, 490, 5658}
\arxiv{1906.05876}

\item K.~R.~Hall \etal{25}
\paper{Quantifying the Thermal Sunyaev-Zel'dovich Effect and Excess Millimeter 
Emission in Quasar Environments},
2019, \href{https://ui.adsabs.harvard.edu/abs/2019MNRAS.490.2315H/abstract}{\mnras, 490, 2315}
\arxiv{1907.11731}

\item
A.~K.~Wright \etal{22}
\paper{KiDS+VIKING-450: A New Combined Optical \& Near-IR Dataset for Cosmology 
and Astrophysics},
2019, \href{https://ui.adsabs.harvard.edu/abs/2019A&A...632A..34W/abstract}{\aap, 632, A34}
\arxiv{1812.06077}

\item
C.~Hikage \etal{30}
\paper{Cosmology from Cosmic Shear Power Spectra with Subaru Hyper Suprime-Cam 
First-Year Data},
2019, \href{https://ui.adsabs.harvard.edu/abs/2019PASJ...71...43H}{\pasj, 71, 43}
\arxiv{1809.09148}

\item
H.~Miyatake \etal{58}
\paper{Weak-Lensing Mass Calibration of ACTPol Sunyaev-Zel'dovich Clusters with 
the Hyper Suprime-Cam Survey},
2019, \href{https://ui.adsabs.harvard.edu/abs/2019ApJ...875...63M}{\apj, 875, 63}
\arxiv{1804.05873}

\item
K.~Knowles \etal{14}
\paper{GMRT 610 MHz Observations of Galaxy Clusters in the ACT Equatorial Sample},
2019, \href{https://ui.adsabs.harvard.edu/abs/2019MNRAS.486.1332K}{\mnras, 486, 1332}
\arxiv{1806.09579}

\item
M.~Brouwer \etal{18}
\paper{Studying Galaxy Troughs and Ridges using Weak Gravitational Lensing with 
the Kilo-Degree Survey},
2018, \href{https://ui.adsabs.harvard.edu/abs/2018MNRAS.481.5189B}{\mnras, 481, 5189}
\arxiv{1805.00562}

\item
R.~Wojtak \etal{17}
\paper{Galaxy Cluster Mass Reconstruction Project - IV. Understanding the 
Effects of Imperfect Membership on Cluster Mass Estimation},
2018, \href{https://ui.adsabs.harvard.edu/abs/2018MNRAS.481..324W}{\mnras, 481, 324}
\arxiv{1806.03199}

\item
A.~Jakobs \etal{20}
\paper{Multi-Wavelength Scaling Relations in Galaxy Groups: a Detailed 
Comparison of GAMA and KiDS Observations to BAHAMAS Simulations},
2018, \href{https://ui.adsabs.harvard.edu/abs/2018MNRAS.480.3338J}{\mnras, 480, 3338}
\arxiv{1712.05463}

\item
A.~Dvornik \etal{14}
\paper{Unveiling Galaxy Bias via the Halo Model, KiDS and GAMA},
2018, \href{https://ui.adsabs.harvard.edu/abs/2018MNRAS.479.1240D}{\mnras, 479, 1240}
\arxiv{1802.00734}

\item
J.~P.~Greco \etal{13}
\paper{Illuminating Low-Surface-Brightness Galaxies with the Hyper Suprime-Cam 
Survey},
2018, \href{https://ui.adsabs.harvard.edu/abs/2018ApJ...857..104G}{\apj, 857, 104}
\arxiv{1709.04474}

\item
J.~F.~Wu, P.~Aguirre, A.~J.~Baker, M.~J.~Devlin, M.~Hilton, J.~P.~Hughes,
L.~Infante, R.~R.~Lindner, \myself,
\paper{Herschel and ALMA Observations of Massive SZE-selected Clusters},
2018, \href{https://ui.adsabs.harvard.edu/abs/2018ApJ...853..195W}{\apj, 853, 195}
\arxiv{1712.04540}

\item
E.~Medezinski \etal{16}
\paper{Source Selection for Cluster Weak Lensing Measurements in the Hyper 
Suprime-Cam Survey},
2018, \href{https://ui.adsabs.harvard.edu/abs/2018PASJ...70...30M}{\pasj, 70, 30}
\arxiv{1706.00427}

\item
E.~Medezinski \etal{12}
\paper{Planck Sunyaev-Zel'dovich Cluster Mass Calibration using Hyper 
Suprime-Cam Weak Lensing},
2018, \href{https://ui.adsabs.harvard.edu/abs/2018PASJ...70S..28M}{\pasj, 70, 28}
\arxiv{1706.00434}

\item
R.~Mandelbaum \etal{27}
\paper{The First-Year Shear Catalog of the Subaru Hyper Suprime-Cam SSP Survey},
2018, \href{https://ui.adsabs.harvard.edu/abs/2018PASJ...70S..25M}{\pasj, 70, 25}
\arxiv{1705.06745}

\item
L.~Old \etal{18}
\paper{Galaxy Cluster Mass Reconstruction Project: III. The Impact of Dynamical 
Substructure on Cluster Mass Estimates},
2018, \href{https://ui.adsabs.harvard.edu/abs/2018MNRAS.475..853O}{\mnras, 475, 853}
\arxiv{1709.10108}

\item
M.~Velliscig \etal{17}
\paper{Galaxy-Galaxy Lensing in EAGLE: Comparison with Data from 180 Square 
Degrees of the KiDS and GAMA Surveys},
2017, \href{https://ui.adsabs.harvard.edu/abs/2017MNRAS.471.2856V}{\mnras, 471, 2856}
\arxiv{1612.04825}

\item
A.~Dvornik \etal{22}
\paper{A KiDS Weak Lensing Analysis of Assembly Bias in GAMA Galaxy Groups},
2017, \href{https://ui.adsabs.harvard.edu/abs/2017MNRAS.468.3251D}{\mnras, 468, 3251}
\arxiv{1703.06657}

\item
M.~M.~Brouwer \etal{22}
\paper{First Test of Verlinde's Theory of Emergent Gravity Using Weak 
Gravitational Lensing Measurements},
2017, \href{https://ui.adsabs.harvard.edu/abs/2017MNRAS.466.2547B}{\mnras, 466, 2547}
\arxiv{1612.03034}

\item
M.~M.~Brouwer \etal{36}
\paper{Dependence of GAMA Galaxy Halo Masses on the Cosmic Web Environment from 
100 Square Degrees of KiDS Weak Lensing Data},
2016, \href{https://ui.adsabs.harvard.edu/abs/2016MNRAS.462.4451B}{\mnras, 462, 4451}
\arxiv{1604.07233}

\item
N.~Battaglia \etal{39}
\paper{Weak-Lensing Mass Calibration of the Atacama Cosmology Telescope 
Equatorial Sunyaev-Zel'dovich Cluster Sample with the Canada-France-Hawaii 
Telescope Stripe 82 Survey},
2016, \href{https://ui.adsabs.harvard.edu/abs/2016JCAP...08..013B}{\jcap, 08, 013}
\arxiv{1509.08930}

\item
S.~Bellstedt \etal{16}
\paper{The Evolution in the Stellar Mass of Brightest Cluster Galaxies over the 
Past 10 Billion Years},
2016, \href{https://ui.adsabs.harvard.edu/abs/2016MNRAS.460.2862B}{\mnras, 460, 2862}
\arxiv{1605.02736}

\item
K.~Knowles \etal{21}
\paper{A Giant Radio Halo in a Low-Mass SZ-Selected Galaxy Cluster: 
ACT-CL~J0256.5+0006},
2016, \href{https://ui.adsabs.harvard.edu/abs/2016MNRAS.459.4240K}{\mnras, 459, 4240}
\arxiv{1506.01547}

\item
D.~Crichton \etal{22}
\paper{Evidence for the Thermal Sunyaev-Zel'dovich Effect Associated with Quasar 
Feedback},
2016, \href{https://ui.adsabs.harvard.edu/abs/2016MNRAS.458.1478C}{\mnras, 458, 1478}
\arxiv{1510.05656}

\item
J.~T.~A.~de~Jong \etal{49}
\paper{The First and Second Data Releases of the Kilo Degree Survey},
2015, \href{https://ui.adsabs.harvard.edu/abs/2015A&A...582A..62D}{\aap, 582, 62}
\arxiv{1507.00742}

\item
K.~Kuijken \etal{35}
\paper{Gravitational Lensing Analysis of the Kilo Degree Survey},
2015, \href{https://ui.adsabs.harvard.edu/abs/2015MNRAS.454.3500K}{\mnras, 454, 3500}
\arxiv{1507.00738}

\item
K.~Y.~Ng, W.~A.~Dawson, D.~Wittman, M.~J.~Jee, J.~P.~Hughes, F.~Menanteau, \myself,
\paper{The Return of the Merging Galaxy Subclusters of El Gordo?},
2015, \href{https://ui.adsabs.harvard.edu/abs/2015MNRAS.453.1531N}{\mnras, 453, 1531}
\arxiv{1412.1826}

\item
M.~Viola \etal{27}
\paper{Dark Matter Halo Properties of GAMA Galaxy Groups from 100 Square Degrees 
of KiDS Weak Lensing Data},
2015, \href{https://ui.adsabs.harvard.edu/abs/2015MNRAS.452.3528V}{\mnras, 452, 3529}
\arxiv{1507.00735}

\item
R.~R.~Lindner \etal{25}
\paper{The Atacama Cosmology Telescope: the LABOCA/ACT Survey of Clusters at All 
Redshifts},
2015, \href{https://ui.adsabs.harvard.edu/abs/2015ApJ...803...79L}{\apj, 803, 79}
\arxiv{1411.7998}

\item
B.~Kirk \etal{23}
\paper{SALT Spectroscopic Observations of Galaxy Clusters Detected by ACT and a 
Type II Quasar Hosted by a Brightest Cluster Galaxy},
2015, \href{https://ui.adsabs.harvard.edu/abs/2015MNRAS.449.4010K}{\mnras, 449, 4010}
\arxiv{1410.7887}

\item
L.~Old \etal{24}
\paper{Galaxy Cluster Mass Reconstruction Project: II. Results for Galaxy-Based 
Techniques with Improved Models},
2015, \href{https://ui.adsabs.harvard.edu/abs/2015MNRAS.449.1897O}{\mnras, 449, 1897}
\arxiv{1502.07347}

\item
M.~B.~Gralla \etal{41}
\paper{A Measurement of the Millimeter Emission and the Sunyaev-Zel'dovich 
Effect Associated with Low-Frequency Radio Sources},
2014, \href{https://ui.adsabs.harvard.edu/abs/2014MNRAS.445..460G}{\mnras, 445, 460}
\arxiv{1310.8281}

\item
L.~Old \etal{21}
\paper{Galaxy Cluster Mass Reconstruction Project: I. Methods and First Results 
on Galaxy-Based Techniques},
2014, \href{https://ui.adsabs.harvard.edu/abs/2014MNRAS.441.1513O}{\mnras, 441, 1513}
\arxiv{1403.4610}

\item
M.~J.~Jee, J.~P.~Hughes, F.~Menanteau, \myself, L.~F.~Barrientos, L.~Infante, 
R.~Mandelbaum, K.~Y.~Ng,
\paper{Weighing ``El Gordo'' with a Precision Scale: Hubble Space Telescope 
Weak-Lensing Analysis of the Galaxy Cluster ACT-CL J0102$\textbf{-}$4915 at 
$z=0.87$},
2014, \href{https://ui.adsabs.harvard.edu/abs/2014ApJ...785...20J}{\apj, 785, 20}
\arxiv{1309.5097}

\item
M.~Hasselfield \etal{44}
\paper{The Atacama Cosmology Telescope: Sunyaev-Zel'dovich Selected Galaxy 
Clusters at 148 GHz from Three Seasons of Data},
2013, \href{https://ui.adsabs.harvard.edu/abs/2013JCAP...07..008H}{\jcap, 07, 008}
\arxiv{1301.0816}

\item
E.~Calabrese \etal{34}
\paper{Cosmological Parameters from Pre-Planck Cosmic Microwave Background 
Measurements},
2013, \href{https://ui.adsabs.harvard.edu/abs/2013PhRvD..87j3012C}{\prd, 87, 103012}
\arxiv{1302.1841}

\item
N.~Sehgal \etal{36}
\paper{The Atacama Cosmology Telescope: Relation between Galaxy Cluster Optical 
Richness and Sunyaev-Zel'dovich Effect},
2013, \href{https://ui.adsabs.harvard.edu/abs/2013ApJ...767...38S}{\apj, 767, 38}
\arxiv{1205.2369}

\item
H.~Miyatake \etal{28}
\paper{Subaru Weak-Lensing Measurement of a $z=0.81$ Cluster Discovered by the 
Atacama Cosmology Telescope Survey},
2013, \href{https://ui.adsabs.harvard.edu/abs/2013MNRAS.429.3627M}{\mnras, 429, 3627}
\arxiv{1209.4643}

\item
B.~D.~Sherwin \etal{31}
\paper{The Atacama Cosmology Telescope: Cross-correlation of CMB Lensing and 
Quasars},
2012, \href{https://ui.adsabs.harvard.edu/abs/2012PhRvD..86h3006S}{\prd, 86, 083006}
\arxiv{1207.4543}

\item
N.~Hand \etal{58}
\paper{Evidence of Galaxy Cluster Motions with the Kinematic Sunyaev-Zel'dovich 
Effect},
2012, \href{https://ui.adsabs.harvard.edu/abs/2012PhRvL.109d1101H}{\prl, 109, 041101}
\arxiv{1203.4219}

\item
E.~D.~Reese \etal{44}
\paper{The Atacama Cosmology Telescope: High-Resolution Sunyaev-Zel'dovich Array 
Observations of ACT SZE-selected Clusters from the Equatorial Strip},
2012, \href{https://ui.adsabs.harvard.edu/abs/2012ApJ...751...12R}{\apj, 751, 12}
\arxiv{1108.3343}




\end{etaremune}



\end{document}




\end{document}

\documentclass[11pt]{article}


\usepackage{bm}
\usepackage{enumerate}
\usepackage{etaremune}
%\usepackage{fullpage}
\usepackage{graphicx}
\usepackage{hyperref}

\addtolength{\oddsidemargin}{-1in}
\addtolength{\evensidemargin}{-1in}
\addtolength{\textwidth}{2in}
\addtolength{\topmargin}{-1.3in}
\addtolength{\textheight}{2.2in}

% For links of references
\hypersetup{colorlinks,
  linkcolor=blue,
  filecolor=blue,
  urlcolor=blue,
  citecolor=blue}

\renewcommand{\familydefault}{\sfdefault}
\usepackage{chancery}

\newcommand\sectitle[1]{
  \hline
  \vspace{0.5cm}
  \noindent
  %\underline{
    \textbf{\Large #1}
  %}
  \\
  \vspace{-0.2cm}
}

\newcommand\subsectitle[1]{
  \vspace{0.3cm}
  \noindent
  %\underline{
    \textbf{\large #1}
  %}
  \\
  \vspace{-0.3cm}
}

\newcommand\technical[2]{
  \noindent
    {\large\bf #1:} #2\\
  }

\newcommand\itemdate[1]{\textbf{[#1]}}
\newcommand\itemdates[2]{\textbf{[#1 -- #2]}}
\newcommand\email[1]{\href{mailto:#1}{\texttt{#1}}}
\newcommand\pucv{Pontificia Universidad Cat\'olica de Valpara\'iso}


%not used if publication list not shown

\def\aap{A\&A}
\def\apj{ApJ}
\def\apjs{ApJS}
\def\baas{\textit{Bull.\ of the Am.\ Ast.\ Soc.}}
\def\gemfoc{Gemini Focus}
\def\jcap{JCAP}
\def\mnras{MNRAS}
\def\msngr{The Messenger}
\def\pasj{PASJ}
\def\prd{Phys.\ Rev.\ D}
\def\prl{Phys.\ Rev.\ Letters}
\def\ssr{Space Sci.\ Rev.}

\newcommand{\myself}{\textbf{\color{red} C.~Sif\'on}}
\newcommand\includemyself{\textbf\small{(including C.~Sif\'on)}}
\newcommand{\accepted}[1]{accepted for publication in #1}
\newcommand{\etal}[1]{et al.\ (#1 co-authors),}
\newcommand{\etalwithme}[1]{et al.\ (#1 coauthors incl.\ \myself),}
\newcommand{\paper}[1]{\textbf{``#1''}}
\newcommand{\submitted}[1]{submitted to #1}
\renewcommand{\title}[1]{\noindent\textbf{\huge #1}\\}
% arXiv links
\newcommand{\arxiv}[1]
    {\href{https://arxiv.org/abs/#1}{\texttt{\color{magenta}[arXiv]}}}




\begin{document}

% \begin{figure}[t]
\begin{minipage}[b]{0.46\linewidth}
\flushleft
% \noindent
\hspace{-0.7cm}
{\bf\huge Crist\'obal Sif\'on}\\\vspace{0.2cm}
\hspace{-0.5cm}{\large Profesor Auxiliar\\
\hspace{-0.65cm}        Instituto de F\'isica, Facultad de Ciencias\\
\hspace{-0.65cm}        \pucv\\
\hspace{-0.65cm}        Casilla 4059, Valpara\'iso, Chile}\\
\end{minipage}
\begin{minipage}[b]{0.49\linewidth}
\flushright
{\large E-mail: {\texttt cristobal.sifon@pucv.cl}\\
        Phone: +56 (32) 227 4698\\
        \url{https://github.com/cristobal-sifon/}}
\end{minipage}
\vspace{0.4cm}
\hline



\sectitle{Research Interests}

My research focuses on galaxy cluster physics including observable--mass scaling relations for cosmological analyses and the transformation of galaxies in and around galaxy clusters. I am also interested in intrinsic galaxy alignments, both as contaminants for cosmic shear and as a physical mechanism in their own right. I use various tools and techniques to study these phenomena, including weak gravitational lensing, spectroscopy, the exploitation of optical surveys in general, and most recently also of hydrodynamical simulations.

\vspace{0.5cm}
\technical{Collaborations}
{
 4MOST Chilean Cluster Galaxy Evolution Survey (CHANCES) ---
 4MOST Hemisphere Survey (4HS) ---
 Atacama Cosmology Telescope (ACT) ---
 Canadian Cluster Comparison Project (CCCP) ---
 Cerro Chajnator Atacama Telescope (CCAT) ---
 CMB-S4 ---
 Galaxy Cluster Mass Reconstruction Project ---
 Kilo-Degree Survey (KiDS) ---
 Legacy Survey of Space and Time Dark Energy Science Collaboration (LSST-DESC) ---
 Multi-Epoch Nearby Cluster Survey (MENeaCS) ---
 Simons Observatory.
}


\sectitle{Employment}

\noindent
\itemdates{2022}{Present} Profesor Auxiliar, \pucv\ (PUCV), Chile\\
\itemdates{2019}{2022} Profesor Asociado, PUCV\\
\itemdates{2016}{2019} Postdoctoral Research Associate, Princeton University, USA


\subsectitle{Education}

\noindent
\itemdates{2012}{2016} Ph.D.~Astrophysics, Universiteit Leiden, The Netherlands\\
\itemdates{2010}{2012} M.Sc.~Astrophysics, P.~Universidad Cat\'olica de Chile (PUC), Chile\\
\itemdates{2005}{2010} B.Sc.~Astronomy, PUC\\


%\subsectitle{Internships}

%\noindent
%\itemdate{2011} Science Intern, Gemini South Observatory (6 months)\\
%\itemdate{2011} Internship, Rutgers University (2 months)\\
%\itemdate{2009} Science Intern, Gemini South Observatory (6 months, \emph{B.Sc.\ thesis})\\

%%%

\sectitle{Teaching \& Mentoring}

\subsectitle{Graduate Research Mentoring}

\noindent
\itemdates{2023}{Present} Javier Urrutia, PUCV: MSc thesis advisor.\\
\itemdates{2020}{2022} Camila Aros, PUCV: MSc thesis advisor.\\
\itemdates{2017}{2019} Naomi Robertson, Oxford University (UK): co-advised PhD thesis project (Advisor: Joanna Dunkley).\\
\itemdates{2013}{2014} Joshua Albert, Universiteit Leiden: co-advised MSc thesis project (Advisor: Huub R\"ottgering). 

\subsectitle{Undergraduate Research Mentoring}

\noindent
% \itemdate{2023} Ignacia Moya, PUCV: Undergraduate Senior thesis.\\
% \itemdate{2022} Pablo Garrido, PUCV: Undergraduate Senior thesis.\\
% \itemdate{2021} Lya Marmolejo, PUCV: Undergraduate summer research project (USRP) and Senior thesis.\\
% \itemdate{2020} Nicole Mej\'ia, Universidad Nacional Aut\'onoma de Honduras (Honduras): Four-month undergraduate
%     research project through the Central American-Caribbean Bridge in Astrophysics Program
%     (\href{https://cencabridgeastro.weebly.com/our-team.html}{URL}).\\
% \itemdate{2020} Felipe Jorquera, PUCV: USRP.\\
% \itemdate{2018} Malik Walker, Princeton University: USRP and Junior Project.\\
% I have been advisor of three Senior theses and four summer projects at PUCV, plus one summer project and junior project at Princeton and a four-month research project through the Central American-Caribbean Bridge in Astrophysics Program (\href{https://cencabridgeastro.weebly.com/our-team.html}{URL}).
\itemdate{PUCV} 3 Senior theses and 4 Summer projects.\\
\itemdate{Princeton} Summer project and Junior project.\\
\itemdate{Others} Four-month research project through the Central American-Caribbean Bridge in Astrophysics (\href{https://cencabridgeastro.weebly.com/our-team.html}{URL}).

\pagebreak

\subsectitle{Courses Taught}

%\noindent
%\itemdates{2021}{2022} Galactic Astronomy (Astronomy undergraduates, PUCV)\\
%\itemdates{2020}{2021} Programming (Physics+astronomy undergraduates, PUCV)\\
%\itemdates{2020}{2021} Observational Cosmology (Graduate-level astronomy, PUCV)\\
%\itemdate{2020} Cosmology (Physics+astronomy undergraduates, PUCV)\\
%\noindent\hspace{-0.1cm}
\technical{Graduate}{Data Analysis (2023), Techniques of Observational Astrophysics (2022), Observational Cosmology (2020-2021)}
\technical{Undergraduate}{Astronomical Instrumentation (2023), Galactic Astronomy (2021-2022), Programming (2020-2022), Cosmology (2020)}
\technical{Non-Physics Major}{Basic Astronomy for Engineers (2022)}

\hline

%\pagebreak

\sectitle{Grants}

\noindent
\itemdate{2020} ALMA-ANID Fund to hire a postdoc (\textbf{Co-PI}, 2 years, US\$77,000)\\
\itemdate{2019} FONDECYT Iniciaci\'on research grant (\textbf{PI}, 3 years, US\$125,000)


\subsectitle{Successful Observing Proposals (as PI)}

%\noindent
%I have been the PI of 9 different successful observing proposals in 5 different telescopes:
%
%\noindent
%\itemdate{Magellan/FourStar} (2020AB,2019AB) 6 nights for near-infrared imaging of galaxy clusters\\
%\itemdate{Very Large Array} (2019A) 4.5 h to study AGN feedback in galaxy clusters\\
%\itemdate{Giant Metrewave Radio Telescope} (2017B,2013B) 44 h to study diffuse radio emission in clusters\\
%\itemdate{Gemini South/GMOS} (2017B) 24 h for optical imaging and spectroscopy of high-redshift galaxy clusters\\
%\itemdate{VLT Survey Telescope/OmegaCAM} (2015A) 6 h for optical imaging of galaxy clusters\\

\noindent
I have been the PI of 12 successful observing proposals totalling hundreds of 
observing hours in optical (Gemini-South/GMOS, VST/OmegaCAM), near-infrared 
(Magellan/Fourstar), submm (APEX/CONCERTO), and radio (GMRT, VLA) telescopes.\\

\technical{Observing Experience}
{I have spent roughly 180 hours observing with 
optical (Gemini South/GMOS) and near-infrared (NTT/SofI, Magellan/Fourstar) instruments performing both 
imaging and spectroscopy of galaxy clusters.}


%\sectitle{Language skills}
%
%{Native Spanish, fluent English, basic Dutch}

\sectitle{Community Activity}

\noindent 
\textbf{Journals:} I have served as a referee for Astronomy \& Astrophysics, The 
    Astrophysical Journal, Monthly Notices of the Royal
    Astronomical Society, and Nature Astronomy.\\
\noindent
\textbf{Telescope Allocation Committees:}
Canadian Astronomical Society, \textit{Chandra} X-ray Observatory\\
\noindent
\textbf{Grant Allocation Committees:} Swiss National Science Foundation

% \subsectitle{Informal courses}

% \noindent
% \itemdate{2016} \emph{Making Better Figures}, Universiteit Leiden
% (see \href{https://home.strw.leidenuniv.nl/~kenworthy/teaching/better_figures/}{here})

\subsectitle{Press Articles Authored}

\noindent
% \itemdate{March 2023} \emph{CHANCES: A CHileAN Cluster galaxy Evolution Survey} (The Messenger)
\itemdate{July 2017} \emph{Galaxy clusters: Falling into line} (Nature Astronomy \emph{News \& Views})\\
\itemdate{July 2013} \emph{Featured Science: Dynamical masses of galaxy clusters discovered with the Sunyaev-Zel'dovich effect} (Gemini Focus)
%\emph{Galaxy clusters: Falling into line} (Nature Astronomy \emph{News \& Views}, July 2017)\\
%\emph{Dynamical masses of galaxy clusters discovered with the Sunyaev-Zel'dovich effect} (Gemini Focus \emph{Featured Science}, July 2013)

\subsectitle{Outreach}

\noindent
\itemdate{Aug.\ 2023} Guest at ``Conversemos de Astronomía'' podcast (in Spanish, available \href{https://open.spotify.com/episode/4PBTlnI3ufDtBwXIGee5mp?si=KGS43HFnSuiVdVPYaoBMqA}{here}).\\
\itemdate{Nov.\ 2022} Guest at ``Rockstars'' podcast by Radio TXS (in Spanish, available \href{https://soundcloud.com/txsplus/rockstars-con-gabriel-leon-y-cristobal-sifon-4-de-noviembre-del-2022?utm_source=Email&utm_campaign=social_sharing&utm_medium=widgetutm_content=https%3A%2F%2Fsoundcloud.com%2Ftxsplus%2Frockstars-con-gabriel-leon-y-cristobal-sifon-4-de-noviembre-del-2022}{here}).\\
\itemdate{Oct.\ 2022} Interview for Radio Valent\'in Letelier, Valparaíso, to talk about CHANCES.\\
\itemdate{Mar.\ 2021} Online public talk in the conext of the Chilean \textit{Day of Astronomy} (in Spanish, available \href{https://www.youtube.com/watch?v=MgrKSd6JWkE}{here}).\\
\itemdates{2018}{2019} Assisted with \emph{Public Astronomical Observations in Spanish}, Princeton University.\\
\itemdates{2013}{2014} Assisted with \emph{Public Observations at the Old Observatory}, Leiden Observatory.\\
\itemdate{2012} Co-taught an \emph{Astronomy Course for Seniors}, PUC.\\
\itemdate{2011} Participated in \emph{Starry Nights}, observation nights for 
elementary and middle school students in social risk organized by ESO-Santiago.\\
\itemdate{2010} Invited talk on board the ``FFG14 Almirante Latorre'' Chilean Navy ship, Valpara\'iso, Chile.\\
\itemdate{2010} \emph{The Universe}, a series of talks for elementary school students in social risk organized by PUC.\\

\pagebreak

\sectitle{Technical skills}

I am an experienced \texttt{Python} programmer and I have some familiarity with
\texttt{IDL} and \texttt{Julia}. I wrote {\tt pygmos}, a Python/PyRAF pipeline to reduce 
Gemini-GMOS spectra which is available 
\href{https://github.com/cristobal-sifon/pygmos/}{\texttt{here}}. 
% I also developed an early analysis pipeline for the FLAMINGOS-II infrared imager
% and spectrograph installed in the Gemini-South telescope.
I am one of the lead 
developers and maintainers of the galaxy-galaxy lensing pipeline used by the 
KiDS collaboration (written in \texttt{Python}, but which is not public at the 
moment). Other codes I have written are posted at my 
\href{https://github.com/cristobal-sifon}{\texttt{github}} page.\\


%\vspace{1cm}
\hline
%\hline
%\vspace{0.5cm}
%\pagebreak


\sectitle{Other Work Experience}

\noindent
\itemdates{2020}{2021} Data science \& Machine Learning2 consultant, Minera Centinela, Chile.\\
\itemdates{2007}{2008} Ski instructor at Homewood Mountain Ski Resort in Lake 
Tahoe, CA, USA. Obtained certification as \emph{Level I Ski Instructor} by the 
Professional Ski Instructors of America (PSIA).\\
\itemdates{2006}{2007} Ski lift operator at Sun Valley Resort, Sun Valley, ID, USA.\\


\sectitle{References}

\begin{itemize}
%  \item 
\item Prof.~Henk Hoekstra (\textit{PhD advisor})\\
      Leiden Observatory, Universiteit Leiden\\
      Niels Bohrweg 2, NL-2333 CA Leiden, The Netherlands\\
      Phone: +31 (71) 527 5594\\
      E-mail: \email{hoekstra@strw.leidenuniv.nl}
\item Prof.~John P.~Hughes\\
      Department of Physics and Astronomy, Rutgers University\\
      136 Frelinghuysen Rd., Piscataway, NJ 08854, USA\\
      Phone: +1 (848) 445 8878\\
      E-mail: \email{jph@physics.rutgers.edu}
\item Prof.~Felipe Menanteau\\
      Department of Astronomy, University of Illinois at Urbana-Champaign\\
      1002 W.\ Green St., Urbana, IL 61801, USA\\
      Phone: +1 (217) 244 6297\\
      E-mail: \email{felipe@illinois.edu}
\item Prof.~David N.~Spergel\\
      Center for Computational Astrophysics, Flatiron Institute\\
      160 Fifth Avenue, 7th Floor, New York, NY 10010, USA\\
      Phone: +1 (646) 654 0066\\
      E-mail: \email{dns@astro.princeton.edu}
\item Prof.~L.~Felipe Barrientos (\textit{MSc advisor})\\
      Instituto de Astrof\'isica, P. Universidad Cat\'olica de Chile\\
      Casilla 306, Santiago 22, Chile\\
      Phone: +56 (2) 2354 4941\\
      E-mail: \email{barrientos@astro.uc.cl}
\end{itemize}

\vspace{0.3cm}
\hline


%%%%%%%%%%%%%%%%%%%%%%%%%%%%%%%%%%%%%%
%%%%%%%%%%%%%%%%%%%%%%%%%%%%%%%%%%%%%%
%%%%%%%%%% PUBLICATION LIST %%%%%%%%%%
%%%%%%%%%%%%%%%%%%%%%%%%%%%%%%%%%%%%%%
%%%%%%%%%%%%%%%%%%%%%%%%%%%%%%%%%%%%%%

%% Uncomment these two lines to get the publication list in the main pdf

%\pagebreak
%\title{Recent Publications}


\noindent
I have co-authored 80 scientific articles intended for peer-reviewed 
publication, including 7 first-author papers. They have been cited more than 
3,700 times and have an $h$-index of 35, with more than 300 citations on my 
first-author papers. My publications include three companion reviews on galaxy 
alignments written for a special issue of Space Science Reviews (B.\ Joachimi et 
al.\ 2015, A.\ Kiessling et al.\ 2015, D.\ Kirk et al.\ 2015). The full list of 
publications can be accessed at \href{https://goo.gl/LAu9G4}{this url}. I also 
wrote an invited `News \& Views' article for the 4 July 2017 edition of Nature 
Astronomy, accessible 
\href{https://www.nature.com/articles/s41550-017-0181}{here}.
%
This document is maintained live on
\href{https://github.com/cristobal-sifon/cv/blob/master/Sifon_publications.pdf}{\texttt{github}}.




\begin{etaremune}

\item
B.~Fuzia \etalwithme{22}
\paper{The Atacama Cosmology Telescope: SZ-Based Masses and Dust Emission from IR-Selected Cluster Candidates in the SHELA Survey},
2020, \href{https://ui.adsabs.harvard.edu/abs/2020arXiv200109587F/abstract}{arXiv:2001.09587},
\submitted{\mnras}

\item
S.~Huang \etalwithme{12}
%A.~Leauthaud, A.~Hearin, P.~Behroozi, C.~Bradshaw, F.~Ardila, J.~Speagle,
%A.~Tenneti, K.~Bundy, J.~Greene, \myself, N.~Bahcall,
\paper{Weak Lensing Reveals a Tight Connection Between Dark Matter Halo Mass and the Distribution of Stellar Mass in Massive Galaxies},
2020, \href{https://ui.adsabs.harvard.edu/abs/2020MNRAS.492.3685H/abstract}{\mnras, 492, 3685}
\arxiv{1811.01139}

\item
Q.~Xia \etalwithme{13}
\paper{A Gravitational Lensing Detection of Filamentary Structures Connecting Luminous Red Galaxies},
2020, \href{https://ui.adsabs.harvard.edu/abs/2020A&A...633A..89X/abstract}{\aap, 633, 89}
\arxiv{1909.05852}

\item
H.~Hildebrandt \etalwithme{28}
\paper{KiDS+VIKING-450: Cosmic Shear Tomography with Optical+infrared Data},
2020, \href{https://ui.adsabs.harvard.edu/abs/2020A&A...633A..69H/abstract}{\aap, 633, 69}
\arxiv{1812.06076}

\item
R.~Herbonnet, \myself, H.~Hoekstra, Y.~Bah\'e, R.~F.~J.~van~der~Burg, J.-B.~Melin, A.~von~der~Linden, D.~Sand, S.~Kay, D.~Barnes,
\paper{CCCP and MENeaCS: (Updated) Weak-Lensing Masses for 100 Galaxy Clusters},
2019, \href{https://ui.adsabs.harvard.edu/abs/2019arXiv191204414H/abstract}{arXiv:1912.04414},
\submitted{\mnras}

\item
M.~Madhavacheril \etalwithme{49}
\paper{The Atacama Cosmology Telescope: Component-Separated Maps of CMB Temperature and the Thermal Sunyaev-Zel'dovich Effect},
2019, \href{https://ui.adsabs.harvard.edu/abs/2019arXiv191105717M/abstract}{arXiv:1911.05717},
\submitted{\prd}

\item
Y.~Rong \etalwithme{13}
\paper{Intrinsic Morphology Evolution of Ultra-diffuse Galaxies},
2019, \href{https://ui.adsabs.harvard.edu/abs/2019arXiv190710079R/abstract}{arXiv:1907.10079},
\submitted{\apj}

\item
C.~Hikage \etalwithme{30}
\paper{Cosmology from Cosmic Shear Power Spectra with Subaru Hyper Suprime-Cam First-Year Data},
2019, \href{http://adsabs.harvard.edu/abs/2019PASJ...71...43H}{\pasj, 71, 43}
\arxiv{1809.09148}

\item
H.~Miyatake \etalwithme{58}
\paper{Weak-Lensing Mass Calibration of ACTPol Sunyaev-Zel'dovich Clusters with the Hyper Suprime-Cam Survey},
2019, \href{http://adsabs.harvard.edu/abs/2019ApJ...875...63M}{\apj, 875, 63}
\arxiv{1804.05873}

\item
K.~Knowles \etalwithme{14}
\paper{GMRT 610 MHz Observations of Galaxy Clusters in the ACT Equatorial Sample},
2019, \href{http://adsabs.harvard.edu/abs/2019MNRAS.486.1332K}{\mnras, 486, 1332}
\arxiv{1806.09579}

\item
\myself, R.~Herbonnet, H.~Hoekstra, R.~F.~J.~van~der~Burg, M.~Viola,
\paper{The Galaxy-Subhalo Connection in Low-Redshift Galaxy Clusters from Weak Gravitational Lensing},
2018, \href{http://adsabs.harvard.edu/abs/2018MNRAS.478.1244S}{\mnras, 478, 1244}
\arxiv{1706.06125}

\item
\myself, R.~F.~J.~van~der~Burg, H.~Hoekstra, A.~Muzzin, R.~Herbonnet,
\paper{A First Constraint on the Average Mass of Ultra Diffuse Galaxies from Weak Gravitational Lensing},
2018, \href{http://adsabs.harvard.edu/abs/2018MNRAS.473.3747S}{\mnras, 473, 3747}
\arxiv{1704.07847}

\item
M.~Hilton, M.~Hasselfield, \myself, \etal{43}
\paper{The Atacama Cosmology Telescope: The Two-Season ACTPol Sunyaev-Zel'dovich Effect Selected Cluster Catalog},
2018, \href{http://adsabs.harvard.edu/abs/2018ApJS..235...20H}{\apjs, 235, 20}
\arxiv{1709.05600}


\end{etaremune}



\end{document}

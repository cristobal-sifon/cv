\documentclass[11pt]{article}


\usepackage{bm}
\usepackage{enumerate}
\usepackage{etaremune}
%\usepackage{fullpage}
\usepackage{graphicx}
\usepackage{hyperref}

\addtolength{\oddsidemargin}{-1in}
\addtolength{\evensidemargin}{-1in}
\addtolength{\textwidth}{2in}
\addtolength{\topmargin}{-1.3in}
\addtolength{\textheight}{2.2in}

% For links of references
\hypersetup{colorlinks,
  linkcolor=blue,
  filecolor=blue,
  urlcolor=blue,
  citecolor=blue}

\renewcommand{\familydefault}{\sfdefault}
\usepackage{chancery}

\newcommand\sectitle[1]{
  \hline
  \vspace{0.5cm}
  \noindent
  %\underline{
    \textbf{\Large #1}
  %}
  \\
  \vspace{-0.2cm}
}

\newcommand\subsectitle[1]{
  \vspace{0.3cm}
  \noindent
  %\underline{
    \textbf{\large #1}
  %}
  \\
  \vspace{-0.3cm}
}

\newcommand\technical[2]{
  \noindent
    {\large\bf #1:} #2\\
  }

\newcommand\itemdate[1]{\textbf{[#1]}}
\newcommand\itemdates[2]{\textbf{[#1 -- #2]}}
\newcommand\email[1]{\href{mailto:#1}{\texttt{#1}}}
\newcommand\pucv{Pontificia Universidad Cat\'olica de Valpara\'iso}


%not used if publication list not shown

\def\aap{A\&A}
\def\apj{ApJ}
\def\apjs{ApJS}
\def\baas{\textit{Bull.\ of the Am.\ Ast.\ Soc.}}
\def\gemfoc{Gemini Focus}
\def\jcap{JCAP}
\def\mnras{MNRAS}
\def\msngr{The Messenger}
\def\pasj{PASJ}
\def\prd{Phys.\ Rev.\ D}
\def\prl{Phys.\ Rev.\ Letters}
\def\ssr{Space Sci.\ Rev.}

\newcommand{\myself}{\textbf{\color{red} C.~Sif\'on}}
\newcommand\includemyself{\textbf\small{(including C.~Sif\'on)}}
\newcommand{\accepted}[1]{accepted for publication in #1}
\newcommand{\etal}[1]{et al.\ (#1 co-authors),}
\newcommand{\etalwithme}[1]{et al.\ (#1 coauthors incl.\ \myself),}
\newcommand{\paper}[1]{\textbf{``#1''}}
\newcommand{\submitted}[1]{submitted to #1}
\renewcommand{\title}[1]{\noindent\textbf{\huge #1}\\}
% arXiv links
\newcommand{\arxiv}[1]
    {\href{https://arxiv.org/abs/#1}{\texttt{\color{magenta}[arXiv]}}}




\begin{document}

% \begin{figure}[t]
\begin{minipage}[b]{0.46\linewidth}
\flushleft
% \noindent
\hspace{-0.7cm}
{\bf\huge Crist\'obal Sif\'on}\\\vspace{0.2cm}
\hspace{-0.5cm}{\large Profesor Asociado\\
\hspace{-0.65cm}        Instituto de F\'isica, Facultad de Ciencias\\
\hspace{-0.65cm}        \pucv\\
\hspace{-0.65cm}        Casilla 4059, Valpara\'iso, Chile}\\
\end{minipage}
\begin{minipage}[b]{0.49\linewidth}
\flushright
{\large E-mail: {\texttt cristobal.sifon@pucv.cl}\\
        Phone: +56 (32) 227 4698\\
        \url{http://fis.ucv.cl/cristobal-sifon/}
        \url{https://github.com/cristobal-sifon/}}
\end{minipage}
\vspace{0.4cm}
\hline



\sectitle{Research Interests}

My research focuses on galaxy cluster physics including observable--mass scaling relations for cosmological analyses, brightest cluster galaxies, the mass content of cluster galaxies, and merging clusters. I am also interested in intrinsic galaxy alignments, both as contaminants for cosmic shear and as a physical mechanism in their own right. I use various tools and techniques to study these phenomena, including weak gravitational lensing, spectroscopy, the exploitation of optical surveys in general, and most recently analyses involving hydrodynamical simulations.

\vspace{0.5cm}
\technical{Collaborations}
{
 Atacama Cosmology Telescope (ACT) ---
 Canadian Cluster Comparison Project (CCCP) ---
 Galaxy Cluster Mass Reconstruction Project ---
 Kilo-Degree Survey (KiDS) ---
 Large Synoptic Survey Telescope Dark Energy Science Collaboration (LSST-DESC) ---
 Multi-Epoch Nearby Cluster Survey (MENeaCS) ---
 Simons Observatory.
}


\sectitle{Employment}

\noindent
\itemdates{2019}{Present} Profesor Asociado (Assistant Professor), \pucv\ (PUCV), Chile\\
\itemdates{2016}{2019} Postdoctoral Research Associate, Princeton University, USA


\subsectitle{Education}

\noindent
\itemdates{2012}{2016} Ph.D.~Astrophysics, Universiteit Leiden, The Netherlands\\
\itemdates{2010}{2012} M.Sc.~Astrophysics, P.~Universidad Cat\'olica de Chile (PUC), Chile\\
\itemdates{2005}{2010} B.Sc.~Astronomy, P.~Universidad Cat\'olica de Chile, Chile


\subsectitle{Internships}

\noindent
\itemdate{2011} Science Intern, Gemini South Observatory (6 months)\\
\itemdate{2011} Internship, Rutgers University (2 months)\\
\itemdate{2009} Science Intern, Gemini South Observatory (6 months, \emph{B.Sc.\ thesis})\\

%%%

\sectitle{Teaching \& Mentoring}

\subsectitle{Student Research Mentoring}

\noindent
\itemdates{2020}{present} Camila Aros, PUCV: MSc thesis advisor.\\
\itemdates{2020}{present} Nicole Mej\'ia, Universidad Nacional Aut\'onoma de Honduras (Honduras): Advising undergraduate
    research project through the Central American-Caribbean Bridge in Astrophysics Program.\\
\itemdates{2017}{2019} Naomi Robertson, Oxford University (UK): co-advised PhD thesis project.\\
\itemdate{2018} Malik Walker, Princeton University: Undegraduate Summer Research Program and Junior Project.\\
\itemdates{2013}{2014} Joshua Albert, Universiteit Leiden: co-advised MSc thesis project. 

\pagebreak

\subsectitle{Courses Taught}

\noindent
\itemdate{2020B} Cosmology (undergraduate, PUCV)\\
\itemdate{2020B} Programming (undergraduate, PUCV)\\
\itemdate{2020A} Observational Cosmology (graduate level, PUCV)

\subsectitle{Teaching Assistant}

%\noindent
%\itemdate{2013B} Stellar Dynamics (Leiden, Prof.~S.~Portegies Zwart)\\
%\itemdate{2012A} Extragalactic Astrophysics (PUC, Prof.~L.~F.~Barrientos)\\
%\itemdate{2011A} Extragalactic Astrophysics (PUC, Prof.~L.~F.~Barrientos)\\
%\itemdate{2011A} Laboratory of Thermodynamics and Kinetic Theory (PUC, Prof.~U.~Volkmann)\\
%\itemdate{2010B} Experimental Astrophysics (PUC, Prof.~L.~F.~Barrientos)
\noindent
\itemdate{Leiden} Stellar dynamics; organizer of MSc thesis defense presentations\\
\itemdate{PUC} Extragalactic astrophysics; Experimental astrophysics; Laboratory of thermodynamics and kinetic theory\\

\hline


\sectitle{Grants}

\noindent
\itemdate{2019} Proyecto FONDECYT Iniciaci\'on (\textbf{PI}, 3 years, US\$125,000)


\subsectitle{Successful Observing Proposals (as PI)}

\noindent
I have been the PI of 9 different successful observing proposals in 5 different telescopes:

\noindent
\itemdate{Magellan/FourStar} (2020AB,2019AB) 6 nights for near-infrared imaging of galaxy clusters\\
\itemdate{Very Large Array} (2019A) 4.5 h to study AGN feedback in galaxy clusters\\
\itemdate{Giant Metrewave Radio Telescope} (2017B,2013B) 44 h to study diffuse radio emission in clusters\\
\itemdate{Gemini South/GMOS} (2017B) 24 h for optical imaging and spectroscopy of high-redshift galaxy clusters\\
\itemdate{VLT Survey Telescope/OmegaCAM} (2015A) 6 h for optical imaging of galaxy clusters\\


\technical{Observing Experience}
{I have spent roughly 180 hours observing with 
optical (Gemini South/GMOS) and near-infrared (NTT/SofI, Magellan/Fourstar) instruments performing both 
imaging and spectroscopy of galaxy clusters.}


%\sectitle{Language skills}
%
%{Native Spanish, fluent English, basic Dutch}

\sectitle{Community Activity}

\noindent 
\textbf{Journals:} I have served as a referee for Astronomy \& Astrophysics, The Astrophysical Journal, Monthly Notices of the Royal 
    Astronomical Society, and Nature Astronomy.\\
\noindent
\textbf{Telescope Allocation Committees (TACs):} I have served as a reviewer for the Canadian TAC, as well as for the
    \textit{Chandra} X-ray Observatory.

\subsectitle{Informal courses}

\noindent
\itemdate{2016} \emph{Making Better Figures}, Universiteit Leiden (\url{http://bit.ly/2NTznxW})

\subsectitle{Press articles authored}

\noindent
%\itemdate{July 2017} \emph{Galaxy clusters: Falling into line} (Nature Astronomy \emph{News \& Views})\\
%\itemdate{July 2013} \emph{Featured Science: Dynamical masses of galaxy clusters discovered with the Sunyaev-Zel'dovich effect} (Gemini Focus)
\emph{Galaxy clusters: Falling into line} (Nature Astronomy \emph{News \& Views}, July 2017)\\
\emph{Dynamical masses of galaxy clusters discovered with the Sunyaev-Zel'dovich effect} (Gemini Focus \emph{Featured Science}, July 2013)

\subsectitle{Outreach}

\noindent
\itemdates{2018}{2019} Assisted with \emph{Public Astronomical Observations in Spanish}, Princeton University.\\
\itemdates{2013}{2014} Assisted with \emph{Public Observations at the Old Observatory}, Leiden Observatory.\\
\itemdate{2012} Co-taught an \emph{Astronomy Course for Seniors}, PUC.\\
\itemdate{2011} Participated in \emph{Starry Nights}, observation nights for elementary and middle school students in social risk organized by ESO-Santiago.\\
\itemdate{2010} Invited talk on board the ``FFG14 Almirante Latorre'' Chilean Navy ship, Valpara\'iso, Chile.\\
\itemdate{2010} \emph{The Universe}, a series of talks for elementary school students in social risk organized by PUC.\\


\sectitle{Technical skills}

I am an experienced \texttt{python} programmer, and I also have some experience with IRAF/PyRAF. I have written {\tt pygmos}, a Python/PyRAF pipeline to reduce Gemini-GMOS spectra which is available \href{https://github.com/cristobal-sifon/pygmos/}{\texttt{here}}. I also developed an early analysis pipeline for the FLAMINGOS-II infrared imager and spectrograph installed in the Gemini-South telescope. I am one of three lead developers and maintainers of the galaxy-galaxy lensing pipeline used by the KiDS collaboration (written in \texttt{python}, but which is not public at the moment). Other codes I have written are posted at my \href{https://github.com/cristobal-sifon}{\texttt{github}} page.\\


%\vspace{1cm}
%\hline
%\hline
%\vspace{0.5cm}
%\pagebreak


\sectitle{Other Work Experience}

\noindent
\itemdates{2007}{2008} Ski instructor at Homewood Mountain Ski Resort in Lake Tahoe, CA. Obtained certification as \emph{Level I Ski Instructor} by the Professional Ski Instructors of America (PSIA).\\
\itemdates{2006}{2007} Ski lift operator at Sun Valley Resort, Sun Valley, ID.\\


\sectitle{References}

\begin{itemize}
%  \item 
 \item Prof.~Henk Hoekstra (\textit{PhD advisor})\\
       Leiden Observatory, Universiteit Leiden\\
       Niels Bohrweg 2, NL-2333 CA Leiden, The Netherlands\\
       Phone: +31 (71) 527 5594\\
       E-mail: \email{hoekstra@strw.leidenuniv.nl}
 \item Prof.~David N.~Spergel\\
       Center for Computational Astrophysics, Flatiron Institute\\
       160 Fifth Avenue, 7th Floor, New York, NY 10010, USA\\
       Phone: +1 (646) 654 0066\\
       E-mail: \email{dns@astro.princeton.edu}
 \item Prof.~John P.~Hughes\\
       Department of Physics and Astronomy, Rutgers University\\
       136 Frelinghuysen Rd., Piscataway, NJ 08854, USA\\
       Phone: +1 (848) 445 8878\\
       E-mail: \email{jph@physics.rutgers.edu}
 \item Prof.~L.~Felipe Barrientos (\textit{MSc advisor})\\
       Instituto de Astrof\'isica, P. Universidad Cat\'olica de Chile\\
       Casilla 306, Santiago 22, Chile\\
       Phone: +56 (2) 2354 4941\\
       E-mail: \email{barrientos@astro.uc.cl}
 \item Prof.~Felipe Menanteau\\
       Department of Astronomy, University of Illinois at Urbana-Champaign\\
       1002 W.\ Green St., Urbana, IL 61801, USA\\
       Phone: +1 (217) 244 6297\\
       E-mail: \email{felipe@illinois.edu}
\end{itemize}

\vspace{0.3cm}
\hline


%%%%%%%%%%%%%%%%%%%%%%%%%%%%%%%%%%%%%%
%%%%%%%%%%%%%%%%%%%%%%%%%%%%%%%%%%%%%%
%%%%%%%%%% PUBLICATION LIST %%%%%%%%%%
%%%%%%%%%%%%%%%%%%%%%%%%%%%%%%%%%%%%%%
%%%%%%%%%%%%%%%%%%%%%%%%%%%%%%%%%%%%%%

%% Uncomment these two lines to get the publication list in the main pdf

%\pagebreak
%\title{Recent Publications}


\noindent
I have co-authored 80 scientific articles intended for peer-reviewed 
publication, including 7 first-author papers. They have been cited more than 
3,700 times and have an $h$-index of 35, with more than 300 citations on my 
first-author papers. My publications include three companion reviews on galaxy 
alignments written for a special issue of Space Science Reviews (B.\ Joachimi et 
al.\ 2015, A.\ Kiessling et al.\ 2015, D.\ Kirk et al.\ 2015). The full list of 
publications can be accessed at \href{https://goo.gl/LAu9G4}{this url}. I also 
wrote an invited `News \& Views' article for the 4 July 2017 edition of Nature 
Astronomy, accessible 
\href{https://www.nature.com/articles/s41550-017-0181}{here}.
%
This document is maintained live on
\href{https://github.com/cristobal-sifon/cv/blob/master/Sifon_publications.pdf}{\texttt{github}}.




\begin{etaremune}

\item
B.~Fuzia \etalwithme{22}
\paper{The Atacama Cosmology Telescope: SZ-Based Masses and Dust Emission from IR-Selected Cluster Candidates in the SHELA Survey},
2020, \href{https://ui.adsabs.harvard.edu/abs/2020arXiv200109587F/abstract}{arXiv:2001.09587},
\submitted{\mnras}

\item
S.~Huang \etalwithme{12}
%A.~Leauthaud, A.~Hearin, P.~Behroozi, C.~Bradshaw, F.~Ardila, J.~Speagle,
%A.~Tenneti, K.~Bundy, J.~Greene, \myself, N.~Bahcall,
\paper{Weak Lensing Reveals a Tight Connection Between Dark Matter Halo Mass and the Distribution of Stellar Mass in Massive Galaxies},
2020, \href{https://ui.adsabs.harvard.edu/abs/2020MNRAS.492.3685H/abstract}{\mnras, 492, 3685}
\arxiv{1811.01139}

\item
Q.~Xia \etalwithme{13}
\paper{A Gravitational Lensing Detection of Filamentary Structures Connecting Luminous Red Galaxies},
2020, \href{https://ui.adsabs.harvard.edu/abs/2020A&A...633A..89X/abstract}{\aap, 633, 89}
\arxiv{1909.05852}

\item
H.~Hildebrandt \etalwithme{28}
\paper{KiDS+VIKING-450: Cosmic Shear Tomography with Optical+infrared Data},
2020, \href{https://ui.adsabs.harvard.edu/abs/2020A&A...633A..69H/abstract}{\aap, 633, 69}
\arxiv{1812.06076}

\item
R.~Herbonnet, \myself, H.~Hoekstra, Y.~Bah\'e, R.~F.~J.~van~der~Burg, J.-B.~Melin, A.~von~der~Linden, D.~Sand, S.~Kay, D.~Barnes,
\paper{CCCP and MENeaCS: (Updated) Weak-Lensing Masses for 100 Galaxy Clusters},
2019, \href{https://ui.adsabs.harvard.edu/abs/2019arXiv191204414H/abstract}{arXiv:1912.04414},
\submitted{\mnras}

\item
M.~Madhavacheril \etalwithme{49}
\paper{The Atacama Cosmology Telescope: Component-Separated Maps of CMB Temperature and the Thermal Sunyaev-Zel'dovich Effect},
2019, \href{https://ui.adsabs.harvard.edu/abs/2019arXiv191105717M/abstract}{arXiv:1911.05717},
\submitted{\prd}

\item
Y.~Rong \etalwithme{13}
\paper{Intrinsic Morphology Evolution of Ultra-diffuse Galaxies},
2019, \href{https://ui.adsabs.harvard.edu/abs/2019arXiv190710079R/abstract}{arXiv:1907.10079},
\submitted{\apj}

\item
C.~Hikage \etalwithme{30}
\paper{Cosmology from Cosmic Shear Power Spectra with Subaru Hyper Suprime-Cam First-Year Data},
2019, \href{http://adsabs.harvard.edu/abs/2019PASJ...71...43H}{\pasj, 71, 43}
\arxiv{1809.09148}

\item
H.~Miyatake \etalwithme{58}
\paper{Weak-Lensing Mass Calibration of ACTPol Sunyaev-Zel'dovich Clusters with the Hyper Suprime-Cam Survey},
2019, \href{http://adsabs.harvard.edu/abs/2019ApJ...875...63M}{\apj, 875, 63}
\arxiv{1804.05873}

\item
K.~Knowles \etalwithme{14}
\paper{GMRT 610 MHz Observations of Galaxy Clusters in the ACT Equatorial Sample},
2019, \href{http://adsabs.harvard.edu/abs/2019MNRAS.486.1332K}{\mnras, 486, 1332}
\arxiv{1806.09579}

\item
\myself, R.~Herbonnet, H.~Hoekstra, R.~F.~J.~van~der~Burg, M.~Viola,
\paper{The Galaxy-Subhalo Connection in Low-Redshift Galaxy Clusters from Weak Gravitational Lensing},
2018, \href{http://adsabs.harvard.edu/abs/2018MNRAS.478.1244S}{\mnras, 478, 1244}
\arxiv{1706.06125}

\item
\myself, R.~F.~J.~van~der~Burg, H.~Hoekstra, A.~Muzzin, R.~Herbonnet,
\paper{A First Constraint on the Average Mass of Ultra Diffuse Galaxies from Weak Gravitational Lensing},
2018, \href{http://adsabs.harvard.edu/abs/2018MNRAS.473.3747S}{\mnras, 473, 3747}
\arxiv{1704.07847}

\item
M.~Hilton, M.~Hasselfield, \myself, \etal{43}
\paper{The Atacama Cosmology Telescope: The Two-Season ACTPol Sunyaev-Zel'dovich Effect Selected Cluster Catalog},
2018, \href{http://adsabs.harvard.edu/abs/2018ApJS..235...20H}{\apjs, 235, 20}
\arxiv{1709.05600}


\end{etaremune}



\end{document}

\documentclass[11pt]{article}


\usepackage{bm}
\usepackage{enumerate}
\usepackage{etaremune}
%\usepackage{fullpage}
\usepackage{graphicx}
\usepackage{hyperref}

\addtolength{\oddsidemargin}{-1in}
\addtolength{\evensidemargin}{-1in}
\addtolength{\textwidth}{2in}
\addtolength{\topmargin}{-1.3in}
\addtolength{\textheight}{2.2in}

% For links of references
\hypersetup{colorlinks,
  linkcolor=blue,
  filecolor=blue,
  urlcolor=blue,
  citecolor=blue}

\renewcommand{\familydefault}{\sfdefault}
\usepackage{chancery}

\newcommand\sectitle[1]{
  \hline
  \vspace{0.5cm}
  \noindent
  %\underline{
    \textbf{\Large #1}
  %}
  \\
  \vspace{-0.2cm}
}

\newcommand\subsectitle[1]{
  \vspace{0.3cm}
  \noindent
  %\underline{
    \textbf{\large #1}
  %}
  \\
  \vspace{-0.3cm}
}

\newcommand\technical[2]{
  \noindent
    {\large\bf #1:} #2\\
  }

\newcommand\itemdate[1]{\textbf{[#1]}}
\newcommand\itemdates[2]{\textbf{[#1 -- #2]}}
\newcommand\email[1]{\href{mailto:#1}{\texttt{#1}}}
\newcommand\pucv{Pontificia Universidad Cat\'olica de Valpara\'iso}


%not used if publication list not shown

\def\aap{A\&A}
\def\apj{ApJ}
\def\apjs{ApJS}
\def\baas{\textit{Bull.\ of the Am.\ Ast.\ Soc.}}
\def\gemfoc{Gemini Focus}
\def\jcap{JCAP}
\def\mnras{MNRAS}
\def\msngr{The Messenger}
\def\pasj{PASJ}
\def\prd{Phys.\ Rev.\ D}
\def\prl{Phys.\ Rev.\ Letters}
\def\ssr{Space Sci.\ Rev.}

\newcommand{\myself}{\textbf{\color{red} C.~Sif\'on}}
\newcommand\includemyself{\textbf\small{(including C.~Sif\'on)}}
\newcommand{\accepted}[1]{accepted for publication in #1}
\newcommand{\etal}[1]{et al.\ (#1 co-authors),}
\newcommand{\etalwithme}[1]{et al.\ (#1 coauthors incl.\ \myself),}
\newcommand{\paper}[1]{\textbf{``#1''}}
\newcommand{\submitted}[1]{submitted to #1}
\renewcommand{\title}[1]{\noindent\textbf{\huge #1}\\}
% arXiv links
\newcommand{\arxiv}[1]
    {\href{https://arxiv.org/abs/#1}{\texttt{\color{magenta}[arXiv]}}}




\begin{document}

% \begin{figure}[t]
\begin{minipage}[b]{0.46\linewidth}
\flushleft
% \noindent
\hspace{-0.7cm}
{\bf\huge Crist\'obal Sif\'on}\\\vspace{0.2cm}
\hspace{-0.5cm}{\large Profesor Asociado\\
\hspace{-0.65cm}        Instituto de F\'isica, Facultad de Ciencias\\
\hspace{-0.65cm}        \pucv\\
\hspace{-0.65cm}        Casilla 4059, Valpara\'iso, Chile}\\
\end{minipage}
\begin{minipage}[b]{0.49\linewidth}
\flushright
{\large E-mail: {\texttt cristobal.sifon@pucv.cl}\\
        Phone: +56 (32) 227 4698\\
        \url{http://www.astro.princeton.edu/~sifon/}
        \url{https://github.com/cristobal-sifon/}}
\end{minipage}
\vspace{0.4cm}
\hline



\sectitle{Research Interests}

My research focuses on galaxy cluster physics including observable--mass scaling relations for cosmological analyses, brightest cluster galaxies, the mass content of cluster galaxies, and merging clusters. I am also interested in intrinsic galaxy alignments, both as contaminants for cosmic shear and as a physical mechanism in their own right. I use various tools and techniques to study these phenomena, including weak gravitational lensing, spectroscopy, the exploitation of optical surveys in general, and most recently analyses involving hydrodynamical simulations.

\vspace{0.5cm}
\technical{Collaborations}
{
 Atacama Cosmology Telescope (ACT) ---
 Canadian Cluster Comparison Project (CCCP) ---
 Galaxy Cluster Mass Reconstruction Project ---
 Hyper-Suprime Cam survey (HSC) ---
 Kilo-Degree Survey (KiDS) ---
 Large Synoptic Survey Telescope Dark Energy Science Collaboration (LSST-DESC) ---
 Multi-Epoch Nearby Cluster Survey (MENeaCS) ---
 Simons Observatory.
}


\sectitle{Employment}

\noindent
\itemdates{2019}{Present} Profesor Asociado, \pucv\ (PUCV), Chile\\
\itemdates{2016}{2019} Postdoctoral Research Associate, Princeton University, USA


\subsectitle{Education}

\noindent
\itemdates{2012}{2016} Ph.D.~Astrophysics, Universiteit Leiden, The Netherlands\\
\itemdates{2010}{2012} M.Sc.~Astrophysics, P.~Universidad Cat\'olica de Chile (PUC), Chile\\
\itemdates{2005}{2010} B.Sc.~Astronomy, P.~Universidad Cat\'olica de Chile, Chile


\subsectitle{Internships}

\noindent
\itemdate{2011} Science Intern, Gemini South Observatory (6 months)\\
\itemdate{2011} Internship, Rutgers University (2 months)\\
\itemdate{2009} Science Intern, Gemini South Observatory (6 months, \emph{B.Sc.\ thesis})\\

%%%

\sectitle{Teaching \& Mentoring}

\subsectitle{Student Research Mentoring}

\noindent
\itemdates{2018}{Present} Malik Walker, Princeton University: Undegraduate Summer Research Program and Junior Project.\\
\itemdates{2017}{Present} Naomi Robertson, Oxford University (UK): co-advising PhD thesis project.\\
\itemdates{2013}{2014} Joshua Albert, Universiteit Leiden: co-advised MSc thesis project. 

\subsectitle{Teaching Assistant}

%\noindent
%\itemdate{2013B} Stellar Dynamics (Leiden, Prof.~S.~Portegies Zwart)\\
%\itemdate{2012A} Extragalactic Astrophysics (PUC, Prof.~L.~F.~Barrientos)\\
%\itemdate{2011A} Extragalactic Astrophysics (PUC, Prof.~L.~F.~Barrientos)\\
%\itemdate{2011A} Laboratory of Thermodynamics and Kinetic Theory (PUC, Prof.~U.~Volkmann)\\
%\itemdate{2010B} Experimental Astrophysics (PUC, Prof.~L.~F.~Barrientos)
\noindent
\itemdate{Leiden} Stellar dynamics; organizer of MSc thesis defense presentations\\
\itemdate{PUC} Extragalactic astrophysics; Experimental astrophysics; Laboratory of thermodynamics and kinetic theory\\

\hline

\pagebreak


\sectitle{Successful Observing Proposals (as PI)}

\noindent
I have been the PI of 7 different successful observing proposals in 5 different telescopes:

\noindent
\itemdate{Magellan/FourStar} (2019A) 2 nights for near-infrared imaging of galaxy clusters\\
\itemdate{Very Large Array} (2019A) 4.5 h to study AGN feedback in galaxy clusters\\
\itemdate{Giant Metrewave Radio Telescope} (2017B,2013B) 44 h to study diffuse radio emission in clusters\\
\itemdate{Gemini South/GMOS} (2017B) 24 h for optical imaging and spectroscopy of high-redshift galaxy clusters\\
\itemdate{VLT Survey Telescope/OmegaCAM} (2015A) 6 h for optical imaging of galaxy clusters\\


\technical{Observing Experience}
{I have spent roughly 180 hours observing with 
optical (Gemini South/GMOS) and near-infrared (La Silla-2.2m) instruments performing both 
imaging and spectroscopy of galaxy clusters.}


%\sectitle{Language skills}
%
%{Native Spanish, fluent English, basic Dutch}

\sectitle{Community Activity}

\noindent
I have served as a referee for Astronomy \& Astrophysics, The Astrophysical Journal, Monthly Notices of the Royal Astronomical Society, and Nature Astronomy.

\subsectitle{Informal courses}

\noindent
\itemdate{2016} \emph{Making Better Figures}, Universiteit Leiden (\url{http://bit.ly/2NTznxW})

\subsectitle{Press articles authored}

\noindent
%\itemdate{July 2017} \emph{Galaxy clusters: Falling into line} (Nature Astronomy \emph{News \& Views})\\
%\itemdate{July 2013} \emph{Featured Science: Dynamical masses of galaxy clusters discovered with the Sunyaev-Zel'dovich effect} (Gemini Focus)
\emph{Galaxy clusters: Falling into line} (Nature Astronomy \emph{News \& Views}, July 2017)\\
\emph{Dynamical masses of galaxy clusters discovered with the Sunyaev-Zel'dovich effect} (Gemini Focus \emph{Featured Science}, July 2013)

\subsectitle{Outreach}

\noindent
\itemdates{2018}{Present} Assisting with \emph{Public Astronomical Observations in Spanish}, Princeton University.\\
\itemdates{2013}{2014} Assisted with \emph{Public Observations at the Old Observatory}, Leiden Observatory.\\
\itemdate{2012} Co-taught an \emph{Astronomy Course for Seniors}, PUC.\\
\itemdate{2011} Participated in \emph{Starry Nights}, observation nights for elementary and middle school students in social risk organized by ESO-Santiago.\\
\itemdate{2010} Invited talk on board the ``FFG14 Almirante Latorre'' Chilean Navy ship, Valpara\'iso, Chile.\\
\itemdate{2010} \emph{The Universe}, a series of talks for elementary school students in social risk organized by PUC.\\


\sectitle{Technical skills}

I am an experienced \texttt{python} programmer, and I also have some experience with IRAF/PyRAF. I have written {\tt pygmos}, a Python/PyRAF pipeline to reduce Gemini-GMOS spectra which is available \href{https://github.com/cristobal-sifon/pygmos/}{\texttt{here}}. I also developed an early analysis pipeline for the FLAMINGOS-II infrared imager and spectrograph installed in the Gemini-South telescope. I am one of three lead developers and maintainers of the galaxy-galaxy lensing pipeline used by the KiDS collaboration (written in \texttt{python}, but which is not public at the moment). Other codes I have written are posted at my \href{https://github.com/cristobal-sifon}{\texttt{github}} page.\\


\sectitle{Other Work Experience}

\noindent
\itemdates{2007}{2008} Ski instructor at Homewood Mountain Ski Resort in Lake Tahoe, CA. Obtained certification as \emph{Level I Ski Instructor} by the Professional Ski Instructors of America (PSIA).\\
\itemdates{2006}{2007} Ski lift operator at Sun Valley Resort, Sun Valley, ID.\\

%\vspace{1cm}
%\hline
\hline
\vspace{0.5cm}

\pagebreak


\sectitle{References}

\begin{itemize}
%  \item 
 \item Prof.~Henk Hoekstra (\textit{PhD advisor})\\
       Leiden Observatory, Universiteit Leiden\\
       Niels Bohrweg 2, NL-2333 CA Leiden, The Netherlands\\
       Phone: +31 (71) 527 5594\\
       E-mail: \email{hoekstra@strw.leidenuniv.nl}
 \item Prof.~David N.~Spergel\\
       Center for Computational Astrophysics, Flatiron Institute\\
       160 Fifth Avenue, 7th Floor, New York, NY 10010, USA\\
       Phone: +1 (646) 654 0066\\
       E-mail: \email{dns@astro.princeton.edu}
 \item Prof.~John P.~Hughes\\
       Department of Physics and Astronomy, Rutgers University\\
       136 Frelinghuysen Rd., Piscataway, NJ 08854, USA\\
       Phone: +1 (848) 445 8878\\
       E-mail: \email{jph@physics.rutgers.edu}
 \item Prof.~L.~Felipe Barrientos (\textit{MSc advisor})\\
       Instituto de Astrof\'isica, P. Universidad Cat\'olica de Chile\\
       Casilla 306, Santiago 22, Chile\\
       Phone: +56 (2) 2354 4941\\
       E-mail: \email{barrientos@astro.uc.cl}
 \item Prof.~Felipe Menanteau\\
       Department of Astronomy, University of Illinois at Urbana-Champaign\\
       1002 W.\ Green St., Urbana, IL 61801, USA\\
       Phone: +1 (217) 244 6297\\
       E-mail: \email{felipe@illinois.edu}
\end{itemize}

\vspace{0.3cm}
\hline


%%%%%%%%%%%%%%%%%%%%%%%%%%%%%%%%%%%%%%
%%%%%%%%%%%%%%%%%%%%%%%%%%%%%%%%%%%%%%
%%%%%%%%%% PUBLICATION LIST %%%%%%%%%%
%%%%%%%%%%%%%%%%%%%%%%%%%%%%%%%%%%%%%%
%%%%%%%%%%%%%%%%%%%%%%%%%%%%%%%%%%%%%%

%% Uncomment these two lines to get the publication list in the main pdf

\pagebreak
%%% ADS custom formats %%%
% Refereed:
% %N,\n\{\\bf "%T"\},\n%Y,\\href{\%u\}\{%q,%V,%P\}
% arXiv:
% %N,\n\{\\bf "%T"\},%Y,\\href{\%u\}\{%Q\}
% 
% New version:
% %I,\n\{\\bf ``%T''\},\n%Y,\\href{\%u\}\{%q,%V,%P\}
%%% or use the file $HOME/Documents/papers/authorlist.py to get the author list formatted
%%% this way from the pdf

\title{Publication list}


\noindent
I have co-authored 80 scientific articles intended for peer-reviewed 
publication, including 7 first-author papers. They have been cited more than 
3,700 times and have an $h$-index of 35, with more than 300 citations on my 
first-author papers. My publications include three companion reviews on galaxy 
alignments written for a special issue of Space Science Reviews (B.\ Joachimi et 
al.\ 2015, A.\ Kiessling et al.\ 2015, D.\ Kirk et al.\ 2015). The full list of 
publications can be accessed at \href{https://goo.gl/LAu9G4}{this url}. I also 
wrote an invited `News \& Views' article for the 4 July 2017 edition of Nature 
Astronomy, accessible 
\href{https://www.nature.com/articles/s41550-017-0181}{here}.
%
This document is maintained live on
\href{https://github.com/cristobal-sifon/cv/blob/master/Sifon_publications.pdf}{\texttt{github}}.




\section*{First-Author Papers}

\begin{etaremune}
    
\item
\myself, J.~Han,
\paper{The history and mass content of cluster galaxies in the EAGLE simulation},
2024, \href{https://ui.adsabs.harvard.edu/abs/2024A&A...686A.163S/abstract}{\aap, 686, A163}
\arxiv{2312.12529}

\item
\myself, R.~Herbonnet, H.~Hoekstra, R.~F.~J.~van~der~Burg, M.~Viola,
\paper{The Galaxy-Subhalo Connection in Low-Redshift Galaxy Clusters from Weak 
Gravitational Lensing},
2018, \href{https://ui.adsabs.harvard.edu/abs/2018MNRAS.478.1244S}{\mnras, 478, 1244}
\arxiv{1706.06125}

\item
\myself, R.~F.~J.~van~der~Burg, H.~Hoekstra, A.~Muzzin, R.~Herbonnet,
\paper{A First Constraint on the Average Mass of Ultra Diffuse Galaxies from 
Weak Gravitational Lensing},
2018, \href{https://ui.adsabs.harvard.edu/abs/2018MNRAS.473.3747S}{\mnras, 473, 3747}
\arxiv{1704.07847}

\item 
\myself\ \etal{25}
\paper{The Atacama Cosmology Telescope: Dynamical Masses for 44 SZ-Selected 
Galaxy Clusters over 755 Square Degrees},
2016, \href{https://ui.adsabs.harvard.edu/abs/2016MNRAS.461..248S}{\mnras, 461, 248}
\arxiv{1512.00910}

\item
\myself\ \etal{26}
\paper{The Masses of Satellites in GAMA Galaxy Groups from 100 Square Degrees of 
KiDS Weak Lensing Data},
2015, \href{https://ui.adsabs.harvard.edu/abs/2015MNRAS.454.3938S}{\mnras, 454, 3938}
\arxiv{1507.00737}

\item
\myself, H.~Hoekstra, M.~Cacciato, M.~Viola, F.~K\"ohlinger, 
R.~F.~J.~van~der~Burg, D.~J.~Sand, M.~L.~Graham,
\paper{Constraints on the Alignments of Galaxies in Galaxy Clusters from 
$\sim$14,000 Spectroscopic Members},
2015, \href{https://ui.adsabs.harvard.edu/abs/2015A&A...575A..48S}{\aap, 575, A48}
\arxiv{1406.5196}

\item
\myself, F.~Menanteau, J.~P.~Hughes, M.~Carrasco, L.~F.~Barrientos,
\paper{Strong Lensing Analysis of PLCK G004.5$-$19.5, a Planck-Discovered 
Cluster Hosting a Radio Relic at $z=0.52$},
2014, \href{https://ui.adsabs.harvard.edu/abs/2014A&A...562A..43S}{\aap, 562, A43}
\arxiv{1304.0686}

\item 
\myself\ \etal{36}
\paper{The Atacama Cosmology Telescope: Dynamical Masses and Scaling Relations 
for a Sample of Massive Sunyaev-Zel'dovich Effect Selected Galaxy Clusters},
2013, \href{https://ui.adsabs.harvard.edu/abs/2013ApJ...772...25S}{\apj, 772, 25}
\arxiv{1201.0991}

\end{etaremune}


\section*{Major Contributor Papers}

\begin{etaremune}
    
\item
M.~Hilton, \myself, \etal{133}
\paper{The Atacama Cosmology Telescope: a Catalog of $>$4000 Sunyaev-Zel’dovich 
Galaxy Clusters},
2020, \href{https://ui.adsabs.harvard.edu/abs/2020arXiv200911043H/abstract}{arXiv:2009.11043},
\submitted{\apjs}

\item
M.~S.~Madhavacheril, \myself, \etal{61}
\paper{The Atacama Cosmology Telescope: Weighing Distant Clusters with the Most 
Ancient Light},
2020, \href{https://ui.adsabs.harvard.edu/abs/2020arXiv200907772M/abstract}{\apjl, 903, 13}
\arxiv{2009.07772}

\item

R.~Herbonnet, \myself, H.~Hoekstra, Y.~Bah\'e, R.~F.~J.~van~der~Burg, 
J.-B.~Melin, A.~von~der~Linden, D.~Sand, S.~Kay, D.~Barnes,
\paper{CCCP and MENeaCS: (Updated) Weak-Lensing Masses for 100 Galaxy Clusters},
2020, \href{https://ui.adsabs.harvard.edu/abs/2020MNRAS.497.4684H/abstract}{\mnras, 497, 4684}
\arxiv{1912.04414}

\item
M.~Hilton, M.~Hasselfield, \myself, \etal{43}
\paper{The Atacama Cosmology Telescope: The Two-Season ACTPol Sunyaev-Zel'dovich 
Effect Selected Cluster Catalog},
2018, \href{https://ui.adsabs.harvard.edu/abs/2018ApJS..235...20H}{\apjs, 235, 20}
\arxiv{1709.05600}

\item
J.~G.~Albert, \myself, A.~Stroe, F.~Mernier, H.~T.~Intema, H.~J.~A.~R\"ottgering, 
G.~Brunetti,
\paper{Complex Diffuse Emission in the $z=0.52$ Cluster PLCK G004.5$-$19.5},
2017, \href{https://ui.adsabs.harvard.edu/abs/2017A&A...607A...4A}{\aap, 607, A4}
\arxiv{1708.00789}

\item
R.~F.~J.~van~der~Burg, H.~Hoekstra, A.~Muzzin, \myself, \etal{17}
\paper{The Abundance of Ultra-Diffuse Galaxies from Groups to Clusters: UDGs Are 
Relatively More Common in More Massive Haloes},
2017, \href{https://ui.adsabs.harvard.edu/abs/2017A&A...607A..79V}{\aap, 607, A79}
\arxiv{1706.02704}

\item
E.~van~Uitert, M.~Cacciato, H.~Hoekstra, M.~Brouwer, \myself, \etal{29}
\paper{The Stellar-to-Halo Mass Relation of GAMA Galaxies from 100 Square 
Degrees of KiDS Weak Lensing Data},
2016, \href{https://ui.adsabs.harvard.edu/abs/2016MNRAS.459.3251V}{\mnras, 459, 3251}
\arxiv{1601.06791}

\item
D.~Kirk, M.~L.~Brown, H.~Hoekstra, B.~Joachimi, T.~D.~Kitching, R.~Mandelbaum, 
\myself, M.~Cacciato, A.~Choi, A.~Kiessling, A.~Leonard, A.~Rassat, 
B.~Malte~Sch\"afer,
\paper{Galaxy Alignments: Observations and Impact on Cosmology},
2015, \href{https://ui.adsabs.harvard.edu/abs/2015SSRv..193..139K/abstract}{\ssr, 193, 139}
\arxiv{1504.05465}

\item
A.~Kiessling, M.~Cacciato, B.~Joachimi, D.~Kirk, T.~D.~Kitching, A.~Leonard, 
R.~Mandelbaum, B.~Malte~Sch\"afer, \myself, M.~L.~Brown, A.~Rassat,
\paper{Galaxy Alignments: Theory, Modelling \& Simulations},
2015, \href{https://ui.adsabs.harvard.edu/abs/2015SSRv..193...67K/abstract}{\ssr, 193, 67}
\arxiv{1504.05546}

\item
B.~Joachimi, M.~Cacciato, T.~D.~Kitching, A.~Leonard, R.~Mandelbaum, 
B.~Malte~Sch\"afer, \myself, H.~Hoekstra, A.~Kiessling, D.~Kirk, A.~Rassat,
\paper{Galaxy Alignments: an Overview},
2015, \href{https://ui.adsabs.harvard.edu/abs/2015SSRv..193....1J/abstract}{\ssr, 193, 1}
\arxiv{1504.05456}

\item
R.~F.~J.~van~der~Burg, H.~Hoekstra, A.~Muzzin, \myself, M.~L.~Balogh, S.~McGee,
\paper{Evidence for the Inside-Out Growth of the Stellar Mass Distribution in 
Galaxy Clusters since $z\sim1$},
2015, \href{https://ui.adsabs.harvard.edu/abs/2015A&A...577A..19V}{\aap, 577, 19}
\arxiv{1412.2137}

\item
M.~Hilton, M.~Hasselfield, \myself, \etal{26}
\paper{The Atacama Cosmology Telescope: The Stellar Content of Galaxy Clusters 
Selected Using the Sunyaev-Zel'dovich Effect},
2013, \href{https://ui.adsabs.harvard.edu/abs/2013MNRAS.435.3469H/abstract}{\mnras, 435, 3469}
\arxiv{1301.0780}

\item
F.~Menanteau, \myself, \etal{26}
\paper{The Atacama Cosmology Telescope: Physical Properties of 
Sunyaev-Zel'dovich Effect Clusters on the Celestial Equator},
2013, \href{https://ui.adsabs.harvard.edu/abs/2013ApJ...765...67M/abstract}{\apj, 765, 67}
\arxiv{1210.4048}

\item
F.~Menanteau, J.~P.~Hughes, \myself, \etal{27}
\paper{The Atacama Cosmology Telescope: ACT-CL J0102--4915 ``El Gordo,'' a 
Massive Merging Cluster at Redshift 0.87},
2012, \href{https://ui.adsabs.harvard.edu/abs/2012ApJ...748....7M/abstract}{\apj, 748, 7}
\arxiv{1109.0953}


\end{etaremune}


\section*{Contributing Author Papers {\small (All including \myself)}}

\begin{etaremune}
    %% \item\n%8.1O,\n\paper{%T},\n%Y, \href{%u}{%j, %V, %p}\n\arxiv{%X}\n

%%%%%%%%%%%%%%%%%
%%% Submitted %%%
%%%%%%%%%%%%%%%%%

\subsection*{Submitted}

\begin{etaremune}

\item
F.~J.~Qu, and 53 colleagues,
\paper{The Atacama Cosmology Telescope DR6 and DESI: Structure growth measurements from the cross-correlation of DESI Legacy Imaging galaxies and CMB lensing from ACT DR6 and \textit{Planck} PR4},
2024, \href{https://ui.adsabs.harvard.edu/abs/2024arXiv241010808Q}{arXiv:2410.10808}
\submitted{\apj}

\item
F.~McCarthy, and 29 colleagues,
\paper{The Atacama Cosmology Telescope: Large-scale velocity reconstruction with the kinematic Sunyaev--Zel'dovich effect and DESI LRGs},
2024, \href{https://ui.adsabs.harvard.edu/abs/2024arXiv241006229M}{arXiv:2410.06229}
%\arxiv{2410.06229}
\submitted{\mnras}

\item
E.~Schiappucci, and 10 colleagues,
\paper{Constraining cosmological parameters using the pairwise kinematic Sunyaev-Zel'dovich effect with CMB-S4 and future galaxy cluster surveys},
2024, \href{https://ui.adsabs.harvard.edu/abs/2024arXiv240918368S}{arXiv:2409.18368}
%\arxiv{2409.18368}
\submitted{\prd}

\item
E.~K.~Biermann, and 26 colleagues,
\paper{The Atacama Cosmology Telescope: Systematic Transient Search of Single Observation Maps},
2024, \href{https://ui.adsabs.harvard.edu/abs/2024arXiv240908429B}{arXiv:2409.08429}
%\arxiv{2409.08429}
\submitted{\apj}

\item
M.~Lokken, and 75 colleagues,
\paper{Superclustering with the Atacama Cosmology Telescope and Dark Energy Survey: II.~Anisotropic large-scale coherence in hot gas, galaxies, and dark matter},
2024, \href{https://ui.adsabs.harvard.edu/abs/2024arXiv240904535L}{arXiv:2409.04535}
%\arxiv{2409.04535}
\submitted{\apj}

\item
G.~S.~Farren, and 18 colleagues,
\paper{The Atacama Cosmology Telescope: Multi-probe cosmology with unWISE galaxies and ACT DR6 CMB lensing},
2024, \href{https://ui.adsabs.harvard.edu/abs/2024arXiv240902109F}{arXiv:2409.02109}
%\arxiv{2409.02109}
\submitted{\apj}

\item
B.~Hadzhiyska, and 72 colleagues,
\paper{Evidence for large baryonic feedback at low and intermediate redshifts from kinematic Sunyaev-Zel'dovich observations with ACT and DESI photometric galaxies},
2024, \href{https://ui.adsabs.harvard.edu/abs/2024arXiv240707152H}{arXiv:2407.07152}
\submitted{\prl}

\item
N.~Sailer, and 65 colleagues,
\paper{Cosmological constraints from the cross-correlation of DESI Luminous Red Galaxies with CMB lensing from Planck PR4 and ACT DR6},
2024, \href{https://ui.adsabs.harvard.edu/abs/2024arXiv240704607S}{arXiv:2407.04607}
\submitted{\jcap}

\item
J.~Kim, and 72 colleagues,
\paper{The Atacama Cosmology Telescope DR6 and DESI: Structure formation over cosmic time with a measurement of the cross-correlation of CMB Lensing and Luminous Red 
Galaxies},
2024, \href{https://ui.adsabs.harvard.edu/abs/2024arXiv240704606K}{arXiv:2407.04606}
\submitted{\jcap}

\item
L.~Wenzl, and 26 colleagues,
\paper{The Atacama Cosmology Telescope: DR6 Gravitational Lensing and SDSS BOSS cross-correlation measurement and constraints on gravity with the $E_G$ statistic},
2024, \href{https://ui.adsabs.harvard.edu/abs/2024arXiv240512795W}{arXiv:2405.12795}
\submitted{\prd}

\item
W.~R.~Coulton, and 39 colleagues,
\paper{The Atacama Cosmology Telescope: Detection of Patchy Screening of the Cosmic Microwave Background},
2024, \href{https://ui.adsabs.harvard.edu/abs/2024arXiv240113033C}{arXiv:2401.13033}
%\arxiv{arXiv:2401.13033}

\item
C.~Vargas, and 20 colleagues
\paper{The Atacama Cosmology Telescope: Extragalactic Point Sources in the Southern Surveys at 150, 220 and 280 GHz observed between 2008-2010},
2023, \href{https://ui.adsabs.harvard.edu/abs/2023arXiv231017535V}{arXiv:2310.17535}
\submitted{\apj}

\item
W.~Luo, and 13 colleagues
\paper{Dark matter halos of luminous AGNs from galaxy-galaxy lensing with the HSC Subaru Strategic Program},
2022, \href{https://ui.adsabs.harvard.edu/abs/2022arXiv220403817L}{arXiv:2204.03817}
\submitted{\mnras}

\end{etaremune}

%%%%%%%%%%%%%%%%
%%% Accepted %%%
%%%%%%%%%%%%%%%%

%\subsection*{Accepted for publication}

%\begin{etaremune}


%\end{etaremune}


%%%%%%%%%%%%%%%%%
%%% Published %%%
%%%%%%%%%%%%%%%%%

\subsection*{Published}

\begin{etaremune}

\item
P.~Doze, and 12 colleagues,
\paper{A Multiwavelength Approach to Constraining the Merger Properties of ACT-CL J0034.4+0225},
2024, \href{https://ui.adsabs.harvard.edu/abs/2024ApJ...974...49D}{\apj, 974, 49}

\item
J.~van Marrewijk, and 20 colleagues,
\paper{XLSSC 122 caught in the act of growing up: Spatially resolved SZ observations of a z = 1.98 galaxy cluster},
2024, \href{https://ui.adsabs.harvard.edu/abs/2024A&A...689A..41V}{\aap, 689, A41}
\arxiv{2310.06120}

\item
N.~MacCrann, and 23 colleagues,
\paper{The Atacama Cosmology Telescope: Reionization kSZ trispectrum methodology and limits},
2024, \href{https://ui.adsabs.harvard.edu/abs/2024MNRAS.532.4247M}{\mnras, 532, 4247}
\arxiv{2405.01188}

\item
F.~Zhong, and 27 colleagues,
\paper{Galaxy Spectra neural Network (GaSNet). II.~Using deep learning for spectral classification and redshift predictions},
2024, \href{https://ui.adsabs.harvard.edu/abs/2024MNRAS.532..643Z}{\mnras, 532, 643}
\arxiv{2311.04146}

\item
A.~H.~Wright, and 53 colleagues,
\paper{The fifth data release of the Kilo Degree Survey: Multi-epoch optical/NIR imaging covering wide and legacy-calibration fields},
2024, \href{https://ui.adsabs.harvard.edu/abs/2024A&A...686A.170W}{\aap, 686, A170}

\item
G.~S.~Farren, and 36 colleagues,
\paper{The Atacama Cosmology Telescope: Cosmology from Cross-correlations of unWISE Galaxies and ACT DR6 CMB Lensing},
2024, \href{https://ui.adsabs.harvard.edu/abs/2024ApJ...966..157F}{\apj, 966, 157}
\arxiv{2309.05659}

\item
N.~MacCrann, and 49 colleagues,
\paper{The Atacama Cosmology Telescope: Mitigating the Impact of Extragalactic Foregrounds for the DR6 Cosmic Microwave Background Lensing Analysis},
2024, \href{https://ui.adsabs.harvard.edu/abs/2024ApJ...966..138M}{\apj, 966, 138}
\arxiv{2304.05196}

\item
C.~Hervías-Caimapo, and 23 colleagues,
\paper{The Atacama cosmology telescope: flux upper limits from a targeted search for extragalactic transients},
2024, \href{https://ui.adsabs.harvard.edu/abs/2024MNRAS.529.3020H}{\mnras, 529, 3020}
\arxiv{2301.07651}

\item
J.~Orlowski-Scherer, and 26 colleagues,
\paper{The Atacama Cosmology Telescope: Millimeter Observations of a Population of Asteroids or: ACTeroids},
2024, \href{https://ui.adsabs.harvard.edu/abs/2024ApJ...964..138O}{\apj, 964, 138}
\arxiv{2306.05468}

\item
K.~Małek, and 24 colleagues,
\paper{Attenuation proxy hidden in surface brightness - colour diagrams. A new strategy for the LSST era},
2024, \href{https://ui.adsabs.harvard.edu/abs/2024A&A...684A..30M}{\aap, 684, A30}
\arxiv{2401.12831}

\item
W.~R.~Coulton, and 153 colleagues
\paper{The Atacama Cosmology Telescope: High-resolution component-separated maps across one-third of the sky},
2023, \href{https://ui.adsabs.harvard.edu/abs/2023arXiv230701258C}{arXiv:2307.01258}
\submitted{\apj}

\item
C.~D.~Kreisch, and 23 colleagues
\paper{The Atacama Cosmology Telescope: The Persistence of Neutrino Self-Interaction in Cosmological Measurements},
2024, \href{https://ui.adsabs.harvard.edu/abs/2024PhRvD.109d3501K/abstract}{\prd, 109, 3501}
\arxiv{2207.03164}

\item
M.~S.~Madhavacheril, and 158 colleagues
\paper{The Atacama Cosmology Telescope: DR6 Gravitational Lensing Map and Cosmological Parameters},
2024, \href{https://ui.adsabs.harvard.edu/abs/2024ApJ...962..113M/abstract}{\apj, 962, 113}
\arxiv{2304.05203}

\item
F.~J.~Qu, and 157 colleagues
\paper{The Atacama Cosmology Telescope: A Measurement of the DR6 CMB Lensing Power Spectrum and its Implications for Structure Growth},
2024, \href{https://ui.adsabs.harvard.edu/abs/2024ApJ...962..112Q/abstract}{\apj, 962, 112}
\arxiv{2304.05202}
\item

S.~Shaikh, and 111 colleagues,
\paper{Cosmology from cross-correlation of ACT-DR4 CMB lensing and DES-Y3 cosmic shear},
2024, \href{https://ui.adsabs.harvard.edu/abs/2024MNRAS.528.2112S}{\mnras, 528, 2112}
\arxiv{2309.04412}

\item
G.~A.~Marques, and 94 colleagues
\paper{Cosmological constraints from the tomography of DES-Y3 galaxies with CMB lensing from ACT DR4},
2024, \href{https://ui.adsabs.harvard.edu/abs/2024JCAP...01..033M/abstract}{\jcap, 01, 033}
\arxiv{2306.17268}
   
\item
R.~Córdova Rosado, and 17 colleagues
\paper{The Atacama Cosmology Telescope: Galactic Dust Structure and the Cosmic PAH Background in Cross-correlation with WISE},
2023, \href{https://ui.adsabs.harvard.edu/abs/2023arXiv230706352C}{\apj, 960, 96}
\arxiv{2307.06352}

\item
D.~Anbajagane, and 113 colleagues
\paper{Cosmological shocks around galaxy clusters: A coherent investigation with DES, SPT \& ACT},
2024, \href{https://ui.adsabs.harvard.edu/abs/2023arXiv231000059A}{\mnras, 527, 9378}
\arxiv{2310.00059}
 
\item
Z.~Atkins, and 27 colleagues
\paper{The Atacama Cosmology Telescope: Map-Based Noise Simulations for DR6},
2023, \href{https://ui.adsabs.harvard.edu/abs/2023JCAP...11..073A/abstract}{\jcap, 11, 073}
\arxiv{2303.04180}

\item
T.~M.~C.~Abbott, and 159 colleagues
\paper{DES Y3 + KiDS-1000: Consistent cosmology combining cosmic shear surveys},
2023, \href{https://ui.adsabs.harvard.edu/abs/2023OJAp....6E..36A}{\oja, 6, 36}
\arxiv{2305.17173}

\item
Y.~Li, and 36 colleagues
\paper{The Atacama Cosmology Telescope: Systematic Transient Search of 3 Day Maps},
2023, \href{https://ui.adsabs.harvard.edu/abs/2023ApJ...956...36L}{\apj, 956, 36}
\arxiv{2303.04767}

\item
M.~Mallaby-Kay, and 82 colleagues
\paper{Kinematic Sunyaev-Zel'dovich effect with ACT, DES, and BOSS: A novel hybrid estimator},
2023, \href{https://ui.adsabs.harvard.edu/abs/2023PhRvD.108b3516M}{\prd, 108, 023516}
\arxiv{2305.06792}

\item
B.~L.~Frye, and 43 colleagues
\paper{The JWST PEARLS View of the El Gordo Galaxy Cluster and of the Structure It Magnifies},
2023, \href{https://ui.adsabs.harvard.edu/abs/2023ApJ...952...81F}{\apj, 952, 81}
\arxiv{2303.03556}

\item
J.~B.~Golden-Marx, and 68 colleagues
\paper{Characterizing the intracluster light over the redshift range $0.2 < z < 0.8$ in the DES-ACT overlap},
2023, \href{https://ui.adsabs.harvard.edu/abs/2023MNRAS.521..478G}{\mnras, 521, 478}
\arxiv{2209.05519}

\item
T.~Kitayama, and 17 colleagues
\paper{Galaxy clusters at $z\sim1$ imaged by ALMA with the Sunyaev-Zel'dovich effect},
2023, \href{https://ui.adsabs.harvard.edu/abs/2023PASJ...75..311K}{\pasj, 75, 311}
\arxiv{2209.09503}

\item
Z.~Li, and 22 colleagues
\paper{The Atacama Cosmology Telescope: limits on dark matter-baryon interactions from DR4 power spectra},
2023, \href{https://ui.adsabs.harvard.edu/abs/2023JCAP...02..046L}{\jcap, 2023, 046}
\arxiv{2208.08985}

\item
O.~Contigiani, H.~Hoekstra, M.~M.~Brouwer, A.~Dvornik, M.~C.~Fortuna, \myself, Z.~Yan, and M.~Vakili,
\paper{Dynamical cluster masses from photometric surveys},
2023, \href{https://ui.adsabs.harvard.edu/abs/2023MNRAS.518.2640C}{\mnras, 518, 2640}
\arxiv{2208.09369}


\item
F.~Radiconi, and 30 colleagues
\paper{The thermal and non-thermal components within and between galaxy clusters Abell 399 and Abell 401},
2022, \href{https://ui.adsabs.harvard.edu/abs/2022MNRAS.517.5232R}{\mnras, 517, 5232}
\arxiv{2206.04697}

\item
S.~S.~Sheppard, and 19 colleagues
\paper{A Deep and Wide Twilight Survey for Asteroids Interior to Earth and Venus},
2022, \href{https://ui.adsabs.harvard.edu/abs/2022AJ....164..168S}{\aj, 164, 168}
\arxiv{2209.06245}

\item
J.~E.~Greene, J.~P.~Greco, A.~D.~Goulding, S.~Huang, E.~Kado-Fong, S.~Danieli, J.~Li, J.~H.~Kim, Y.~Komiyama, A.~Leauthaud, L.~A.~MacArthur, and \myself,
\paper{The Nature of Low-surface-brightness Galaxies in the Hyper Suprime-Cam Survey},
2022, \href{https://ui.adsabs.harvard.edu/abs/2022ApJ...933..150G}{\apj, 933, 150}
\arxiv{2204.11883}

\item
M.~Lokken, and 106 colleagues
\paper{Superclustering with the Atacama Cosmology Telescope and Dark Energy Survey. I.~Evidence for Thermal Energy Anisotropy Using Oriented Stacking},
2022, \href{https://ui.adsabs.harvard.edu/abs/2022ApJ...933..134L}{\apj, 933, 134}
\arxiv{2107.05523}

\item
J.~C.~Hill, and 42 colleagues
\paper{Atacama Cosmology Telescope: Constraints on prerecombination early dark energy},
2022, \href{https://ui.adsabs.harvard.edu/abs/2022PhRvD.105l3536H}{\prd, 105, 123536}
\arxiv{2109.04451}

\item
S.~Pandey, and 126 colleagues
\paper{Cross-correlation of Dark Energy Survey Year 3 lensing data with ACT and P l a n c k thermal Sunyaev-Zel'dovich effect observations. II.~Modeling and constraints on halo pressure profiles},
2022, \href{https://ui.adsabs.harvard.edu/abs/2022PhRvD.105l3526P}{\prd, 105, 123526}
\arxiv{2108.01601}

\item
M.~Gatti, and 130 colleagues
\paper{Cross-correlation of Dark Energy Survey Year 3 lensing data with ACT and Planck thermal Sunyaev-Zel'dovich effect observations. I.~Measurements, systematics tests, and feedback model constraints},
2022, \href{https://ui.adsabs.harvard.edu/abs/2022PhRvD.105l3525G}{\prd, 105, 123525}
\arxiv{2108.01600}

\item
M.~Lungu, and 30 colleagues
\paper{The Atacama Cosmology Telescope: measurement and analysis of 1D beams for DR4},
2022, \href{https://ui.adsabs.harvard.edu/abs/2022JCAP...05..044L}{\jcap, 2022, 044}
\arxiv{2112.12226}

\item
A.~Leauthaud, and 106 colleagues
\paper{Lensing without borders - I.~A blind comparison of the amplitude of galaxy-galaxy lensing between independent imaging surveys},
2022, \href{https://ui.adsabs.harvard.edu/abs/2022MNRAS.510.6150L}{\mnras, 510, 6150}
\arxiv{2111.13805}

\item
A.~D.~Hincks, and 45 colleagues
\paper{A high-resolution view of the filament of gas between Abell 399 and Abell 401 from the Atacama Cosmology Telescope and MUSTANG-2},
2022, \href{https://ui.adsabs.harvard.edu/abs/2022MNRAS.510.3335H}{\mnras, 510, 3335}
\arxiv{2107.04611}

\item
J.~H.~O'Donnell, and 81 colleagues
\paper{The Dark Energy Survey Bright Arcs Survey: Candidate Strongly Lensed Galaxy Systems from the Dark Energy Survey 5000 Square Degree Footprint},
2022, \href{https://ui.adsabs.harvard.edu/abs/2022ApJS..259...27O}{\apjs, 259, 27}
\arxiv{2110.02418}

\item
M.~Aguena, and 24 colleagues
\paper{CLMM: a LSST-DESC cluster weak lensing mass modeling library for cosmology},
2021, \href{https://ui.adsabs.harvard.edu/abs/2021MNRAS.508.6092A}{\mnras, 508, 6092}
\arxiv{2107.10857}

\item
S.~R.~Dicker, and 28 colleagues
\paper{Observations of compact sources in galaxy clusters using MUSTANG2},
2021, \href{https://ui.adsabs.harvard.edu/abs/2021MNRAS.508.2600D}{\mnras, 508, 2600}
\arxiv{2107.06725}

\item
S.~Naess, and 39 colleagues
\paper{The Atacama Cosmology Telescope: A Search for Planet 9},
2021, \href{https://ui.adsabs.harvard.edu/abs/2021ApJ...923..224N}{\apj, 923, 224}
\arxiv{2104.10264}

\item
J.~Kim, M.~J.~Jee, J.~P.~Hughes, M.~Yoon, K.~HyeongHan, F.~Menanteau, \myself, L.~Hovey, and P.~Arunachalam
\paper{Head-to-Toe Measurement of El Gordo: Improved Analysis of the Galaxy Cluster ACT-CL J0102$-$4915 with New Wide-field Hubble Space Telescope Imaging Data},
2021, \href{https://ui.adsabs.harvard.edu/abs/2021ApJ...923..101K}{\apj, 923, 101}
\arxiv{2106.00031}

\item
S.~Adhikari, and 115 colleagues
\paper{Probing Galaxy Evolution in Massive Clusters Using ACT and DES: Splashback as a Cosmic Clock},
2021, \href{https://ui.adsabs.harvard.edu/abs/2021ApJ...923...37A}{\apj, 923, 37}
\arxiv{2008.11663}

\item
Y.~Li, and 32 colleagues
\paper{Constraining Cosmic Microwave Background Temperature Evolution With Sunyaev-Zel'Dovich Galaxy Clusters from the Atacama Cosmology Telescope},
2021, \href{https://ui.adsabs.harvard.edu/abs/2021ApJ...922..136L}{\apj, 922, 136}
\arxiv{2106.12467}

\item
T.~Shin, and 138 colleagues
\paper{The mass and galaxy distribution around SZ-selected clusters},
2021, \href{https://ui.adsabs.harvard.edu/abs/2021MNRAS.507.5758S}{\mnras, 507, 5758}
\arxiv{2105.05914}

\item
Y.~Guan, and 32 colleagues
\paper{The Atacama Cosmology Telescope: Microwave Intensity and Polarization Maps of the Galactic Center},
2021, \href{https://ui.adsabs.harvard.edu/abs/2021ApJ...920....6G}{\apj, 920, 6}
\arxiv{2105.05267}

\item
J.~Orlowski-Scherer, and 37 colleagues
\paper{Atacama Cosmology Telescope measurements of a large sample of candidates from the Massive and Distant Clusters of WISE Survey. Sunyaev-Zeldovich effect confirmation of MaDCoWS candidates using ACT},
2021, \href{https://ui.adsabs.harvard.edu/abs/2021A&A...653A.135O}{\aap, 653, A135}
\arxiv{2105.00068}

\item
E.~M.~Vavagiakis, and 53 colleagues
\paper{The Atacama Cosmology Telescope: Probing the baryon content of SDSS DR15 galaxies with the thermal and kinematic Sunyaev-Zel'dovich effects},
2021, \href{https://ui.adsabs.harvard.edu/abs/2021PhRvD.104d3503V}{\prd, 104, 043503}
\arxiv{2101.08373}

\item
V.~Calafut, and 53 colleagues
\paper{The Atacama Cosmology Telescope: Detection of the pairwise kinematic Sunyaev-Zel'dovich effect with SDSS DR15 galaxies},
2021, \href{https://ui.adsabs.harvard.edu/abs/2021PhRvD.104d3502C}{\prd, 104, 043502}
\arxiv{2101.08374}

\item
M.~Mallaby-Kay, and 59 colleagues
\paper{The Atacama Cosmology Telescope: Summary of DR4 and DR5 Data Products and Data Access},
2021, \href{https://ui.adsabs.harvard.edu/abs/2021ApJS..255...11M}{\apjs, 255, 11}
\arxiv{2103.03154}

\item
K.~Knowles, and 28 colleagues
\paper{MERGHERS pilot: MeerKAT discovery of diffuse emission in nine massive Sunyaev-Zel'dovich-selected galaxy clusters from ACT},
2021, \href{https://ui.adsabs.harvard.edu/abs/2021MNRAS.504.1749K}{\mnras, 504, 1749}
\arxiv{2012.15088}

\item
N.~C.~Robertson, and 50 colleagues
\paper{Strong detection of the CMB lensing and galaxy weak lensing cross-correlation from ACT-DR4, Planck Legacy, and KiDS-1000},
2021, \href{https://ui.adsabs.harvard.edu/abs/2021A&A...649A.146R}{\aap, 649, A146}
\arxiv{2011.11613}

\item
B.~J.~Fuzia, and 21 colleagues
\paper{The Atacama Cosmology Telescope: SZ-based masses and dust emission from IR-selected cluster candidates in the SHELA survey},
2021, \href{https://ui.adsabs.harvard.edu/abs/2021MNRAS.502.4026F}{\mnras, 502, 4026}
\arxiv{2001.09587}

\item
S.~Amodeo, and 54 colleagues
\paper{Atacama Cosmology Telescope: Modeling the gas thermodynamics in BOSS CMASS galaxies from kinematic and thermal Sunyaev-Zel'dovich measurements},
2021, \href{https://ui.adsabs.harvard.edu/abs/2021PhRvD.103f3514A}{\prd, 103, 063514}
\arxiv{2009.05558}

\item
E.~Schaan, and 58 colleagues
\paper{Atacama Cosmology Telescope: Combined kinematic and thermal Sunyaev-Zel'dovich measurements from BOSS CMASS and LOWZ halos},
2021, \href{https://ui.adsabs.harvard.edu/abs/2021PhRvD.103f3513S}{\prd, 103, 063513}
\arxiv{2009.05557}

\item
O.~Darwish, and 54 colleagues
\paper{The Atacama Cosmology Telescope: a CMB lensing mass map over 2100 square degrees of sky and its cross-correlation with BOSS-CMASS galaxies},
2021, \href{https://ui.adsabs.harvard.edu/abs/2021MNRAS.500.2250D}{\mnras, 500, 2250}
\arxiv{2004.01139}

\item
E.~N.~Taylor, and 18 colleagues
\paper{GAMA + KiDS: empirical correlations between halo mass and other galaxy properties near the knee of the stellar-to-halo mass relation},
2020, \href{https://ui.adsabs.harvard.edu/abs/2020MNRAS.499.2896T}{\mnras, 499, 2896}
\arxiv{2006.10040}

\item
S.~Aiola, and 140 colleagues
\paper{The Atacama Cosmology Telescope: DR4 maps and cosmological parameters},
2020, \href{https://ui.adsabs.harvard.edu/abs/2020JCAP...12..047A}{\jcap, 2020, 047}
\arxiv{2007.07288}

\item
S.~Naess, and 61 colleagues
\paper{The Atacama Cosmology Telescope: arcminute-resolution maps of 18 000 square degrees of the microwave sky from ACT 2008-2018 data combined with Planck},
2020, \href{https://ui.adsabs.harvard.edu/abs/2020JCAP...12..046N}{\jcap, 2020, 046}
\arxiv{2007.07290}

\item
S.~K.~Choi, and 138 colleagues
\paper{The Atacama Cosmology Telescope: a measurement of the Cosmic Microwave Background power spectra at 98 and 150 GHz},
2020, \href{https://ui.adsabs.harvard.edu/abs/2020JCAP...12..045C}{\jcap, 2020, 045}
\arxiv{2007.07289}

\item
Z.~Li, and 31 colleagues
\paper{The cross correlation of the ABS and ACT maps},
2020, \href{https://ui.adsabs.harvard.edu/abs/2020JCAP...09..010L}{\jcap, 2020, 010}
\arxiv{2002.05717}

\item
Y.~Rong, and 13 colleagues
\paper{Intrinsic Morphology of Ultra-diffuse Galaxies},
2020, \href{https://ui.adsabs.harvard.edu/abs/2020ApJ...899...78R}{\apj, 899, 78}
\arxiv{1907.10079}

\item
L.~Linke, P.~Simon, P.~Schneider, T.~Erben, D.~J.~Farrow, C.~Heymans, H.~Hildebrandt, A.~M.~Hopkins, A.~Kannawadi, N.~R.~Napolitano, \myself, and A.~H.~Wright
\paper{KiDS+VIKING+GAMA: Testing semi-analytic models of galaxy evolution with galaxy-galaxy-galaxy lensing},
2020, \href{https://ui.adsabs.harvard.edu/abs/2020A&A...640A..59L}{\aap, 640, A59}
\arxiv{2005.02419}

\item
M.~S.~Madhavacheril, and 55 colleagues
\paper{Atacama Cosmology Telescope: Component-separated maps of CMB temperature and the thermal Sunyaev-Zel'dovich effect},
2020, \href{https://ui.adsabs.harvard.edu/abs/2020PhRvD.102b3534M}{\prd, 102, 023534}
\arxiv{1911.05717}

\item
T.~Namikawa, and 53 colleagues
\paper{Atacama Cosmology Telescope: Constraints on cosmic birefringence},
2020, \href{https://ui.adsabs.harvard.edu/abs/2020PhRvD.101h3527N}{\prd, 101, 083527}
\arxiv{2001.10465}

\item
S.~Huang, A.~Leauthaud, A.~Hearin, P.~Behroozi, C.~Bradshaw, F.~Ardila, J.~Speagle, A.~Tenneti, K.~Bundy, J.~Greene, \myself, and N.~Bahcall,
\paper{Weak lensing reveals a tight connection between dark matter halo mass and the distribution of stellar mass in massive galaxies},
2020, \href{https://ui.adsabs.harvard.edu/abs/2020MNRAS.492.3685H}{\mnras, 492, 3685}
\arxiv{1811.01139}

\item
Q.~Xia, and 14 colleagues
\paper{A gravitational lensing detection of filamentary structures connecting luminous red galaxies},
2020, \href{https://ui.adsabs.harvard.edu/abs/2020A&A...633A..89X}{\aap, 633, A89}
\arxiv{1909.05852}

\item
H.~Hildebrandt, and 27 colleagues
\paper{KiDS+VIKING-450: Cosmic shear tomography with optical and infrared data},
2020, \href{https://ui.adsabs.harvard.edu/abs/2020A&A...633A..69H}{\aap, 633, A69}
\arxiv{1812.06076}

\item
J.~S.~Speagle, A.~Leauthaud, S.~Huang, C.~P.~Bradshaw, F.~Ardila, P.~L.~Capak, D.~J.~Eisenstein, D.~C.~Masters, R.~Mandelbaum, S.~More, M.~Simet, and \myself,
\paper{Galaxy-Galaxy lensing in HSC: Validation tests and the impact of heterogeneous spectroscopic training sets},
2019, \href{https://ui.adsabs.harvard.edu/abs/2019MNRAS.490.5658S}{\mnras, 490, 5658}
\arxiv{1906.05876}

\item
K.~R.~Hall, and 25 colleagues
\paper{Quantifying the thermal Sunyaev-Zel'dovich effect and excess millimetre emission in quasar environments},
2019, \href{https://ui.adsabs.harvard.edu/abs/2019MNRAS.490.2315H}{\mnras, 490, 2315}
\arxiv{1907.11731}

\item
A.~H.~Wright, and 24 colleagues
\paper{KiDS+VIKING-450: A new combined optical and near-infrared dataset for cosmology and astrophysics},
2019, \href{https://ui.adsabs.harvard.edu/abs/2019A&A...632A..34W}{\aap, 632, A34}
\arxiv{1812.06077}

\item
K.~Knowles, and 13 colleagues
\paper{GMRT 610 MHz observations of galaxy clusters in the ACT equatorial sample},
2019, \href{https://ui.adsabs.harvard.edu/abs/2019MNRAS.486.1332K}{\mnras, 486, 1332}
\arxiv{1806.09579}

\item
C.~Hikage, and 36 colleagues
\paper{Cosmology from cosmic shear power spectra with Subaru Hyper Suprime-Cam first-year data},
2019, \href{https://ui.adsabs.harvard.edu/abs/2019PASJ...71...43H}{\pasj, 71, 43}
\arxiv{1809.09148}

\item
H.~Miyatake, and 59 colleagues
\paper{Weak-lensing Mass Calibration of ACTPol Sunyaev-Zel’dovich Clusters with the Hyper Suprime-Cam Survey},
2019, \href{https://ui.adsabs.harvard.edu/abs/2019ApJ...875...63M}{\apj, 875, 63}
\arxiv{1804.05873}

\item
M.~M.~Brouwer, and 17 colleagues
\paper{Studying galaxy troughs and ridges using weak gravitational lensing with the Kilo-Degree Survey},
2018, \href{https://ui.adsabs.harvard.edu/abs/2018MNRAS.481.5189B}{\mnras, 481, 5189}
\arxiv{1805.00562}

\item
R.~Wojtak, and 19 colleagues
\paper{Galaxy Cluster Mass Reconstruction Project - IV.~Understanding the effects of imperfect membership on cluster mass estimation},
2018, \href{https://ui.adsabs.harvard.edu/abs/2018MNRAS.481..324W}{\mnras, 481, 324}
\arxiv{1806.03199}

\item
A.~Jakobs, and 20 colleagues
\paper{Multiwavelength scaling relations in galaxy groups: a detailed comparison of GAMA and KiDS observations to BAHAMAS simulations},
2018, \href{https://ui.adsabs.harvard.edu/abs/2018MNRAS.480.3338J}{\mnras, 480, 3338}
\arxiv{1712.05463}

\item
A.~Dvornik, and 13 colleagues
\paper{Unveiling galaxy bias via the halo model, KiDS, and GAMA},
2018, \href{https://ui.adsabs.harvard.edu/abs/2018MNRAS.479.1240D}{\mnras, 479, 1240}
\arxiv{1802.00734}

\item
J.~P.~Greco, and 13 colleagues
\paper{Illuminating Low Surface Brightness Galaxies with the Hyper Suprime-Cam Survey},
2018, \href{https://ui.adsabs.harvard.edu/abs/2018ApJ...857..104G}{\apj, 857, 104}
\arxiv{1709.04474}

\item
E.~Medezinski, and 15 colleagues
\paper{Source selection for cluster weak lensing measurements in the Hyper Suprime-Cam survey},
2018, \href{https://ui.adsabs.harvard.edu/abs/2018PASJ...70...30M}{\pasj, 70, 30}
\arxiv{1706.00427}

\item
L.~Old, and 17 colleagues
\paper{Galaxy Cluster Mass Reconstruction Project - III.~The impact of dynamical substructure on cluster mass estimates},
2018, \href{https://ui.adsabs.harvard.edu/abs/2018MNRAS.475..853O}{\mnras, 475, 853}
\arxiv{1709.10108}

\item
J.~F.~Wu, P.~Aguirre, A.~J.~Baker, M.~J.~Devlin, M.~Hilton, J.~P.~Hughes, L.~Infante, R.~R.~Lindner, and \myself,
\paper{Herschel and ALMA Observations of Massive SZE-selected Clusters},
2018, \href{https://ui.adsabs.harvard.edu/abs/2018ApJ...853..195W}{\apj, 853, 195}
\arxiv{1712.04540}

\item
E.~Medezinski, N.~Battaglia, K.~Umetsu, M.~Oguri, H.~Miyatake, A.~J.~Nishizawa, \myself, D.~N.~Spergel, I.-N.~Chiu, Y.-T.~Lin, N.~Bahcall, and Y.~Komiyama
\paper{Planck Sunyaev-Zel'dovich cluster mass calibration using Hyper Suprime-Cam weak lensing},
2018, \href{https://ui.adsabs.harvard.edu/abs/2018PASJ...70S..28M}{\pasj, 70, S28}
\arxiv{1706.00434}

\item
R.~Mandelbaum, and 30 colleagues
\paper{The first-year shear catalog of the Subaru Hyper Suprime-Cam Subaru Strategic Program Survey},
2018, \href{https://ui.adsabs.harvard.edu/abs/2018PASJ...70S..25M}{\pasj, 70, S25}
\arxiv{1705.06745}

\item
M.~Velliscig, and 16 colleagues
\paper{Galaxy-galaxy lensing in EAGLE: comparison with data from 180 deg$^2$ of the KiDS and GAMA surveys},
2017, \href{https://ui.adsabs.harvard.edu/abs/2017MNRAS.471.2856V}{\mnras, 471, 2856}
\arxiv{1612.04825}

\item
A.~Dvornik, and 21 colleagues
\paper{A KiDS weak lensing analysis of assembly bias in GAMA galaxy groups},
2017, \href{https://ui.adsabs.harvard.edu/abs/2017MNRAS.468.3251D}{\mnras, 468, 3251}
\arxiv{1703.06657}

\item
M.~M.~Brouwer, and 21 colleagues
\paper{First test of Verlinde's theory of emergent gravity using weak gravitational lensing measurements},
2017, \href{https://ui.adsabs.harvard.edu/abs/2017MNRAS.466.2547B}{\mnras, 466, 2547}
\arxiv{1612.03034}

\item
S.~Bellstedt, and 16 colleagues
\paper{The evolution in the stellar mass of brightest cluster galaxies over the past 10 billion years},
2016, \href{https://ui.adsabs.harvard.edu/abs/2016MNRAS.460.2862B}{\mnras, 460, 2862}
\arxiv{1605.02736}

\item
N.~Battaglia, and 41 colleagues
\paper{Weak-lensing mass calibration of the Atacama Cosmology Telescope equatorial Sunyaev-Zeldovich cluster sample with the Canada-France-Hawaii telescope stripe 82 survey},
2016, \href{https://ui.adsabs.harvard.edu/abs/2016JCAP...08..013B}{\jcap, 2016, 013}
\arxiv{1509.08930}

\item
K.~Knowles, and 21 colleagues
\paper{A giant radio halo in a low-mass SZ-selected galaxy cluster: ACT-CL J0256.5+0006},
2016, \href{https://ui.adsabs.harvard.edu/abs/2016MNRAS.459.4240K}{\mnras, 459, 4240}
\arxiv{1506.01547}

\item
D.~Crichton, and 22 colleagues
\paper{Evidence for the thermal Sunyaev-Zel'dovich effect associated with quasar feedback},
2016, \href{https://ui.adsabs.harvard.edu/abs/2016MNRAS.458.1478C}{\mnras, 458, 1478}
\arxiv{1510.05656}

\item
K.~Kuijken, and 34 colleagues
\paper{Gravitational lensing analysis of the Kilo-Degree Survey},
2015, \href{https://ui.adsabs.harvard.edu/abs/2015MNRAS.454.3500K}{\mnras, 454, 3500}
\arxiv{1507.00738}

\item
K.~Y.~Ng, W.~A.~Dawson, D.~Wittman, M.~J.~Jee, J.~P.~Hughes, F.~Menanteau, and \myself,
\paper{The return of the merging galaxy subclusters of El Gordo?},
2015, \href{https://ui.adsabs.harvard.edu/abs/2015MNRAS.453.1531N}{\mnras, 453, 1531}
\arxiv{1412.1826}

\item
M.~Viola, and 26 colleagues
\paper{Dark matter halo properties of GAMA galaxy groups from 100 square degrees of KiDS weak lensing data},
2015, \href{https://ui.adsabs.harvard.edu/abs/2015MNRAS.452.3529V}{\mnras, 452, 3529}
\arxiv{1507.00735}

\item
J.~T.~A.~de Jong, and 48 colleagues
\paper{The first and second data releases of the Kilo-Degree Survey},
2015, \href{https://ui.adsabs.harvard.edu/abs/2015A&A...582A..62D}{\aap, 582, A62}
\arxiv{1507.00742}

\item
B.~Kirk, and 22 colleagues
\paper{SALT spectroscopic observations of galaxy clusters detected by ACT and a type II quasar hosted by a brightest cluster galaxy},
2015, \href{https://ui.adsabs.harvard.edu/abs/2015MNRAS.449.4010K}{\mnras, 449, 4010}
\arxiv{1410.7887}

\item
L.~Old, and 23 colleagues
\paper{Galaxy Cluster Mass Reconstruction Project - II.~Quantifying scatter and bias using contrasting mock catalogues},
2015, \href{https://ui.adsabs.harvard.edu/abs/2015MNRAS.449.1897O}{\mnras, 449, 1897}
\arxiv{1502.07347}

\item
R.~R.~Lindner, and 26 colleagues
\paper{The Atacama Cosmology Telescope: The LABOCA/ACT Survey of Clusters at All Redshifts},
2015, \href{https://ui.adsabs.harvard.edu/abs/2015ApJ...803...79L}{\apj, 803, 79}
\arxiv{1411.7998}

\item
M.~B.~Gralla, and 40 colleagues
\paper{A measurement of the millimetre emission and the Sunyaev-Zel'dovich effect associated with low-frequency radio sources},
2014, \href{https://ui.adsabs.harvard.edu/abs/2014MNRAS.445..460G}{\mnras, 445, 460}
\arxiv{1310.8281}

\item
L.~Old, and 20 colleagues
\paper{Galaxy cluster mass reconstruction project - I.~Methods and first results on galaxy-based techniques},
2014, \href{https://ui.adsabs.harvard.edu/abs/2014MNRAS.441.1513O}{\mnras, 441, 1513}
\arxiv{1403.4610}

\item
M.~J.~Jee, J.~P.~Hughes, F.~Menanteau, \myself, R.~Mandelbaum, L.~F.~Barrientos, L.~Infante, and K.~Y.~Ng
\paper{Weighing "El Gordo" with a Precision Scale: Hubble Space Telescope Weak-lensing Analysis of the Merging Galaxy Cluster ACT-CL J0102-4915 at $z = 0.87$},
2014, \href{https://ui.adsabs.harvard.edu/abs/2014ApJ...785...20J}{\apj, 785, 20}
\arxiv{1309.5097}

\item
M.~Hasselfield, and 43 colleagues
\paper{The Atacama Cosmology Telescope: Sunyaev-Zel'dovich selected galaxy clusters at 148 GHz from three seasons of data},
2013, \href{https://ui.adsabs.harvard.edu/abs/2013JCAP...07..008H}{\jcap, 2013, 008}
\arxiv{1301.0816}

\item
E.~Calabrese, and 33 colleagues
\paper{Cosmological parameters from pre-planck cosmic microwave background measurements},
2013, \href{https://ui.adsabs.harvard.edu/abs/2013PhRvD..87j3012C}{\prd, 87, 103012}
\arxiv{1302.1841}

\item
N.~Sehgal, and 35 colleagues
\paper{The Atacama Cosmology Telescope: Relation between Galaxy Cluster Optical Richness and Sunyaev-Zel'dovich Effect},
2013, \href{https://ui.adsabs.harvard.edu/abs/2013ApJ...767...38S}{\apj, 767, 38}
\arxiv{1205.2369}

\item
H.~Miyatake, and 28 colleagues
\paper{Subaru weak lensing measurement of a z = 0.81 cluster discovered by the Atacama Cosmology Telescope Survey},
2013, \href{https://ui.adsabs.harvard.edu/abs/2013MNRAS.429.3627M}{\mnras, 429, 3627}
\arxiv{1209.4643}

\item
B.~D.~Sherwin, and 30 colleagues
\paper{The Atacama Cosmology Telescope: Cross-correlation of cosmic microwave background lensing and quasars},
2012, \href{https://ui.adsabs.harvard.edu/abs/2012PhRvD..86h3006S}{\prd, 86, 083006}
\arxiv{1207.4543}

\item
N.~Hand, and 57 colleagues
\paper{Evidence of Galaxy Cluster Motions with the Kinematic Sunyaev-Zel'dovich Effect},
2012, \href{https://ui.adsabs.harvard.edu/abs/2012PhRvL.109d1101H}{\prl, 109, 041101}
\arxiv{1203.4219}

\item
E.~D.~Reese, and 43 colleagues
\paper{The Atacama Cosmology Telescope: High-resolution Sunyaev-Zel'dovich Array Observations of ACT SZE-selected Clusters from the Equatorial Strip},
2012, \href{https://ui.adsabs.harvard.edu/abs/2012ApJ...751...12R}{\apj, 751, 12}
\arxiv{1108.3343}

\end{etaremune}

\end{etaremune}



\end{document}




\end{document}

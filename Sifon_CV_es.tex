\documentclass[11pt]{article}


\usepackage{bm}
\usepackage{enumerate}
\usepackage{etaremune}
% \usepackage{fullpage}
\usepackage{graphicx}
\usepackage{hyperref}

\addtolength{\oddsidemargin}{-.875in}
\addtolength{\evensidemargin}{-.875in}
\addtolength{\textwidth}{1.75in}
\addtolength{\topmargin}{-1in}
\addtolength{\textheight}{2in}

% For links of references
\hypersetup{colorlinks,
  linkcolor=blue,
  filecolor=blue,
  urlcolor=blue,
  citecolor=blue}

\renewcommand{\familydefault}{\sfdefault}

\newcommand\sectitle[1]{
  \hline
  \vspace{0.5cm}
  \noindent
  %\underline{
    \textbf{\Large #1}
  %}
  \\
  \vspace{-0.2cm}
}

\newcommand\subsectitle[1]{
  \vspace{0.3cm}
  \noindent
  %\underline{
    \textbf{\large #1}
  %}
  \\
  \vspace{-0.3cm}
}

\newcommand\technical[2]{
  \noindent
    {\large\bf #1:} #2\\
  }

\newcommand\itemdate[1]{\textbf{[#1]}}
\newcommand\itemdates[2]{\textbf{[#1 -- #2]}}
\newcommand\email[1]{\href{mailto:#1}{\texttt{#1}}}



%not used if publication list not shown

\def\aap{A\&A}
\def\apj{ApJ}
\def\apjl{ApJL}
\def\apjs{ApJS}
\def\baas{\textit{Bull.\ of the Am.\ Ast.\ Soc.}}
\def\gemfoc{Gemini Focus}
\def\jcap{JCAP}
\def\mnras{MNRAS}
\def\msngr{The Messenger}
\def\pasj{PASJ}
\def\prd{Phys.\ Rev.\ D}
\def\prl{Phys.\ Rev.\ Letters}
\def\ssr{Space Sci.\ Rev.}

\newcommand{\myself}{\textbf{\color{red} C.~Sif\'on}}
\newcommand\includemyself{\textbf\small{(including C.~Sif\'on)}}
\newcommand{\accepted}[1]{accepted for publication in #1}
\newcommand{\etal}[1]{et al.\ (#1 co-authors),}
\newcommand{\etalwithme}[1]{et al.\ (#1 coauthors incl.\ \myself),}
\newcommand{\paper}[1]{\textbf{``#1''}}
\newcommand{\submitted}[1]{submitted to #1}
\renewcommand{\title}[1]{\noindent\textbf{\huge #1}\\}
% arXiv links
\newcommand{\arxiv}[1]
    {\href{https://arxiv.org/abs/#1}{\texttt{\color{magenta}[arXiv]}}}




\begin{document}

% \begin{figure}[t]
\begin{minipage}[b]{0.46\linewidth}
\flushleft
% \noindent
\hspace{-0.7cm}
{\bf\huge Crist\'obal Sif\'on}\\\vspace{0.2cm}
\hspace{-0.5cm}{\large Profesor Asociado\\
\hspace{-0.65cm}        Instituto de F\'isica, Facultad de Ciencias\\
\hspace{-0.65cm}        \pucv\\
\hspace{-0.65cm}        Casilla 4059, Valpara\'iso, Chile}\\
\end{minipage}
\begin{minipage}[b]{0.49\linewidth}
\flushright
{\large E-mail: {\texttt cristobal.sifon@pucv.cl}\\
        Teléfono: +56 (32) 227 4698\\
        \url{http://fis.ucv.cl/csifon/}
        \url{https://github.com/cristobal-sifon/}}
\end{minipage}
\vspace{0.4cm}
\hline



\sectitle{Investigación}

Mi investigación se enfoca en la física de cúmulos de galaxias, incluyendo la 
relación entre observables y masa para análisis cosmológicos, evolución de 
galaxias en cúmulos y cúmulos en proceso de colisión, además de los 
alineamientos intrínsecos entre galaxias. Uso un número de herramientas y 
técnicas para estudiar estos fenómenos, incluyendo mediciones del efecto de 
lente gravitacional débil, fotometría y espectroscopía óptica e infrarroja, 
datos entregados por mapeos ópticos, observaciones del fondo cósmico de 
microondas y más recientemente simulaciones computacionales.

\vspace{0.5cm}
\technical{Colaboraciones Científicas}
{
 4MOST Hemisphere Survey (4HS) ---
 Atacama Cosmology Telescope (ACT) ---
 Canadian Cluster Comparison Project (CCCP) ---
 Chilean Cluster galaxy Evolution Survey (CHANCES) ---
 Galaxy Cluster Mass Reconstruction Project ---
 Kilo-Degree Survey (KiDS) ---
 Legacy Survey of Space and Time Dark Energy Science Collaboration (LSST-DESC) ---
 Multi-Epoch Nearby Cluster Survey (MENeaCS) ---
 Simons Observatory.
}


\sectitle{Empleos}

\noindent
\itemdates{2019}{Presente} Profesor Asociado, \pucv\ (PUCV), Chile\\
\itemdates{2016}{2019} Investigador Postdoctoral Asociado, Princeton University, EE.UU.


\subsectitle{Educación}

\noindent
\itemdates{2012}{2016} Ph.D.~Astrofísica, Universiteit Leiden, Países Bajos\\
\itemdates{2010}{2012} M.Sc.~Astrofísica, P.~Universidad Cat\'olica de Chile (PUC), Chile\\
\itemdates{2005}{2010} B.Sc.~Astronomía, P.~Universidad Cat\'olica de Chile, Chile\\


\sectitle{Docencia y Supervisión}

\subsectitle{Supervisión de Investigación de Estudiantes}

\itemdates{2020}{Presente} Camila Aros, PUCV: tesis de Magíster.\\
\itemdate{2021} Lya Marmolejo, PUCV (actualmente estudiante de Magíster U.\ La Serena): Proyecto de investigación de 
verano y tesina de Licenciatura.\\
\itemdate{2020} Nicole Mej\'ia, Universidad Nacional Aut\'onoma de Honduras 
(Honduras): Proyecto de investigación de pregrado a través del Central American-Caribbean Bridge in Astrophysics Program
    (\href{https://cencabridgeastro.weebly.com/our-team.html}{URL}; duración: 4 meses).\\
\itemdate{2020} Felipe Jorquera, PUCV: Proyecto de investigación de verano.\\
\itemdates{2017}{2019} Naomi Robertson, Universidad de Oxford (Reino Unido, actualmente investigadora
postdoctoral en Cambridge U., Reino Unido): Co-supervisor de proyecto de tesis de doctorado.\\
\itemdate{2018} Malik Walker, Universidad de Princeton (actualmente estudiante de doctorado en
Johns Hopkins U., EE.UU.): Proyecto de investigación de verano y proyecto de investigación semestral de pregrado.\\
\itemdates{2013}{2014} Joshua Albert, Universidad de Leiden (actualmente investigador asociado, U.\ Leiden): Co-supervisor de 
proyecto de tesis de Magíster.

%\pagebreak

\subsectitle{Cursos dictados}

\noindent
\emph{Licenciatura en Física y Licenciatura en Física, mención Astronomía}

\itemdate{2022} Astronomía Básica (curso de servicio para Ingeniería Civil Industrial)\\
\itemdate{2022,2021} Astronomía Galáctica\\
\itemdate{2021,2020} Programación\\

\noindent
\emph{Magíster y Doctorado en Física}

\itemdate{2022} Técnicas observacionales en Astrofísica (planeado 2do semestre)\\
\itemdate{2021,2020} Cosmología Observacional

\subsectitle{Comités de tesis}

\itemdate{2021} Daniela Grandón, Proyecto de tesis Doctorado en Física, Universidad de Chile\\
\itemdate{2021} Camila Varas, Licenciatura en Astronomía, PUC\\


\sectitle{Financiamiento}

\noindent
\itemdate{2020} Fondo ALMA-ANID para contratar un investigador postdoctoral (\textbf{Co-I}, 2 años, US\$77,000)\\
\itemdate{2019} FONDECYT Iniciaci\'on (\textbf{PI}, 3 años, US\$125,000)


\subsectitle{Propuestas de observación exitosas (como PI)}

\noindent
He liderado 12 propuestas de observación exitosas totalizando cientos 
de horas de observación en telescopios ópticos (Gemini-South/GMOS, 
VST/OmegaCAM), Infrarrojo cercano (Magellan/Fourstar), submilimétrico 
(APEX/CONCERTO) y radio (GMRT, VLA).\\

\technical{Experiencia observando}
{He observado aproximadamente 200 horas con instrumentos ópticos (Gemini 
South/GMOS) e infrarrojos (NTT/SofI, Magellan/Fourstar), realizando tanto 
fotometría como espectroscopía de galaxias y cúmulos de galaxias.}


%\sectitle{Idiomas}
%
%Castellano nativo, inglés fluido y holandés básico.


\sectitle{Actividad comunitaria}

\noindent\textbf{Revistas científicas}: he sido evaluador de artículos para Astronomy 
    \& Astrophysics, The Astrophysical Journal, Monthly Notices of the Royal 
    Astronomical Society, y Nature Astronomy.\\
\noindent\textbf{Comités de evaluación de telescopios (TACs):} he sido evaluador de 
    proyectos de observación para el TAC Canadiense y para el Observatorio de rayos 
    X \textit{Chandra}.

\subsectitle{Cursos informales}

\noindent
\itemdate{2016} \emph{Mejores figuras científicas}, Universiteit Leiden
(más información \href{https://home.strw.leidenuniv.nl/~kenworthy/teaching/better_figures/}{aquí})

\subsectitle{Artículos de prensa escritos}

\noindent
\emph{Galaxy clusters: Falling into line} (Nature Astronomy \emph{News \& 
Views}, July 2017)\\
\emph{Dynamical masses of galaxy clusters discovered with the Sunyaev-Zel'dovich 
effect} (Gemini Focus \emph{Featured Science}, July 2013)

\subsectitle{Actividades de Divulgación}

\noindent
\itemdate{2021} Charla pública virtual en el contexto del Día de la 
Astronomía.\\
\itemdates{2018}{2019} \emph{Observaciones Astronómicas públicas
en Castellano}, Universidad de Princeton.\\
\itemdates{2013}{2014} \emph{Observaciones Astronómicas públicas
en el Observatorio Antiguo}, Universidad de Leiden.\\
\itemdate{2012} Curso \emph{Astronomía para adultos mayores}, PUC.\\
\itemdate{2011} \emph{Noches estrelladas}, observaciones astronómicas para 
escolares en riesgo social organizado por ESO.\\
\itemdate{2010} Charla invitada a bordo del buque de la Armada Chilena ``FFG14 
Almirante Latorre'', Valpara\'iso, Chile.\\
\itemdate{2010} \emph{El Universo}, serie de charlas para escolares en riesgo 
social organizada por PUC.\\


\sectitle{Habilidades técnicas}

Soy un programador experimentado en \textit{Python} y tengo experiencia con 
\texttt{IRAF/PyRAF}. Soy uno de los desarrolladores y mantenedores del código de 
análisis de lentes gravitacionales utilizado por la colaboración KiDS, que está 
escrito en \texttt{Python} pero que a la fecha no está disponible públicamente. 
Todos mis programas y códigos están disponibles en mi sitio de 
\href{https://github.com/cristobal-sifon}{\texttt{github}}.\\


%\vspace{1cm}
\hline
%\hline
%\vspace{0.5cm}
%\pagebreak


\sectitle{Otra experiencia de trabajo}

\itemdates{2007}{2008} Instructor de esquí en Homewood Mountain Ski Resort en 
Lake Tahoe, California, EE.UU. Obtuve certiciación como \emph{Instructor de 
Esquí Nivel I} otorgado por los Instructores de Esquí Profesionales de Estados 
Unidos (PSIA).\\
\itemdates{2006}{2007} Operador de andariveles en el centro de 
esquí Sun Valley Resort en Idaho, EE.UU.\\

%\hline
%\pagebreak

\sectitle{Referencias}

\begin{itemize}
%  \item 
 \item Prof.~Henk Hoekstra (\textit{Supervisor PhD})\\
       Leiden Observatory, Universiteit Leiden\\
       Niels Bohrweg 2, NL-2333 CA Leiden, The Netherlands\\
       Phone: +31 (71) 527 5594\\
       E-mail: \email{hoekstra@strw.leidenuniv.nl}
 \item Prof.~David N.~Spergel\\
       Center for Computational Astrophysics, Flatiron Institute\\
       160 Fifth Avenue, 7th Floor, New York, NY 10010, USA\\
       Phone: +1 (646) 654 0066\\
       E-mail: \email{dns@astro.princeton.edu}
 \item Prof.~John P.~Hughes\\
       Department of Physics and Astronomy, Rutgers University\\
       136 Frelinghuysen Rd., Piscataway, NJ 08854, USA\\
       Phone: +1 (848) 445 8878\\
       E-mail: \email{jph@physics.rutgers.edu}
 \item Prof.~L.~Felipe Barrientos (\textit{Supervisor MSc})\\
       Instituto de Astrof\'isica, P. Universidad Cat\'olica de Chile\\
       Casilla 306, Santiago 22, Chile\\
       Phone: +56 (2) 2354 4941\\
       E-mail: \email{barrientos@astro.uc.cl}
 \item Prof.~Felipe Menanteau\\
       Department of Astronomy, University of Illinois at Urbana-Champaign\\
       1002 W.\ Green St., Urbana, IL 61801, USA\\
       Phone: +1 (217) 244 6297\\
       E-mail: \email{felipe@illinois.edu}
\end{itemize}

\vspace{0.3cm}
\hline


%%%%%%%%%%%%%%%%%%%%%%%%%%%%%%%%%%%%%%
%%%%%%%%%%%%%%%%%%%%%%%%%%%%%%%%%%%%%%
%%%%%%%%%% PUBLICATION LIST %%%%%%%%%%
%%%%%%%%%%%%%%%%%%%%%%%%%%%%%%%%%%%%%%
%%%%%%%%%%%%%%%%%%%%%%%%%%%%%%%%%%%%%%

%% Uncomment these two lines to get the publication list in the main pdf

%\pagebreak
%\title{Selected recent publications {\small (All including \myself)}}


\noindent
I have co-authored 165 scientific articles intended for peer-reviewed 
publication, including 9 first-author papers. They have been cited more than 
9,900 times, with more than 400 citations on my 
first-author papers. The full list of publications can be accessed at the 
\href{https://goo.gl/LAu9G4}{SAO/NASA Astrophysics Data System}.
%
This document is maintained live on
\href{https://github.com/cristobal-sifon/cv/blob/master/Sifon_publications.pdf}{\texttt{github}}.




\begin{etaremune}

\item
M.~Shirasaki, \myself, and 14 colleagues,
\paper{Masses of Sunyaev-Zel'dovich Galaxy Clusters Detected by The Atacama Cosmology
Telescope: Stacked Lensing Measurements with Subaru HSC Year 3 data}
2024, \href{https://ui.adsabs.harvard.edu/abs/2024arXiv240708201S/abstract}{arXiv:2407.08201},
\accepted{\prd}

\item
\myself\ and J.~Han,
\paper{The history and mass content of cluster galaxies in the EAGLE simulation},
2024, \href{https://ui.adsabs.harvard.edu/abs/2024A&A...686A.163S/abstract}{\aap, 686, A163},
\arxiv{2312.12529}
   
\item
N.~C.~Robertson, \myself, and 23 colleagues,
\paper{ACT-DR5 Sunyaev-Zel’dovich Clusters: Weak Lensing Mass Calibration with KiDS},
2024, \href{https://ui.adsabs.harvard.edu/abs/2024A&A...681A..87R/abstract}{\aap, 681, 87}
\arxiv{2304.10219}

\item
W. Coulton, and 153 colleagues,
\paper{Atacama Cosmology Telescope: High-resolution component-separated maps across one third of the sky},
2024, \href{https://ui.adsabs.harvard.edu/abs/2024PhRvD.109f3530C}{\prd, 109, 063530}
\arxiv{2307.01258}

\item
M. S. Madhavacheril, and 158 colleagues,
\paper{The Atacama Cosmology Telescope: DR6 Gravitational Lensing Map and Cosmological Parameters},
2024, \href{https://ui.adsabs.harvard.edu/abs/2024ApJ...962..113M}{\apj, 962, 113}
\arxiv{2304.05203}

\item
F. J. Qu, and 157 colleagues,
\paper{The Atacama Cosmology Telescope: A Measurement of the DR6 CMB Lensing Power Spectrum and Its Implications for Structure Growth},
2024, \href{https://ui.adsabs.harvard.edu/abs/2024ApJ...962..112Q}{\apj, 962, 112}
\arxiv{2304.05202}

\item
Dark Energy Survey and Kilo-Degree Survey Collaborations, and 160 colleagues,
\paper{DES Y3 + KiDS-1000: Consistent cosmology combining cosmic shear surveys},
2023, \href{https://ui.adsabs.harvard.edu/abs/2023OJAp....6E..36D}{The Open Journal of Astrophysics, 6, 36}
\arxiv{2305.17173}

\item
M.~Hilton, \myself, and 133 colleagues
\paper{The Atacama Cosmology Telescope: a Catalog of $>$4000 Sunyaev-Zel’dovich 
Galaxy Clusters},
2021, \href{https://ui.adsabs.harvard.edu/abs/2021ApJS..253....3H/abstract}{\apjs, 253, 3}
\arxiv{2009.11043}

\item
M. Aguena, and 24 colleagues,
\paper{CLMM: a LSST-DESC cluster weak lensing mass modeling library for cosmology},
2021, \href{https://ui.adsabs.harvard.edu/abs/2021MNRAS.508.6092A}{\mnras, 508, 6092}
\arxiv{2107.10857}

\item
J. Kim, M. J. Jee, J. P. Hughes, M. Yoon, K. HyeongHan, F. Menanteau, \myself, L. Hovey, and P. Arunachalam,
\paper{Head-to-Toe Measurement of El Gordo: Improved Analysis of the Galaxy Cluster ACT-CL J0102$-$4915 with New Wide-Field Hubble Space Telescope Imaging Data},
2021, \href{https://ui.adsabs.harvard.edu/abs/2021ApJ...923..101K}{\apj, 923, 101}
\arxiv{2106.00031}

\item
M. Mallaby-Kay, and 59 colleagues,
\paper{The Atacama Cosmology Telescope: Summary of DR4 and DR5 Data Products and Data Access},
2021, \href{https://ui.adsabs.harvard.edu/abs/2021ApJS..255...11M}{\apjs, 255, 11}
\arxiv{2103.03154}

\item
M.~S.~Madhavacheril, \myself, and 61 colleagues
\paper{The Atacama Cosmology Telescope: Weighing Distant Clusters with the Most 
Ancient Light},
2020, \href{https://ui.adsabs.harvard.edu/abs/2020ApJ...903L..13M/abstract}{\apjl, 903, 13}
\arxiv{2009.07772}

\item
R.~Herbonnet, \myself, H.~Hoekstra, Y.~Bah\'e, R.~F.~J.~van~der~Burg, 
J.-B.~Melin, A.~von~der~Linden, D.~Sand, S.~Kay, D.~Barnes,
\paper{CCCP and MENeaCS: (Updated) Weak-Lensing Masses for 100 Galaxy Clusters},
2020, \href{https://ui.adsabs.harvard.edu/abs/2020MNRAS.497.4684H/abstract}{\mnras, 497, 4684}
\arxiv{1912.04414}


\end{etaremune}



\end{document}

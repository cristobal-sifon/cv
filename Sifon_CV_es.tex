\documentclass[11pt]{article}


\usepackage{bm}
\usepackage{enumerate}
\usepackage{etaremune}
%\usepackage{fullpage}
\usepackage{graphicx}
\usepackage{hyperref}

\addtolength{\oddsidemargin}{-1in}
\addtolength{\evensidemargin}{-1in}
\addtolength{\textwidth}{2in}
\addtolength{\topmargin}{-1.3in}
\addtolength{\textheight}{2.2in}

% For links of references
\hypersetup{colorlinks,
  linkcolor=blue,
  filecolor=blue,
  urlcolor=blue,
  citecolor=blue}

\renewcommand{\familydefault}{\sfdefault}
\usepackage{chancery}

\newcommand\sectitle[1]{
  \hline
  \vspace{0.5cm}
  \noindent
  %\underline{
    \textbf{\Large #1}
  %}
  \\
  \vspace{-0.2cm}
}

\newcommand\subsectitle[1]{
  \vspace{0.3cm}
  \noindent
  %\underline{
    \textbf{\large #1}
  %}
  \\
  \vspace{-0.3cm}
}

\newcommand\technical[2]{
  \noindent
    {\large\bf #1:} #2\\
  }

\newcommand\itemdate[1]{\textbf{[#1]}}
\newcommand\itemdates[2]{\textbf{[#1 -- #2]}}
\newcommand\email[1]{\href{mailto:#1}{\texttt{#1}}}
\newcommand\pucv{Pontificia Universidad Cat\'olica de Valpara\'iso}


%not used if publication list not shown

\def\aap{A\&A}
\def\apj{ApJ}
\def\apjs{ApJS}
\def\baas{\textit{Bull.\ of the Am.\ Ast.\ Soc.}}
\def\gemfoc{Gemini Focus}
\def\jcap{JCAP}
\def\mnras{MNRAS}
\def\msngr{The Messenger}
\def\pasj{PASJ}
\def\prd{Phys.\ Rev.\ D}
\def\prl{Phys.\ Rev.\ Letters}
\def\ssr{Space Sci.\ Rev.}

\newcommand{\myself}{\textbf{\color{red} C.~Sif\'on}}
\newcommand\includemyself{\textbf\small{(including C.~Sif\'on)}}
\newcommand{\accepted}[1]{accepted for publication in #1}
\newcommand{\etal}[1]{et al.\ (#1 co-authors),}
\newcommand{\etalwithme}[1]{et al.\ (#1 coauthors incl.\ \myself),}
\newcommand{\paper}[1]{\textbf{``#1''}}
\newcommand{\submitted}[1]{submitted to #1}
\renewcommand{\title}[1]{\noindent\textbf{\huge #1}\\}
% arXiv links
\newcommand{\arxiv}[1]
    {\href{https://arxiv.org/abs/#1}{\texttt{\color{magenta}[arXiv]}}}




\begin{document}

% \begin{figure}[t]
\begin{minipage}[b]{0.46\linewidth}
\flushleft
% \noindent
\hspace{-0.7cm}
{\bf\huge Crist\'obal Sif\'on}\\\vspace{0.2cm}
\hspace{-0.5cm}{\large Profesor Asociado\\
\hspace{-0.65cm}        Instituto de F\'isica, Facultad de Ciencias\\
\hspace{-0.65cm}        \pucv\\
\hspace{-0.65cm}        Casilla 4059, Valpara\'iso, Chile}\\
\end{minipage}
\begin{minipage}[b]{0.49\linewidth}
\flushright
{\large E-mail: {\texttt cristobal.sifon@pucv.cl}\\
        Teléfono: +56 (32) 227 4698\\
        \url{http://fis.ucv.cl/csifon/}
        \url{https://github.com/cristobal-sifon/}}
\end{minipage}
\vspace{0.4cm}
\hline



\sectitle{Investigación}

Mi investigación se enfoca en la física de cúmulos de galaxias, incluyendo la 
relación entre observables y masa para análisis cosmológicos, evolución de 
galaxias en cúmulos y cúmulos en proceso de colisión, además de los 
alineamientos intrínsecos entre galaxias. Uso un número de herramientas y 
técnicas para estudiar estos fenómenos, incluyendo mediciones del efecto de 
lente gravitacional débil, fotometría y espectroscopía óptica e infrarroja, 
datos entregados por mapeos ópticos, observaciones del fondo cósmico de 
microondas y más recientemente simulaciones computacionales.

\vspace{0.5cm}
\technical{Colaboraciones Científicas}
{
 Atacama Cosmology Telescope (ACT) ---
 Canadian Cluster Comparison Project (CCCP) ---
 Galaxy Cluster Mass Reconstruction Project ---
 Kilo-Degree Survey (KiDS) ---
 Legacy Survey of Space and Time Dark Energy Science Collaboration (LSST-DESC) ---
 Multi-Epoch Nearby Cluster Survey (MENeaCS) ---
 Simons Observatory.
}


\sectitle{Empleos}

\noindent
\itemdates{2019}{Presente} Profesor Asociado, \pucv\ (PUCV), Chile\\
\itemdates{2016}{2019} Investigador Postdoctoral Asociado, Princeton University, EE.UU.


\subsectitle{Educación}

\noindent
\itemdates{2012}{2016} Ph.D.~Astrofísica, Universiteit Leiden, Países Bajos\\
\itemdates{2010}{2012} M.Sc.~Astrofísica, P.~Universidad Cat\'olica de Chile (PUC), Chile\\
\itemdates{2005}{2010} B.Sc.~Astronomía, P.~Universidad Cat\'olica de Chile, Chile\\


\sectitle{Docencia y Supervisión}

\subsectitle{Supervisión de Investigación de Estudiantes}

\itemdates{2021}{Presente} Lya Marmolejo, PUCV: Proyecto de investigación de 
verano y tesina de Licenciatura.\\
\itemdates{2020}{Presente} Camila Aros, PUCV: tesis de Magíster.\\
\itemdate{2020} Nicole Mej\'ia, Universidad Nacional Aut\'onoma de Honduras 
(Honduras): Proyecto de investigación de pregrado a través del Central American-Caribbean Bridge in Astrophysics Program
    (\href{https://cencabridgeastro.weebly.com/our-team.html}{URL}; duración: 4 meses).\\
\itemdate{2020} Felipe Jorquera, PUCV: Proyecto de investigación de verano.\\
\itemdates{2017}{2019} Naomi Robertson, Universidad de Oxford (Reino Unido): 
Co-supervisor de proyecto de tesis de doctorado.\\
\itemdate{2018} Malik Walker, Universidad de Princeton: Proyecto de 
investigación de verano y proyecto de investigación semestral de pregrado.\\
\itemdates{2013}{2014} Joshua Albert, Universidad de Leiden: Co-supervisor de 
proyecto de tesis de Magíster.

%\pagebreak

\subsectitle{Cursos dictados}

\noindent
\emph{Licenciatura en Física y Licenciatura en Física, mención Astronomía}

\itemdate{2021} Astronomía Galáctica\\
\itemdate{2021,2020} Programación\\

\noindent
\emph{Magíster y Doctorado en Física}

\itemdate{2021,2020} Cosmología Observacional

\subsectitle{Comités de tesis}

\itemdate{2021} Camila Varas, Licenciatura en Astronomía, PUC\\


\sectitle{Financiamiento}

\noindent
\itemdate{2020} Fondo ALMA-ANID para contratar un investigador postdoctoral (\textbf{Co-I}, 2 años, US\$77,000)\\
\itemdate{2019} FONDECYT Iniciaci\'on (\textbf{PI}, 3 años, US\$125,000)


\subsectitle{Propuestas de observación exitosas (como PI)}

\noindent
He liderado 10 propuestas de observación exitosas totalizando cientos de horas 
de observación en telescopios ópticos (Gemini-South/GMOS, VST/OmegaCAM), 
Infrarrojo cercano (Magellan/Fourstar) y radio (GMRT, VLA).\\

\technical{Experiencia observando}
{He observado aproximadamente 200 horas con instrumentos ópticos (Gemini 
South/GMOS) e infrarrojos (NTT/SofI, Magellan/Fourstar), realizando tanto 
fotometría como espectroscopía de galaxias y cúmulos de galaxias.}


%\sectitle{Idiomas}
%
%Castellano nativo, inglés fluido y holandés básico.


\sectitle{Actividad comunitaria}

\noindent\textbf{Revistas científicas}: he sido evaluador de artículos para Astronomy 
    \& Astrophysics, The Astrophysical Journal, Monthly Notices of the Royal 
    Astronomical Society, and Nature Astronomy.\\
\noindent\textbf{Comités de evaluación de telescopios (TACs):} he sido evaluador de 
    proyectos de observación para el TAC Canadiense y para el Observatorio de rayos 
    X \textit{Chandra}.

\subsectitle{Cursos informales}

\noindent
\itemdate{2016} \emph{Mejores figuras científicas}, Universiteit Leiden
(más información \href{https://home.strw.leidenuniv.nl/~kenworthy/teaching/better_figures/}{aquí})

\subsectitle{Artículos de prensa escritos}

\noindent
\emph{Galaxy clusters: Falling into line} (Nature Astronomy \emph{News \& 
Views}, July 2017)\\
\emph{Dynamical masses of galaxy clusters discovered with the Sunyaev-Zel'dovich 
effect} (Gemini Focus \emph{Featured Science}, July 2013)

\subsectitle{Actividades de Divulgación}

\noindent
\itemdate{2021} Charla pública virtual en el contexto del Día de la 
Astronomía.\\
\itemdates{2018}{2019} \emph{Observaciones Astronómicas públicas
en Castellano}, Universidad de Princeton.\\
\itemdates{2013}{2014} \emph{Observaciones Astronómicas públicas
en el Observatorio Antiguo}, Universidad de Leiden.\\
\itemdate{2012} Curso \emph{Astronomía para adultos mayores}, PUC.\\
\itemdate{2011} \emph{Noches estrelladas}, observaciones astronómicas para 
escolares en riesgo social organizado por ESO.\\
\itemdate{2010} Charla invitada a bordo del buque de la Armada Chilena ``FFG14 
Almirante Latorre'', Valpara\'iso, Chile.\\
\itemdate{2010} \emph{El Universo}, serie de charlas para escolares en riesgo 
social organizada por PUC.\\


\sectitle{Habilidades técnicas}

Soy un programador experimentado en \textit{Python} y tengo experiencia con 
\texttt{IRAF/PyRAF}. Soy uno de los desarrolladores y mantenedores del código de 
análisis de lentes gravitacionales utilizado por la colaboración KiDS, que está 
escrito en \texttt{Python} pero que a la fecha no está disponible públicamente. 
Todos mis programas y códigos están disponibles en mi sitio de 
\href{https://github.com/cristobal-sifon}{\texttt{github}}.\\


%\vspace{1cm}
\hline
%\hline
%\vspace{0.5cm}
%\pagebreak


\sectitle{Otra experiencia de trabajo}

\itemdates{2007}{2008} Instructor de esquí en Homewood Mountain Ski Resort en 
Lake Tahoe, California, EE.UU. Obtuve certiciación como \emph{Instructor de 
Esquí Nivel I} otorgado por los Instructores de Esquí Profesionales de Estados 
Unidos (PSIA).\\
\itemdates{2006}{2007} Operador de andariveles en el centro de 
esquí Sun Valley Resort en Idaho, EE.UU.\\

%\hline
%\pagebreak

\sectitle{Referencias}

\begin{itemize}
%  \item 
 \item Prof.~Henk Hoekstra (\textit{Supervisor PhD})\\
       Leiden Observatory, Universiteit Leiden\\
       Niels Bohrweg 2, NL-2333 CA Leiden, The Netherlands\\
       Phone: +31 (71) 527 5594\\
       E-mail: \email{hoekstra@strw.leidenuniv.nl}
 \item Prof.~David N.~Spergel\\
       Center for Computational Astrophysics, Flatiron Institute\\
       160 Fifth Avenue, 7th Floor, New York, NY 10010, USA\\
       Phone: +1 (646) 654 0066\\
       E-mail: \email{dns@astro.princeton.edu}
 \item Prof.~John P.~Hughes\\
       Department of Physics and Astronomy, Rutgers University\\
       136 Frelinghuysen Rd., Piscataway, NJ 08854, USA\\
       Phone: +1 (848) 445 8878\\
       E-mail: \email{jph@physics.rutgers.edu}
 \item Prof.~L.~Felipe Barrientos (\textit{Supervisor MSc})\\
       Instituto de Astrof\'isica, P. Universidad Cat\'olica de Chile\\
       Casilla 306, Santiago 22, Chile\\
       Phone: +56 (2) 2354 4941\\
       E-mail: \email{barrientos@astro.uc.cl}
 \item Prof.~Felipe Menanteau\\
       Department of Astronomy, University of Illinois at Urbana-Champaign\\
       1002 W.\ Green St., Urbana, IL 61801, USA\\
       Phone: +1 (217) 244 6297\\
       E-mail: \email{felipe@illinois.edu}
\end{itemize}

\vspace{0.3cm}
\hline


%%%%%%%%%%%%%%%%%%%%%%%%%%%%%%%%%%%%%%
%%%%%%%%%%%%%%%%%%%%%%%%%%%%%%%%%%%%%%
%%%%%%%%%% PUBLICATION LIST %%%%%%%%%%
%%%%%%%%%%%%%%%%%%%%%%%%%%%%%%%%%%%%%%
%%%%%%%%%%%%%%%%%%%%%%%%%%%%%%%%%%%%%%

%% Uncomment these two lines to get the publication list in the main pdf

%\pagebreak
%\title{Recent Publications}


\noindent
I have co-authored 80 scientific articles intended for peer-reviewed 
publication, including 7 first-author papers. They have been cited more than 
3,700 times and have an $h$-index of 35, with more than 300 citations on my 
first-author papers. My publications include three companion reviews on galaxy 
alignments written for a special issue of Space Science Reviews (B.\ Joachimi et 
al.\ 2015, A.\ Kiessling et al.\ 2015, D.\ Kirk et al.\ 2015). The full list of 
publications can be accessed at \href{https://goo.gl/LAu9G4}{this url}. I also 
wrote an invited `News \& Views' article for the 4 July 2017 edition of Nature 
Astronomy, accessible 
\href{https://www.nature.com/articles/s41550-017-0181}{here}.
%
This document is maintained live on
\href{https://github.com/cristobal-sifon/cv/blob/master/Sifon_publications.pdf}{\texttt{github}}.




\begin{etaremune}

\item
B.~Fuzia \etalwithme{22}
\paper{The Atacama Cosmology Telescope: SZ-Based Masses and Dust Emission from IR-Selected Cluster Candidates in the SHELA Survey},
2020, \href{https://ui.adsabs.harvard.edu/abs/2020arXiv200109587F/abstract}{arXiv:2001.09587},
\submitted{\mnras}

\item
S.~Huang \etalwithme{12}
%A.~Leauthaud, A.~Hearin, P.~Behroozi, C.~Bradshaw, F.~Ardila, J.~Speagle,
%A.~Tenneti, K.~Bundy, J.~Greene, \myself, N.~Bahcall,
\paper{Weak Lensing Reveals a Tight Connection Between Dark Matter Halo Mass and the Distribution of Stellar Mass in Massive Galaxies},
2020, \href{https://ui.adsabs.harvard.edu/abs/2020MNRAS.492.3685H/abstract}{\mnras, 492, 3685}
\arxiv{1811.01139}

\item
Q.~Xia \etalwithme{13}
\paper{A Gravitational Lensing Detection of Filamentary Structures Connecting Luminous Red Galaxies},
2020, \href{https://ui.adsabs.harvard.edu/abs/2020A&A...633A..89X/abstract}{\aap, 633, 89}
\arxiv{1909.05852}

\item
H.~Hildebrandt \etalwithme{28}
\paper{KiDS+VIKING-450: Cosmic Shear Tomography with Optical+infrared Data},
2020, \href{https://ui.adsabs.harvard.edu/abs/2020A&A...633A..69H/abstract}{\aap, 633, 69}
\arxiv{1812.06076}

\item
R.~Herbonnet, \myself, H.~Hoekstra, Y.~Bah\'e, R.~F.~J.~van~der~Burg, J.-B.~Melin, A.~von~der~Linden, D.~Sand, S.~Kay, D.~Barnes,
\paper{CCCP and MENeaCS: (Updated) Weak-Lensing Masses for 100 Galaxy Clusters},
2019, \href{https://ui.adsabs.harvard.edu/abs/2019arXiv191204414H/abstract}{arXiv:1912.04414},
\submitted{\mnras}

\item
M.~Madhavacheril \etalwithme{49}
\paper{The Atacama Cosmology Telescope: Component-Separated Maps of CMB Temperature and the Thermal Sunyaev-Zel'dovich Effect},
2019, \href{https://ui.adsabs.harvard.edu/abs/2019arXiv191105717M/abstract}{arXiv:1911.05717},
\submitted{\prd}

\item
Y.~Rong \etalwithme{13}
\paper{Intrinsic Morphology Evolution of Ultra-diffuse Galaxies},
2019, \href{https://ui.adsabs.harvard.edu/abs/2019arXiv190710079R/abstract}{arXiv:1907.10079},
\submitted{\apj}

\item
C.~Hikage \etalwithme{30}
\paper{Cosmology from Cosmic Shear Power Spectra with Subaru Hyper Suprime-Cam First-Year Data},
2019, \href{http://adsabs.harvard.edu/abs/2019PASJ...71...43H}{\pasj, 71, 43}
\arxiv{1809.09148}

\item
H.~Miyatake \etalwithme{58}
\paper{Weak-Lensing Mass Calibration of ACTPol Sunyaev-Zel'dovich Clusters with the Hyper Suprime-Cam Survey},
2019, \href{http://adsabs.harvard.edu/abs/2019ApJ...875...63M}{\apj, 875, 63}
\arxiv{1804.05873}

\item
K.~Knowles \etalwithme{14}
\paper{GMRT 610 MHz Observations of Galaxy Clusters in the ACT Equatorial Sample},
2019, \href{http://adsabs.harvard.edu/abs/2019MNRAS.486.1332K}{\mnras, 486, 1332}
\arxiv{1806.09579}

\item
\myself, R.~Herbonnet, H.~Hoekstra, R.~F.~J.~van~der~Burg, M.~Viola,
\paper{The Galaxy-Subhalo Connection in Low-Redshift Galaxy Clusters from Weak Gravitational Lensing},
2018, \href{http://adsabs.harvard.edu/abs/2018MNRAS.478.1244S}{\mnras, 478, 1244}
\arxiv{1706.06125}

\item
\myself, R.~F.~J.~van~der~Burg, H.~Hoekstra, A.~Muzzin, R.~Herbonnet,
\paper{A First Constraint on the Average Mass of Ultra Diffuse Galaxies from Weak Gravitational Lensing},
2018, \href{http://adsabs.harvard.edu/abs/2018MNRAS.473.3747S}{\mnras, 473, 3747}
\arxiv{1704.07847}

\item
M.~Hilton, M.~Hasselfield, \myself, \etal{43}
\paper{The Atacama Cosmology Telescope: The Two-Season ACTPol Sunyaev-Zel'dovich Effect Selected Cluster Catalog},
2018, \href{http://adsabs.harvard.edu/abs/2018ApJS..235...20H}{\apjs, 235, 20}
\arxiv{1709.05600}


\end{etaremune}



\end{document}
